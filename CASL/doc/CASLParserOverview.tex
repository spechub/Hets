\documentclass{article}

\parindent 0pt
\parskip 5pt

\begin{document}

\title{CASL basic items}

\author{C. Maeder}

\maketitle

\section{Preliminaries}

see the documentation for HetCATS and Common

\section{Compiling the CASL parser}

\texttt{cd HetCATS/CASL} and call \texttt{./ghc-call} (a script that contains
all necessary ghc options and parameters). An alternative is to call
\texttt{gmake capa} in the top directory.

This creates a binary called \texttt{capa} that can parse (and pretty
print) ``Basic Specification with Subsorts'' of the CASL Summary
version 1.0.1 \\ (www.brics.dk/Projects/CoFI/Documents/CASL/Summary/).

Structured specifications (see \texttt{Static/hetpa.hs}) must be parsed by the
heterogenous parser.

Note: if module \texttt{AS\_Basic\_CASL} can not be found, then
\texttt{AS\_Basic\_CASL.hs} was not ``drifted'' (i.e., derived from
\texttt{AS\_Basic\_CASL.der.hs}). 

Also \texttt{Common/AS\_Annotation.hs} must have been drifted.

\section{Testing}

A test call might be: 

\texttt{./capa < test/BasicSpec.casl}

The binary \texttt{capa} is also used to test various parsers. Test cases are
given as \texttt{*.casl} files in the \texttt{test} subdirectory. Calling
\texttt{./runcheck.sh ../capa} performs many tests and compares the results
with corresponding \texttt{*.output} files.

Alternatively a test can be run using \texttt{runhugs}. For this the
file \texttt{capa.lhs} is executable and the test can be run
by \texttt{./runcheck.sh ../capa.lhs}.

\texttt{Wrong*.casl} files contain wrong CASL code and should produce
error messages. Apart from extra tests, correct (\texttt{Bla.casl})
and wrong files (\texttt{WrongBla.casl}) are tested. If a \texttt{diff}
with the expected output fails, then also the number of
``\texttt{error}'' occurrences (in the produced output) is counted. For a
successful comparison a ``\texttt{passed}'' is emitted.

Calling ``\texttt{./runcheck.sh ../capa set}'' will update the
\texttt{*.output} files (and a subsequent \texttt{./runcheck.sh ../capa} should
always pass). 

Alternatively, all checks can be preformed by \texttt{make check}. Output
files will be overwritten by \texttt{make output}. With \texttt{cvs up} or
\texttt{cvs diff} changes can be compared to checked-in versions of the output
files.

\section{Source files in \texttt{HetCATS/CASL}}

\begin{description}
\item[AS\_Basic\_CASL.der.hs] defines the abstract syntax tree for CASL
\item[Formula.hs] parsers \texttt{term} and \texttt{formula}
\item[Latin.hs] used by \texttt{Static.hs}
\item[LiteralFuns.hs] used by \texttt{Print\_AS\_Basic.hs}
\item[Logic\_CASL.hs] contains the instance for the class \texttt{Logic.Logic}
\item[MixfixParser.hs] uses precedence and associativity to resolve
  mixfix terms (and formulae). Also \%list, \%string, \%number and
  \%floating annotations are resolved
\item[Morphism.hs] symbol and morphism stuff for \texttt{Logic}
\item[OpItem.hs] parsers \texttt{opItem} and \texttt{predItem}
\item[Parse\_AS\_Basic.hs] supplies the top-level parsers \texttt{basicSpec,
  basicItems, dotFormulae} and \texttt{sigItems}
\item[Print\_AS\_Basic.hs] pretty prints data types of
  \texttt{AS\_Basic\_CASL.hs}
\item[RunMixfixParser.hs] is a driver to test the mixfix
  analysis
\item[RunStaticAna.hs] is an additional driver to test the static 
  analysis (call \texttt{capa analysis < file})  
\item[Sign.hs] (unused) was the analysed abstract syntax 
\item[SortItem.hs] parsers \texttt{sortItem} and \texttt{datatype} (requires
  \texttt{Formula.hs} for subsort definitions)
\item[Static.hs] (unused) implements the static analysis
\item[StaticAna.hs] implements a simple static analysis
\item[Sublogic.hs] used by \texttt{Logic\_CASL.hs}
\item[Sublogics.hs] older version
\item[SymbolAnalysis] (unused) 
\item[SymbolParser.hs] parses symbols and symbol maps that are not needed for
  basic specs but for heterogeneous structured specifications
  (\texttt{Logic\_CASL.hs}). These parsers do not deal with annotations,
  currently
\item[capa.lhs] is the main module that simply lists the test parsers
  and is executable for hugs
\end{description}


\section{Remarks}

The \texttt{MixfixParser} only operates on \texttt{FORMULA} and
\texttt{TERM} from \texttt{AS\_Basic\_CASL} and not on
types in \texttt{Sign.hs}. 

Type checking (considering subtypes and overloading) of terms and
formulae is still missing. The static analysis currently collects the bare
signature and the subsort relation.

\end{document}

