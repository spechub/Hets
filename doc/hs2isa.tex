% File:    kri0.tex
% Author:  Paolo Torrini <paolot@helios.dai.ed.ac.uk>
% Created: Mon Feb 15 1999

\documentclass{llncs}
%%\documentclass{/home/polh/Documents/Hs2Isa/llcns}
%%\documentstyle{/home/polh/Documents/Hs2Isa/llcns}
%%\documentclass[a4paper,12pt]{article}

\usepackage[dvips]{graphics}
\usepackage{amsfonts}
\usepackage{amssymb}
%\usepackage{psfig}
\usepackage{fancybox}
\usepackage{epsfig}
\usepackage{graphicx}
\usepackage{color}
%\usepackage{semrot}

%% == Definitions of frequently used symbolic terms




% MORE MACROS

% ENVIRONMENTS

\newcounter{proposition}[subsection]
\renewcommand{\theproposition}
{\arabic{section}.\arabic{subsection}:\arabic{proposition}}

\newcounter{dee}[section]
\renewcommand{\thedee}
{\arabic{section}.\arabic{dee}}






\newcounter{definition}
\renewcommand{\thedefinition}
{\arabic{definition}}

\newcounter{theorem}
\renewcommand{\thetheorem}
{\arabic{theorem}}

\newcounter{observation}
\renewcommand{\theobservation}
{\arabic{observation}}

\newcounter{lemma}
\renewcommand{\thelemma}
{\arabic{lemma}}


% short env.s

\newcounter{ShortProp}
\renewcommand{\theShortProp}
{\arabic{ShortProp}}

%\newcommand{\sprop}{\noindent\refstepcounter{ShortProp}\textbf{Proposition\the%ShortProp}. \ }

\newcommand{\sprop}{\noindent\refstepcounter{ShortProp}\textbf{Lemma\theShortProp}. \ }

\newcounter{ShortDef}
\renewcommand{\theShortDef}
{\arabic{ShortDef}}

\newcommand{\propref}[1]{\noindent Lemma \ref{#1}}

\newcommand{\sdef}{\noindent\refstepcounter{ShortDef}\textbf{Definition\theShortDef}. \ }

\newcounter{IntAx}
\renewcommand{\theIntAx}
{\arabic{IntAx}}

\newcommand{\defref}[1]{\noindent Definition \ref{#1}}

\newcommand{\intax}{\noindent\refstepcounter{IntAx}\textbf{A\theIntAx}. \ }

\newcounter{ClAx}
\renewcommand{\theClAx}
{\arabic{ClAx}}

\newcommand{\clax}{\noindent\refstepcounter{ClAx}\textbf{A'\theClAx}. \ }


\newcounter{JAx}
\renewcommand{\theJAx}
{\arabic{JAx}}

%\newcommand{\Jax}{\noindent\refstepcounter{JAx}\textbf{J\theJAx}. \ }
\newcommand{\Jax}{\noindent\refstepcounter{JAx}\textbf{A\theJAx}. \ }


%end short env.s


%\newcounter{corollary}
%\renewcommand{\thecorollary}
%{\arabic{corollary}}

\newenvironment{defn}
{\refstepcounter{definition}%
 \begin{description}%
 \item[\textbf{Definition} 
       \thedefinition] \mbox{ }}%
{\end{description}}

\newenvironment{defA}
{\refstepcounter{definition}%
 \begin{description}%
 \item[\textbf{Def.} 
       \thedefinition]}%
{\end{description}}

\newenvironment{thmA}
{\refstepcounter{theorem}%
 \begin{description}%
 \item[\textbf{Theorem} 
       \thetheorem] \mbox{ }}%
{\end{description}}

\newenvironment{obsA}
{\refstepcounter{observation}%
 \begin{description}%
 \item[\textbf{Obs.} 
       \theobservation]}%
{\end{description}}

\newenvironment{lemA}
{\refstepcounter{lemma}%
 \begin{description}%
 \item[\textbf{Lem.} 
       \thelemma]}%
{\end{description}}


\newenvironment{jprop}
{\refstepcounter{lemma}%
 \begin{description}%
 \item[\textbf{Proposition} 
       \thelemma] \mbox{ }}%
{\end{description}}

\newenvironment{pf}
{\begin{description}%
 \item[\textbf{pf: }]}%
{\end{description}}

\newenvironment{Prop}
{\refstepcounter{proposition}%
 \begin{description}%
 \item[\theproposition]}%
{\end{description}}

\newenvironment{AProp}
{\refstepcounter{proposition}%
 \begin{description}%
 \item[\textbf{proposition} 
       \theproposition]}%
{\end{description}}

\newenvironment{BProp}
{\refstepcounter{proposition}%
 \begin{description}%
 \item[\textbf{definition} 
       \theproposition]}%
{\end{description}}

\newenvironment{CCProp}
{\refstepcounter{proposition}%
 \begin{description}%
 \item[\theproposition]}%
{\end{description}}


\newenvironment{CProp}
{\refstepcounter{dee}%
 \begin{description}%
 \item[\thedee]}%
{\end{description}}

\newenvironment{CdProp}
{\refstepcounter{dee}%
 \begin{description}%
 \item[\thedee \ \textbf{D}]}%
{\end{description}}


\newenvironment{DProp}[1]
{\refstepcounter{proposition}%
 \begin{description}%
 \item[\text{#1} \ ]}%
{\end{description}}

\newenvironment{FProp}[1]
{\refstepcounter{proposition}%
 \begin{description}%
 \item[\textbf{#1} \theproposition]}%
{\end{description}}


\newcounter{Pproposition}

\newenvironment{PProp}
{\refstepcounter{Pproposition}%
 \begin{description}%
 \item[\thePproposition]}%
{\end{description}}


\newtheorem{prop}{Proposition}

\newtheorem{definit}[prop]{Definition}

\newtheorem{corol}[prop]{Corollary}


% COMMANDS

\newcommand{\escmathcom}[3]{ \newcommand{#1}[#2]{\mbox{#3}}}


% OTHER MACROS

% Set theory

%\escmathcom{\eq}{0}{$\ \equiv \ $}
\escmathcom{\eq}{0}{$ \ = \ $}

\escmathcom{\tset}{1}{$\mathcal{#1}$}  % short version
\escmathcom{\tindset}{2}{${\mathcal{#2}}_{#1}$}
\escmathcom{\tindtop}{2}{${\mathcal{#1}}_{\mathcal{#2}}$}
\escmathcom{\powerset}{1}{$\wp (#1)$}

\escmathcom{\mi}{1}{$\mathit{#1}$}
\escmathcom{\ms}{1}{$\mathsf{#1}$}
\escmathcom{\wt}{1}{$\mathtt{#1}$}
\escmathcom{\mcal}{1}{$\mathcal{#1}$}
\escmathcom{\mr}{1}{$\mathcal{#1}$}



% Models and Structures

\escmathcom{\Struct}{1}{$\mathcal{#1}$}
\escmathcom{\AStruct}{1}{$\mathfrak{#1}$}
\escmathcom{\BStruct}{1}{$\mathfrak{#1}_{\rho}$}


% Lattice theory

\escmathcom{\true}{0}{$\top$}
\escmathcom{\false}{0}{$\bot$}
\escmathcom{\porder}{0}{$\leq$}
\escmathcom{\sorder}{1}{$\sqsubset_{\mathcal{#1}}$}
%\escmathcom{\specord}{0}{$\sqsubset_{\mbox{\tiny $\Opens$}}$}
%\escmathcom{\specord}{0}{$\prec_{\mbox{\tiny $\Opens$}}$}
\escmathcom{\specord}{0}{$\prec$} 
\escmathcom{\rspecord}{0}{$\prec_{\mbox{\tiny $\Regions$}}$}
\escmathcom{\finer}{0}{$\ll$}
\escmathcom{\smeet}{0}{$ \ \sqcap \ $}
\escmathcom{\sjoin}{0}{$ \ \sqcup \ $}
\escmathcom{\meet}{0}{$\sqcap$}
\escmathcom{\join}{0}{$\sqcup$}
\escmathcom{\ameet}{0}{$\bigwedge$}
\escmathcom{\ajoin}{0}{$\bigvee$}
\escmathcom{\bigjoin}{0}{$\bigvee$}
\escmathcom{\diimp}{0}{$\Leftrightarrow$}
\escmathcom{\upclosedA}{1}{$\{ \wt{#1} \upharpoonright \}$}
\escmathcom{\upclosed}{1}{$\{ \wt{#1} \uparrow \}$}
\escmathcom{\pointset}{1}{$ \wt{#1} \uparrow_{\mbox{\tiny $\Regions$}}$}

% Logical symbols

\escmathcom{\niimp}{0}{$\nRightarrow$}
\escmathcom{\nto}{0}{$\nrightarrow$}
\escmathcom{\bnot}{0}{$\neg$}

\escmathcom{\divc}{0}{$\Vert$}
\escmathcom{\diva}{0}{$\Vert$}



% Spatial IPL2 (heyt13)


\escmathcom{\retre}{0}{$\mathsf{res}$}
\escmathcom{\exte}{0}{$\mathsf{ext}$}
\escmathcom{\inst}{0}{$\mathsf{inst}$}
\escmathcom{\recoOne}{0}{$\mathsf{rec1}$}
\escmathcom{\recoTwo}{0}{$\mathsf{rec2}$}

\escmathcom{\wff}{0}{$\mathsf{wff}$}
\escmathcom{\terM}{0}{$\mathsf{term}$}

\escmathcom{\iovl}{0}{$\mathsf{overlap}_{i}$}
\escmathcom{\overlapping}{0}{$\mathsf{overlap}$}
\escmathcom{\ovl}{0}{$\mathsf{overlap}$}
%\escmathcom{\tp}{0}{$\mathsf{tp}$}
\escmathcom{\src}{0}{$\mathsf{ic}_{s}$}
\escmathcom{\wrc}{0}{$\mathsf{ic}_{w}$}
\escmathcom{\sdc}{0}{$\mathsf{ap}_{s}$}
\escmathcom{\wdc}{0}{$\mathsf{ap}_{w}$}
\escmathcom{\isrc}{0}{$\mathsf{ic}_{i}$}
\escmathcom{\tp}{0}{$\mathsf{tp}$}
\escmathcom{\stp}{0}{$\mathsf{tp}_{s}$}
\escmathcom{\istp}{0}{$\mathsf{tp}_{i}$}
\escmathcom{\Regular}{0}{$\mathsf{regl}$}



%\escmathcom{\spc}{0}{$\mathsf{spc}$}
%\escmathcom{\Tspc}{0}{$\mathsf{SPC}$}

\escmathcom{\Models}{0}{$\mathsf{\Vdash}$}

\escmathcom{\screen}{0}{$\mathsf{screen}$}
\escmathcom{\Alpha}{0}{$\Lambda$}
\escmathcom{\cell}{0}{$\mathsf{cell}$}
\escmathcom{\disjoint}{0}{$\mathsf{disjoint}$}
\escmathcom{\rprime}{0}{$\mathsf{prime}$}
\escmathcom{\atom}{0}{$\mathsf{atmcell}$}

\escmathcom{\Ascreen}{0}{$\mathsf{atoms}$}
\escmathcom{\Aatom}{0}{$\mathsf{atom}$}


\escmathcom{\twocover}{0}{$\mathsf{2cover}$}
\escmathcom{\primecomp}{0}{$\mathsf{pcomp}$}
\escmathcom{\concomp}{0}{$\mathsf{ccomp}$}
\escmathcom{\tessel}{0}{$\mathsf{tll}$}
\escmathcom{\diip}{0}{$\mathsf{diip}$}
\escmathcom{\doip}{0}{$\mathsf{doip}$}
\escmathcom{\diic}{0}{$\mathsf{diic}$}
\escmathcom{\doic}{0}{$\mathsf{doic}$}


\escmathcom{\spc}{0}{$\mathsf{con}_{s}$}
\escmathcom{\Tspc}{0}{$\mathsf{SC}$}
%\escmathcom{\mcon}{0}{$\mathsf{c}$}
\escmathcom{\connec}{0}{$\mathsf{con}$}
\escmathcom{\dc}{0}{$\mathsf{ap}$}
\escmathcom{\ntp}{0}{$\mathsf{ntp}$}
\escmathcom{\wtp}{0}{$\mathsf{tp}_{w}$}
\escmathcom{\wntp}{0}{$\mathsf{ntp}_{w}$}
\escmathcom{\ers}{0}{$\mathsf{ers}$}
\escmathcom{\rc}{0}{$\mathsf{ic}$}
\escmathcom{\cim}{0}{$\mathsf{cell\_of\_mesh}$}
%\escmathcom{\split}{0}{$\mathsf{nspl}$}
%\escmathcom{\Tsplit}{0}{$\mathsf{NSPL}$}
\escmathcom{\rcon}{0}{$\mathsf{con}$}
\escmathcom{\nsplit}{0}{$\mathsf{ndiv}$}
\escmathcom{\srcon}{0}{$\mathsf{con}_{s}$}
\escmathcom{\snsplit}{0}{$\mathsf{ndiv}_{s}$}
\escmathcom{\split}{0}{$\mathsf{ndiv}$}
\escmathcom{\regular}{0}{$\mathsf{regl}$}
%\escmathcom{\Ttriv}{0}{$\mathsf{triv}$}
\escmathcom{\triv}{0}{$\mathsf{degnr}$}
%\escmathcom{\Tnemp}{0}{$\mathsf{ne}$}
\escmathcom{\nempty}{0}{$\mathsf{nempt}$}
\escmathcom{\nesc}{0}{$\mathsf{nesc}$}
%\escmathcom{\pos}{0}{$\mathsf{pos}$}
\escmathcom{\pos}{0}{$\mathsf{nempt}$}
\escmathcom{\unispace}{0}{$\mathsf{unisp}$}



\escmathcom{\comp}{0}{$\circ$}

\escmathcom{\isgran}{0}{$\mathsf{is\_gran}$}
\escmathcom{\gran}{0}{$\mathsf{gran}$}
\escmathcom{\hasgran}{0}{$\mathsf{has\_gran}$}
\escmathcom{\maxgran}{0}{$\mathsf{maxgran}$}
\escmathcom{\yolk}{0}{$\mathsf{yolk}$}
\escmathcom{\Tsplit}{0}{$\mathsf{NS}$}
\escmathcom{\Trc}{0}{$\mathsf{RC}$}
\escmathcom{\Ters}{0}{$\mathsf{ERS}$}
\escmathcom{\Tntp}{0}{$\mathsf{NTP}$}
\escmathcom{\Tdc}{0}{$\mathsf{DC}$}
\escmathcom{\Tpos}{0}{$\mathsf{EXT}$}

%\escmathcom{\white}{0}{$\mathsf{white}$}

\escmathcom{\digcom}{0}{$\mathsf{dcomp}$}
\escmathcom{\digsup}{0}{$\mathsf{dsup}$}
\escmathcom{\inreg}{0}{$\mathsf{dint}$}
\escmathcom{\outreg}{0}{$\mathsf{dout}$}
\escmathcom{\digrep}{0}{$\mathsf{drep}$}
\escmathcom{\cpd}{0}{$\mathsf{cpd}$}

\escmathcom{\connectedA}{0}{$\mathsf{scon}$}
\escmathcom{\connectedB}{0}{$\mathsf{sc}$}
\escmathcom{\connectedC}{0}{$\mathsf{rsc}$}
\escmathcom{\rbo}{0}{$\mathsf{rbo}$}
\escmathcom{\rcccon}{0}{$\mathsf{RC}$}
\escmathcom{\rccover}{0}{$\mathsf{OL}$}
\escmathcom{\rccdisc}{0}{$\mathsf{DC}$}
\escmathcom{\rccdisj}{0}{$\mathsf{DJ}$}
\escmathcom{\srcccon}{0}{$\mathsf{RCsc}$}
\escmathcom{\srccover}{0}{$\mathsf{OLsc}$}
\escmathcom{\rccconN}{0}{$\mathsf{RCn}$}
\escmathcom{\rccoverN}{0}{$\mathsf{OLn}$}


% Spatial IPL2 (ARW2001)

\escmathcom{\Disj}{0}{$\mathsf{DJ}$}

\escmathcom{\Hpart}{0}{$\mathsf{p}$}
\escmathcom{\Hdc}{0}{$\mathsf{wdc}$}
\escmathcom{\Hdj}{0}{$\mathsf{dj}$}
\escmathcom{\Hcon}{0}{$\mathsf{spc}$}
\escmathcom{\Hsc}{0}{$\mathsf{spc}$}
\escmathcom{\Hwc}{0}{$\mathsf{wc}$}
\escmathcom{\Hdisc}{0}{$\mathsf{dsc}$}
\escmathcom{\Hol}{0}{$\mathsf{cov}$}
\escmathcom{\Hrc}{0}{$\mathsf{fmc}$}
%\escmathcom{\nemp}{0}{$\mathsf{cpos}$}
\escmathcom{\nemp}{0}{$\mathsf{nempty}$}

\escmathcom{\Wpart}{0}{$\mathsf{Part}$}
\escmathcom{\Wdc}{0}{$\mathsf{wDC}$}
\escmathcom{\Wdj}{0}{$\mathsf{DJ}$}
\escmathcom{\Wcon}{0}{$\mathsf{Con}$}
\escmathcom{\Wsc}{0}{$\mathsf{Con}$}
\escmathcom{\Wol}{0}{$\mathsf{OV}$}
\escmathcom{\Wrc}{0}{$\mathsf{fmC}$}
\escmathcom{\Wne}{0}{$\mathsf{NE}$}


%\escmathcom{\conn}{0}{$\odot$}
\escmathcom{\conn}{0}{$\Hcon$}
%\escmathcom{\nemp}{0}{$\triangledown$}
%\escmathcom{\nempt}{0}{$\triangledown$}



% Spatial intuitive relations (ARW2001 slides, fig.)

\escmathcom{\HHpart}{0}{$\mathsf{part\_of}$}
\escmathcom{\HHdc}{0}{$\mathsf{r\_discon}$}
\escmathcom{\HHdj}{0}{$\mathsf{disj}$}
\escmathcom{\HHcon}{0}{$\mathsf{s\_con}$}
\escmathcom{\HHdisc}{0}{$\mathsf{s\_discon}$}
\escmathcom{\HHol}{0}{$\mathsf{overlap}$}
\escmathcom{\HHrc}{0}{$\mathsf{r\_con}$}
\escmathcom{\HHreg}{0}{$\mathsf{regular}$}
\escmathcom{\HHint}{0}{$\mathsf{Int}$}
\escmathcom{\HHclos}{0}{$\mathsf{Clos}$}

\escmathcom{\horse}{0}{$\to$}

% Topology


\escmathcom{\Regions}{0}{$\mcal{R}$}
\escmathcom{\Region}{0}{$\mcal{R}$}
\escmathcom{\ran}{0}{$\mcal{R}$}
\escmathcom{\Opens}{0}{$\mcal{O}$}
\escmathcom{\Closeds}{0}{$\mcal{C}$}
\escmathcom{\Inter}{0}{$\mathsf{i}$}
\escmathcom{\Closure}{0}{$\mathsf{c}$}
\escmathcom{\Int}{1}{$\mathsf{i}(#1)$}
\escmathcom{\Clos}{1}{$\mathsf{c}(#1)$}
\escmathcom{\Boun}{1}{$\mathsf{b}(#1)$}
%\escmathcom{\Reg}{1}{$\mathsf{r}(#1)$}
\escmathcom{\Reg}{0}{$\mathsf{R}$}
\escmathcom{\Heyt}{1}{$\mathsf{h}(#1)$}
\escmathcom{\SComp}{1}{$\ov{#1}$}
\escmathcom{\setimp}{0}{$\iimp$}
\escmathcom{\hcomp}{0}{$\backsim$}

%\escmathcom{\atom}{0}{$\mathsf{atom}$}
\escmathcom{\region}{0}{$\mathsf{region}$}
\escmathcom{\Base}{1}{$\phi (\wt{#1})$}

\escmathcom{\zInt}{0}{$\mathsf{I}$}
\escmathcom{\zClos}{0}{$\mathsf{C}$}
\escmathcom{\zBoun}{0}{$\mathsf{Bd}$}
\escmathcom{\zRegl}{0}{$\approx$}
\escmathcom{\Rr}{0}{$\approx$}
\escmathcom{\zHco}{0}{$\ineg$}
\escmathcom{\zBco}{0}{$\neg$}


\escmathcom{\emp}{0}{$\varnothing$}
\escmathcom{\sseq}{0}{$ \ \subseteq \ $}
\escmathcom{\nsseq}{0}{$ \ \nsubseteq \ $}
\escmathcom{\qsseq}{0}{$ \ \sqsubseteq \ $}
\escmathcom{\nqsseq}{0}{$ \ \not\sqsubseteq \ $}
\escmathcom{\qsset}{0}{$ \ \sqsubset \ $}
\escmathcom{\sqsseq}{0}{$\sqsubseteq$}
\escmathcom{\nqseq}{0}{$\nsqsubseteq$}
\escmathcom{\nsqsseq}{0}{$\not\sqsubseteq$}


% Names of Logical Systems

\escmathcom{\Itwo}{0}{$\mathit{I2}$}
\escmathcom{\Ctwo}{0}{$\mathit{C2}$}
\escmathcom{\Dtwo}{0}{$\mathit{D2}$}
\escmathcom{\Ctwor}{0}{$\mathit{C2}_{r}$}
\escmathcom{\Dtwor}{0}{$\mathit{D2}_{r}$}
\escmathcom{\Ctwoc}{0}{$\mathit{C2}_{c}$}
\escmathcom{\Dtwoc}{0}{$\mathit{D2}_{c}$}
\escmathcom{\Htwo}{0}{$\mathit{I2}$}
\escmathcom{\Htwoi}{0}{$\mathit{I2}_{m}$}
\escmathcom{\Htwor}{0}{$\mathit{I2}_{r}$}
\escmathcom{\Htwoc}{0}{$\mathit{I2}_{c}$}
%\escmathcom{\twoC}{0}{$\mathit{2C}$}
%\escmathcom{\twoD}{0}{$\mathit{2D}$}
\escmathcom{\NCtwo}{0}{$\mathit{NC2}$}
\escmathcom{\NDtwo}{0}{$\mathit{ND2}$}
\escmathcom{\VCtwo}{0}{$\mathit{VC2}$}
\escmathcom{\VDtwo}{0}{$\mathit{VD2}$}
\escmathcom{\Sfour}{0}{$\mathit{S4}$}
\escmathcom{\Sfive}{0}{$\mathit{S5}$}





\escmathcom{\IP}{0}{\emph{H}}
%\escmathcom{\Sfour}{0}{\emph{S4}}
%\escmathcom{\Sfive}{0}{\emph{S5}}
\escmathcom{\Sfourfive}{0}{\emph{S4-S5}}
\escmathcom{\IPtwo}{0}{\emph{H2}}
\escmathcom{\IPD}{0}{\emph{D2}}
%\escmathcom{\IPD}{0}{\emph{2tA}}
\escmathcom{\NIPD}{0}{\emph{ND2}}
\escmathcom{\VIPD}{0}{\emph{VD2}}
%\escmathcom{\NIPD}{0}{\emph{N2tA}}
%\escmathcom{\VIPD}{0}{\emph{V2tA}}
\escmathcom{\SIPD}{0}{$\Sigma$\emph{H2D}}
\escmathcom{\CP}{0}{\emph{C}}
\escmathcom{\CPP}{0}{\emph{2C}}


\escmathcom{\IPT}{0}{\emph{IPT}}
\escmathcom{\IPC}{0}{\emph{IPC}}
\escmathcom{\IPQ}{0}{\emph{IPC}}
\escmathcom{\IPL}{0}{\emph{IPL}}
\escmathcom{\IPLtwo}{0}{\emph{IP2}}
\escmathcom{\IPLK}{0}{$\mathit{IP2N}$}
\escmathcom{\IPCcd}{0}{$\mathit{IPC}_{D}$}
\escmathcom{\IPCk}{0}{$\mathit{IPC}_{K}$}
\escmathcom{\IPCcdk}{0}{$\mathit{IPC}_{DK}$}
\escmathcom{\RCC}{0}{\emph{RCC}}
\escmathcom{\RCCe}{0}{\emph{RCC8}}
\escmathcom{\RCCint}{0}{$\mcal{I}_{8}$}
\escmathcom{\CPC}{0}{\emph{CPC}}
\escmathcom{\CPL}{0}{\emph{CPL}}
\escmathcom{\LLAMA}{0}{\emph{LLAMA}}


% Sets of Expressions

\escmathcom{\TERM}{0}{$\mbox{\emph{Term}}$}
\escmathcom{\SUBST}{0}{$\mbox{\emph{Subst}}$}
\escmathcom{\PROP}{0}{$\mbox{\emph{Prop}}$}
\escmathcom{\WFF}{0}{$\mbox{\emph{Wff}}$}
\escmathcom{\OPER}{0}{$\mbox{\emph{Oper}}$}
\escmathcom{\ATOM}{0}{$\mbox{\emph{PropAtom}}$}
\escmathcom{\VAR}{0}{$\mbox{\emph{Var}}$}
\escmathcom{\CONS}{0}{$\mbox{\emph{Cons}}$}
\escmathcom{\CONST}{0}{$\mbox{\emph{Const}}$}
\escmathcom{\EVAL}{0}{$\mbox{\emph{EvalExp}}$}
\escmathcom{\FUN}{0}{$\mbox{\emph{Fun}}$}
\escmathcom{\PRED}{0}{$\mbox{\emph{Pred}}$}
\escmathcom{\ATOMwff}{0}{$\mbox{\emph{AtomWff}}$}


% Semantical Symbolism

\escmathcom{\vtopol}{0}{$\Vdash$}

\escmathcom{\zInter}{0}{$\rho$}
\escmathcom{\zIVal}{2}{$\Vert #1 \Vert_{#2}$}
\escmathcom{\zVal}{1}{$\Vert #1 \Vert$}
\escmathcom{\kVal}{1}{$\lceil #1 \rceil$}


\escmathcom{\Root}{0}{$\textbf{0}$}
\escmathcom{\bool}{0}{$\mbox{\emph{Bool}}$}
\escmathcom{\Domain}{1}{$\mathcal{#1}$}  
\escmathcom{\KDomain}{2}{\tindset{#2}{#1}}  
\escmathcom{\FDomain}{2}{\tindset{#2}{#1}}  

\escmathcom{\Th}{0}{$\Delta$}
\escmathcom{\tModel}{0}{$\vDash \ $}
\escmathcom{\ntModel}{0}{$\nvDash \ $}
\escmathcom{\tpModel}{0}{$\vDash_{+} \ $}
\escmathcom{\MODEL}{1}{$\vDash_{#1}$}
\escmathcom{\REAL}{0}{$\rho$}
\escmathcom{\VAL}{0}{$\mbox{\emph{VAL}}$}
\escmathcom{\VALR}{0}{${\mbox{\emph{VAL}}}_{\rho}$}
\escmathcom{\intt}{1}{$\Vert #1 \Vert$}

\escmathcom{\ASG}{0}{$\mbox{\emph{Assg}}$}
\escmathcom{\KASG}{1}{${\emph{Assg}}_{#1}$}
\escmathcom{\ASSG}{0}{$\sigma$}
\escmathcom{\eqASSG}{1}{$\sigma' =_{#1} \sigma$}  

\escmathcom{\tVAL}{0}{$\mbox{\emph{Val}}$}
\escmathcom{\tVALR}{0}{${\mbox{\emph{Val}}}_{\rho}$}
\escmathcom{\tVALRT}{0}{${\mbox{\emph{ValSet}}}_{\rho}$}

%\escmathcom{\kVAL}{1}{$\mbox{\emph{Val}}^{#1}$}
\escmathcom{\kVALR}{0}{${\mbox{\emph{Val}}}_{\rho}$}
\escmathcom{\kkVALR}{1}{${\mbox{\emph{Val}}}_{\rho}^{#1}$}

\escmathcom{\INTa}{2}{$\vert #1 \vert_{#2}$}
\escmathcom{\INTt}{3}{$\Vert #2 \Vert_{#3}^{#1}$}
\escmathcom{\INTk}{3}{$\lceil #2 \rceil_{#3}^{#1}$}

\escmathcom{\TSETR}{2}{$\Vert #1 \Vert_{#2}$}
\escmathcom{\KSETR}{2}{$\lceil #1 \rceil_{#2}$}


% Other symbols

\escmathcom{\And}{0}{\textnormal{and}}
\escmathcom{\Or}{0}{\textnormal{or}}
\escmathcom{\If}{0}{\textnormal{if}}
\escmathcom{\Then}{0}{\textnormal{then}}
\escmathcom{\EX}{0}{\textnormal{exists}}
\escmathcom{\FOR}{0}{\textnormal{for all}}



% END NEW MACROS












\def\h{\hbox}

\newcommand{\mathcom}[3]{ \newcommand{#1}[#2]{\mbox{$#3$}}}
\newcommand{\remathcom}[3]{ \renewcommand{#1}[#2]{\mbox{$#3$}}}

\newcommand{\mdef}[2]{ \newcommand{#1}{\mbox{$#2$}} }
 
\mathcom{\y}{0}{\ \vdash\ }               % yeilds 
\mathcom{\ys}{0}{\vdash_{\cal S}}         % yeilds sub S
\mathcom{\yt}{0}{\vdash_{\theta}}         % yeilds sub theta

\mathcom{\absurd}{0}{\mathbf{f}}                 % absurdity
\mathcom{\imp}{0}{\ \rightarrow\ }            % implication arrow
\mathcom{\con}{0}{\ \wedge\ }                 % conjunction
\mathcom{\dis}{0}{\ \vee\ }                   % disjunction
\mathcom{\n}{0}{\neg}                     % negation
\mathcom{\dimp}{0}{\ \leftrightarrow\ }       % mat equiv

\mathcom{\corresponds}{0}{\ \Lleftarrow\! \! \Rrightarrow\ }


%%\mathcom{\th}{0}{\theta}                  % theta


\mathcom{\A}{0}{\forall}                  % universal quantifier
\mathcom{\E}{0}{\exists}                  % existential quant

\mathcom{\Dec}{1}{{\cal D}(#1)}           % D(#1)

\mathcom{\elt}{0}{\in}

\mathcom{\tuple}{1}{\langle #1 \rangle}

% \mathcom{\equivdef}{0}{\equiv_{\mbox{\em \tiny def}}}
% \mathcom{\eqdef}{0}{=_{\mbox{\em \tiny def}}}

\def\equivdef{\mathrel{\ \equiv_{\mbox{\em \tiny
def}}\ }}

\def\eqdef{\mathrel{\ =_{\mbox{\em \tiny def}}\ }}


\newtheorem{Rule}{Rule}
\newtheorem{Lemma}{Lemma}
\newtheorem{Theorem}{Theorem}


% Redeclaration of symbols for intuitionistic logic
% see Diller p.126
\def\ineg{\mathop\sim}
\def\iimp{\mathbin\Rightarrow}
\def\iiff{\mathbin\Leftrightarrow}
% new symbols from amssymb
%\def\iimp{\mathbin\rightarrowtail}
%\def\iiff{\mathbin\leftrightsquigarrow}

\def\icon{\mathbin\curlywedge}
\def\idis{\mathbin\curlyvee}

%Definitions for RCC inserts
\newcommand{\notch}{\Longrightarrow\kern-16pt{/}\ \ \ }
\newcommand{\ch}{\Longrightarrow}
\newcommand{\pr}[1]{\mbox{\sf #1}}

\newcommand{\ov}[1]{\overline{#1}}
\newcommand{\void}[0]{\emptyset}


% RCC relations
\mathcom{\C}{0}{\pr{C}}
\remathcom{\P}{0}{\pr{P}}
\remathcom{\O}{0}{\pr{O}}
\mathcom{\PO}{0}{\pr{PO}}
\mathcom{\PP}{0}{\pr{PP}}
\mathcom{\NTPP}{0}{\pr{NTPP}}
\mathcom{\NTPPI}{0}{\pr{NTPP}^{-1}}
\mathcom{\TPP}{0}{\pr{TPP}}
\mathcom{\TPPI}{0}{\pr{TPP}^{-1}}
\mathcom{\TP}{0}{\pr{TP}}
\mathcom{\NTP}{0}{\pr{NTP}}
\mathcom{\NTPi}{0}{\pr{NTPi}}
\mathcom{\EC}{0}{\pr{EC}}
\mathcom{\DC}{0}{\pr{DC}}
\mathcom{\DR}{0}{\pr{DR}}
\mathcom{\EQ}{0}{\pr{EQ}}
\mathcom{\CG}{0}{\pr{CG}}

\mathcom{\TC}{0}{\pr{TC}}
\mathcom{\NTC}{0}{\pr{NTC}}

\mathcom{\Pin}{0}{\pr{Pi}} % \Pi is Greek letter
\mathcom{\PPi}{0}{\pr{PPi}}
\mathcom{\NTPPi}{0}{\pr{NTPPi}}

\mathcom{\OP}{0}{\pr{OP}}

\mathcom{\conv}{0}{\pr{conv}}
\mathcom{\rsum}{0}{\pr{sum}}
\mathcom{\rdiff}{0}{\pr{diff}}
\mathcom{\rcompl}{0}{\pr{compl}}
\mathcom{\rprod}{0}{\pr{prod}}
\mathcom{\cp}{0}{\pr{cp}}
\mathcom{\Us}{0}{\pr{Us}}
\mathcom{\us}{0}{\pr{u}}
\mathcom{\NULL}{0}{\pr{NULL}}
\mathcom{\rnull}{0}{\emptyset}
\mathcom{\CONV}{0}{\pr{CONV}}
\mathcom{\CON}{0}{\pr{CON}}

% Adaptations for new latex compatibility
\renewcommand{\Box}{\square}
\def\cal{\mathcal}

% meta language in definitions etc
\mathcom{\ml}{1}{\;\;\;\; \mbox{#1} \;\;\;\;}


%-- TOPLOG SYMBOLS

\mathcom{\Co}{0}{{\cal C}_{0} \ }
\mathcom{\CoX}{0}{{\cal C}_{0}}
\mathcom{\Cop}{0}{{\cal C}_{0}^+ \ }
\mathcom{\CopX}{0}{{\cal C}_{0}^+}
\mathcom{\mcop}{0}{\models_{C_o^+}}

\mathcom{\Lop}{0}{{\cal L}_{0}^+ \ }
\mathcom{\LopX}{0}{{\cal L}_{0}^+}
\def\mlo{\models_{{\mathcal  L}_0}}
\def\mlop{\models_{{\mathcal  L}_0^+}}

\mathcom{\Ci}{0}{{\cal C}_{1} \ }
\mathcom{\CiX}{0}{{\cal C}_{1}}

\mathcom{\Io}{0}{{\cal I}_{0} \ }
\mathcom{\IoX}{0}{{\cal I}_{0}}
\mathcom{\Iop}{0}{{\cal I}_{0}^+ \ }
\mathcom{\IopX}{0}{{\cal I}_{0}^+}

%%\mathcom{\inter}{1}{\langle #1 \rangle}
\mathcom{\inter}{1}{i( #1 )}

\newfont{\myssfont}{cmss12}

\mathcom{\Maps}{2}{\, _{#1}\!\!\!\rightleftharpoons^{#2} \ }

\mathcom{\CotoST}{0}{\Maps{C_0}{ST}}
\mathcom{\CtoST}{0}{\Maps{C\,}{ST}}
\mathcom{\IotoST}{0}{\Maps{I_0}{ST}}

\mathcom{\uni}{0}{\cal U \, }
\mathcom{\uniX}{0}{\cal U}
\mathcom{\U}{0}{\cal U}

\mathcom{\eu}{0}{\,=\,\uni}

\mathcom{\ol}{1}{\overline{#1}}

\mathcom{\Lsse}{0}{{\cal L}_{sse}}
\mathcom{\Lssei}{0}{{\cal L}_{ssei}}

\mathcom{\Luse}{0}{{\cal L}_{use}}
\mathcom{\Lusei}{0}{{\cal L}_{usei}}

\mathcom{\disdots}{0}{\dis\!\!\!\ldots\!\!\dis}
\mathcom{\condots}{0}{\con\ldots\con}


%% Miscelleneous

% Enclose arguments in square double brackets to
% refer to semantic value of an expression
\def\semvalue#1{[\![#1]\!]}

\def\forces{\mathop{\setbox0=\hbox{$\vdash$}\ \rule{0.04em}{1\ht0}\mkern1.5mu\box0}}

\mathcom{\hence}{0}{\Longrightarrow \hspace{2em}}

\def\sliderule{\centerline{\rule{4in}{1mm}}}

\def\Cbox{\mathop{\square\mkern-10mu
                   \mbox{\raise0.45ex\hbox{\tiny\sf C}}
                  \mkern2mu
                 }
         }

\def\letterbox#1{\mathop{\square\mkern-10mu
                   \mbox{\raise0.2ex\hbox{\small\sf #1}}
                  \mkern2mu
                 }
         }

%\def\tBox{\letterbox{\lower0.4ex\hbox{t}}}
\def\sBox{\letterbox{s}}
\def\tBox{\letterbox{t}}

\def\overstrike#1#2{\mathop{%
               \setbox1=\hbox{$#1$}%
               \setbox2=\hbox{$#2$}%
               \ifdim 1\wd1<1\wd2 % 
                    { \rlap{\mbox{\kern0.5\wd2\kern-0.5\wd1 \box1}} \box2 }
              \else {   \rlap{\mbox{$#1$}}
                        \rlap{\mbox{\kern0.5\wd1\kern-0.5\wd2 \box2}}
                        \phantom{#1} } 
              \fi }}

\def\rbox{\boxminus}
\def\rdia{\overstrike{\Diamond}{-}}
\def\cdia{\overstrike{\Diamond}{\rule{0.04em}{1.5ex}}}
\def\cbox{\inbox{\rule{0.04em}{1.6ex}}}
\def\cbox{\mathop{\setbox0=\hbox{$\square$}
                  \overstrike{\square}{\hbox{\rule{0.04em}{1\ht0}}}
             }}

\def\diaplus{\overstrike{\Diamond}{+}}




%% Make a modal box with a symbol centred inside
\def\inbox#1{\mathop{\setbox0=\hbox{$\square$}%
                      \setbox1=\hbox{#1}%
                      \rlap{\hbox{$\square$}}
                      \rlap{\mbox{\kern0.5\wd0\kern-0.5\wd1%
                                  \box1}}%
                      \phantom{\square}}}

\def\ibox{\mathop{\square\mkern-10mu
                   \mbox{\raise0.45ex\hbox{\tiny\em i}}
                  \mkern7mu
                 }
         }
\def\cdiamond{\mathop{\Diamond\mkern-12.7mu
                   \mbox{\raise0.4ex\hbox{\tiny\em c}}
                  \mkern7mu
                 }
         }


\def\s5box{\mathop{\square\mkern-10mu
                   \mbox{\raise0.45ex\hbox{\tiny 5}}
                  \mkern7mu
                 }
         }

\def\Fbox{\mathop{\square\mkern-11mu
                   \mbox{\raise0.45ex\hbox{\tiny \bf F}}
                  \mkern2mu
                 }
         }

\def\Box{\mathop\square}
\def\Diamond{\mathop\lozenge}

\def\bigI{\mathop{%
\hbox{%
%%
\def\thickness{1pt}%
\def\width{6pt}%
\def\height{0.6em}%
\def\gap{0.15em}%
%%
\dimen0=\thickness%
\divide\dimen0 by 2%
\dimen1=\width%
\divide\dimen1 by 2%
%%
\kern\gap%
\rlap{\vrule width\width height\thickness depth0pt}%
\rlap{\kern\dimen1 \kern-\dimen0%
\vrule width\thickness height\height depth0pt}%
\raise\height\hbox{\vrule width\width height\thickness depth0pt}%
\kern\gap}}}

\def\mconv{\mathop{% 
      \mbox{\raise0.35ex\hbox{$\scriptstyle\bigcirc$}}
                  }
          }
\def\mpconv{\circleddash}


\def\ri{\mathclose{\raise1ex\hbox{$\smile$}}}

\def\eqtag#1{\eqno ({\bf #1})}

\def\mc:#1{\hbox{${\mathcal{#1}}$}}
\def\mb#1{{\mathbf{#1}}}
 
\def\ifff{\qquad \hbox{if and only if} \qquad}

\def\d{\delta}

\def\QED{$\blacksquare$}

\def\epsrcgrant{GR/K65041}
\def\VUGgrant{GR/M56807}

\def\Boxx{\boxtimes}

%% Force a math symbol to go into scriptsize
\def\mscriptsize#1{\hbox{\scriptsize $#1$}}
\def\mlarge#1{\hbox{\large $#1$}}

\def\boxtheorem#1#2{~\\
\centerline{\fbox{
\parbox{5in}{
\centerline{\bf #1}
#2
}}}
\vspace{1ex}}

\def\proof#1#2{
\begin{quotation}
\noindent {\bf Proof of #1:}
#2
\QED
\end{quotation}
}

\def\proofpar{\\ \hspace*{1.5em}}


%%%% ENVIRONMENTS

%% specify my list parameters
%% These can be overridden by declarations within
%% a document.

\def\mytopsep{1ex}
\def\mypartopsep{0ex}
\def\myitemsep{0.3ex}
\def\myparsep{0ex}
\def\mylistparindent{2em}
\def\myleftmargin{4.5em}
\def\mylabelwidth{4.5em}
\def\mylabelsep{1em}
\def\myitemindent{0em}

\def\setmylistparams{%
             \setlength{\topsep}{\mytopsep}
             \setlength{\partopsep}{\mypartopsep}
             \setlength{\itemsep}{\myitemsep}
             \setlength{\parsep}{\myparsep}
             \setlength{\listparindent}{\mylistparindent}
             \setlength{\leftmargin}{\myleftmargin}
             \setlength{\labelwidth}{\mylabelwidth}
             \setlength{\labelsep}{\mylabelsep}
             \setlength{\itemindent}{\myitemindent}
        }


\newcounter{tempvalue}
% A compact list making environment
\newenvironment{clist}
    { %\vspace{1ex}
      \begin{list}
            {$\bullet$}
            \setmylistparams
    }
    { \end{list}\addvspace{1ex} 
    }
% A compact list making environment
% with -- as the default item tag
\newenvironment{cdashlist}
    { %\vspace{1ex}
      \begin{list}
            {--}
            {\setlength{\topsep}{0.2ex}
             \setlength{\partopsep}{0ex}
             \setlength{\itemsep}{0.3ex}
             \setlength{\parsep}{0ex}
             \setlength{\listparindent}{2em}
            }
    }
    { \end{list}\addvspace{1ex} 
    }


% compact list with wide labels for item tags
\newenvironment{taglist}[1]
    { %\vspace{1ex}
      \begin{list}
            {$\bullet$}
            {\setlength{\topsep}{0.2ex}
             \setlength{\partopsep}{0ex}
             \setlength{\itemsep}{0.3ex}
             \setlength{\parsep}{0ex}
             \setlength{\listparindent}{2em}
             \setlength{\leftmargin}{#1}
             \setlength{\labelwidth}{#1}
             %\addtolength{\labelwidth}{-1em}
             \setlength{\labelsep}{0em}
            }
    }
    { \end{list}\addvspace{1ex} 
    }

\newcounter{cenumcount}
\newenvironment{cenum}
    { %\vspace{1ex}
      \begin{list}
            {\arabic{cenumcount}.}
            {\usecounter{cenumcount}
             \setlength{\topsep}{0.2ex}
             \setlength{\partopsep}{0ex}
             \setlength{\itemsep}{0.3ex}
             \setlength{\parsep}{0ex}
             \setlength{\listparindent}{2em}
            }
    }
    { \end{list}\addvspace{1ex} 
    }

%% Labelled list environment
\newenvironment{llist}[1]{%
\newcounter{#1}
\begin{list}{{\bf #1\arabic{#1}})\hspace{1em}}{\usecounter{#1}}
}
{\end{list}}

%% Labelled list continued
%% (assumes counter already exists)
\newenvironment{llistcont}[1]{%
\setcounter{tempvalue}{\value{#1}}
\begin{list}{{\bf #1\arabic{#1}})\hspace{1em}}{%
\usecounter{#1}
\setcounter{#1}{\value{tempvalue}}
}
}
{\end{list}}

%%% compact versions of llist and llistcont
\newenvironment{cllist}[1]{%
\newcounter{#1}
\begin{list}{{\bf #1\arabic{#1}})}
             { \usecounter{#1}
               \setmylistparams
             }
}
{\end{list}}

%% Labelled list continued
%% (assumes counter already exists)
\newenvironment{cllistcont}[1]{%
\setcounter{tempvalue}{\value{#1}}
\begin{list}{{\bf #1\arabic{#1}})}{%
                 \usecounter{#1}
                 \setcounter{#1}{\value{tempvalue}}
                \setmylistparams
          }
}
{\end{list}}



%% Define a labelled list type
%% #1 name of control sequence for the list type
%% #2 prefix for numbering
\def\llisttype#1#2{%
\newcounter{#2}
\expandafter\def\csname #1list\endcsname##1{%
\begin{cllistcont}{#2}
##1
\end{cllistcont}}
\expandafter\def\csname #1\endcsname##1{%
\begin{cllistcont}{#2}
\item
##1
\end{cllistcont}}
}

\def\litem#1#2{%
\begin{clist}
\item[{\bf #1})] 
#2
\end{clist}}

\newenvironment{chapabst}%
{\begin{quotation}\small\noindent }{\end{quotation}}

% Split line displayed formulae
% (not actually an environment)
\newcommand{\splitdismath}[4]{%
{\setlength{\arraycolsep}{0em}
\begin{eqnarray}
& #1 & #2 \nonumber \\
 &    & #3 
\label{#4}
\end{eqnarray}
}
}

% Displayed equation with tabbing

\newcommand{\tabmathn}[2]
{ \begin{equation}
\vbox{
\vspace{-2ex}
\begin{tabbing}
#1
\end{tabbing}
\vspace{-3.5ex}
}
\label{#2}
\end{equation}
}

\newcommand{\displaytab}[1]
{ \begin{displaymath}
\vbox{
\vspace{-2ex}
\begin{tabbing}
#1
\end{tabbing}
\vspace{-3.5ex}
}
\end{displaymath}
}

\def\myeq#1{$$\eqalign{#1}$$}


%%% Reinstate \eqalign macros removed by Lamport
\catcode`\@=11
\def\eqalign#1{\null \,\vcenter {\openup \jot \m@th 
\ialign {\strut \hfil $\displaystyle{##}$&$\displaystyle {{}##}$\hfil
 \crcr #1\crcr }}\,}

\def\eqalignno#1{\displ@y \tabskip \centering 
\halign to\displaywidth {\hfil $\@lign \displaystyle{##}$\tabskip \z@skip
 &$\@lign \displaystyle {{}##}$\hfil \tabskip \centering 
 &\llap {$\@lign ##$}\tabskip \z@skip \crcr 
 #1\crcr }}

\def\leqalignno#1{\displ@y \tabskip \centering
 \halign to\displaywidth {\hfil $\@lign \displaystyle {##}$\tabskip \z@skip
 &$\@lign \displaystyle {{}##}$\hfil \tabskip \centering
 &\kern -\displaywidth \rlap {$\@lign ##$}\tabskip \displaywidth \crcr 
 #1\crcr}}

%%% Define something like \bibliography but only records the
%%% bibfiles (does not input the bbl file)
\def\bibliographyfiles#1{%
  \if@filesw
    \immediate\write\@auxout{\string\bibdata{#1}}%
  \fi}

%%% Define a separate command to input the bbl file.
\def\bibliographytext{%
  \@input@{\jobname.bbl}}

%% Change indent space allocated by \thebibliography
%% use before \bibliography or \bibliographytext 
\def\setbibindent#1{%
\let\oldthebibliography\thebibliography
\def\thebibliography##1{\oldthebibliography{#1}}
}



\catcode`\@=12

\long\def\skipover#1{}

\long\def\correction#1{\footnote{TO CORRECT: #1}}
\def\nocorrections{\long\def\correction##1{}}
\def\question#1{{\bf Question:} {\em #1} }
\long\def\todo#1{{\bf To Do:} {\em #1} }


\def\twiddle{$\sim$}

%% Macro for URLs
%% puts them in tt format and  breaks line after dot if needed.

\def\urlspace{-0.5em}
\def\url#1{{\tt \urlsub#1\end}}
%\def\urldot{. \hspace{-1em}\urlsub}
\def\urldot{. \linebreak[1]\hspace\urlspace\urlsub}
\def\urlcolon{: \linebreak[1]\hspace\urlspace\hspace\urlspace\urlsub}
\def\urlslash{/ \linebreak[1]\hspace\urlspace\urlsub}
\def\urltilde{\twiddle\urlsub}
\def\urlsub#1{\ifx#1\end \let\next=\relax
      \else \ifx#1.\let\next=\urldot
      \else \ifx#1:\let\next=\urlcolon
      \else \ifx#1/\let\next=\urlslash
      \else \ifx#1~\let\next=\urltilde
      \else#1\let\next=\urlsub\fi\fi\fi\fi\fi\next}


% Input a file in verbatim mode (eg a program file)
\def\verbinput#1{\expandafter\begin{verbatim}\input#1}
%NB: usage: \verbinput{filename}\end{verbatim}


\def\mb#1{{\mathbf{#1}}}


%%% MACROS for SLIDES

\def\headd#1{%
\centerline{
\begin{tabular}{|c|} \hline #1 \\ \hline \end{tabular}
           } 
\vspace{1ex minus 1ex}
}

\def\head1#1{%
\centerline{
\large \begin{tabular}{c}  #1  \end{tabular}
           }
\vspace{1ex minus 1ex}
}

\def\heading#1{%
\centerline{\shadowbox{
\large \begin{tabular}{c} #1 \end{tabular}
           }  }
\vspace{1ex minus 1ex}
}

\def\bheading#1{%
\centerline{\shadowbox{
\large\bf \begin{tabular}{c} #1 \end{tabular}
           }  }
\vspace{1ex minus 1ex}
}

\def\printlandscape{\special{landscape}}    % Works with dvips.
\def\printportrait{\special{portrait}}

\def\leedslogo#1{\centerline{
\psfig{file=uol.eps,width=#1} %% May look wrong in xdvi
}}

\def\titleslide#1#2{
\begin{slide*}
~

\vfill
\heading{#1}

\vfill
\begin{center}
        {\large\bf Brandon Bennett} 

\vspace{3ex}
        Division of AI \\
        School of Computer Studies \\
        University of Leeds\\ 
        Leeds LS2 9JT, England \\
        {\tt brandon@scs.leeds.ac.uk} \\
\end{center}

\vfill
\centerline{
\psfig{file=uol.eps,width=0.7in} %% May look wrong in xdvi
}
\vfill
\centerline{#2}
\vfill
\end{slide*}
}


\def\nlf{\\ \mbox{} \hfill}

%%\def\fixhyphens{\usepackage[english]{babel}}

%% FIX HYPHENATION
\lefthyphenmin=2
\righthyphenmin=3

%% special commands for spell checker
\def\sic#1{#1}
\def\sico!#1!{#1}
\def\sicname#1{#1}
\def\accept#1{}
\def\acceptname#1{}

%\newenvironment{nospell}{}{}
\def\spelloff{}
\def\spellon{}

\def\ifundefined#1{\expandafter\ifx\csname#1\endcsname\relax}

\ifundefined{pdfoutput}
  \def\graftype{eps}
\else
   \def\graftype{png}
\fi

\def\graphic[#1]#2{\includegraphics[#1]{#2.\graftype}}

\def\dayth{\number\day\ifcase\day \or
st\or nd\or rd\or th\or th\or th\or th\or th\or th\or th\or 
th\or th\or th\or th\or th\or th\or th\or th\or th\or th\or
st\or nd\or rd\or th\or th\or th\or th\or th\or th\or th\or
st\fi}

\def\monthname{\ifcase\month \or
January\or February\or March\or April\or May\or June\or
July\or August\or September\or October\or November\or December\fi}

%%% Local Variables: 
%%% mode: latex
%%% TeX-master: t
%%% End: 


% Use \includegraphics{*.eps} for pictures

\begin{document}


\title{Translating Haskell into Isabelle}

\author{Paolo Torrini, Till Mossakowski, Christian Maeder} 
\institute {Informatik, Universitaet Bremen}

\date{}

\maketitle

\abstract{\scriptsize \bf We have defined and implemented as functions
  of Hets two partial translations of Haskell into Isabelle
  higher-order logics. Hets is a Haskell-based proof-management and
  program development system that allows for integration with other
  verification tools.  Our application can translate simple Haskell
  programs to HOLCF and, under stronger restrictions, to HOL.  Both
  translations use the static analysis of Programatica, are
  based on shallow embedding and rely on denotational semantics.}\\
% The translation of monads uses the AWE extension of Isabelle.}\\

\sloppy

\noindent 
Translations between programming and specification languages, as well
as the corresponding integration of compilers, analyzers and
theorem-provers, can provide useful support to the formal development
and verification of programs.  Verifying a program involves a
formulation of the requirements in a logic, a translattion of the
program to the logic, and a proof of the correctness statement. The
translation must rest on a formal semantics of the programming
language given in the logic. It should allow for representations of
programs that make proofs as easy as possible. And, not the least on
account of safety, it should be carried out automatically. It has long
been argued that functional languages, based on notions closer to
general, mathematical ones, can make the task of proving assertions
about them easier, owing to the clarity and relative simplicity of
their semantics \cite{Thompson92}.

In the following we present automated translations of Haskell programs
into Isabelle higher-order logic based on a denotational semantics.
Haskell is a strongly typed, purely functional language with lazy
evaluation, polymorphic types extended with type constructor classes,
and a syntax for side effects and pseudo-imperative code based on
monadic operators \cite{HaskellRep}.  The translations are implemented
as functions of Hets \cite{Hets}, an Haskell-based application
designed to support heterogeneous specification and formal development
of programs. Hets supplies with parsing, static analysis, proof
management and interface to various language-specific tools.  Hets
relies for interactive proofs on an interface with Isabelle, a generic
theorem-prover written in SML that includes the formalization of many
logics \cite{Paulson94isa}. Hets uses Programatica \cite{Prog04} for
the parsing and the static analysis of Haskell programs.
Programatica, a Haskell-specific formal development system built at
OGI, has a proof management on its own that includes a specification
logic and translations to different proof tools, notably to Isabelle
\cite{Huff}, albeit following a different approach from ours (see
section~\ref{sec:Translations}).


\section{Isabelle}

Isabelle-HOL (hereafter simply HOL) is the implementation in Isabelle
of classical higher-order logic based on simply typed lambda calculus
and extended with axiomatic type classes. It provides considerable
support for reasoning about programming functions, both in terms of
rich libraries and efficient automation. Since the late nineties, it
has essentially superseded FOL (classical first-order logic) as an
Isabelle standard. HOL has an implementation of recursive functions
based on Knaster-Tarski fixed-point theorem. All functions are total;
partiality may be dealt with by lifting types through the
\emph{option} type constructor.

HOLCF \cite{holcf} is HOL conservatively extended with the logic of
computable functions --- a formalization of domain theory. In HOL,
types --- elements of class \emph{type} --- are just sets; functions
may not be computable, and a recursive function may require
discharging proof obligations already at the definition stage --- in
fact, a specific measure has to be given for the function to be proved
monotonic.  In contrast, domain theory has each type interpreted as an
element of class \emph{pcpo} (pointed complete partially ordered sets)
i.e. as a set with a partial order which is closed w.r.t.
$\omega$-chains and has a bottom. All functions defined over domains,
including partial ones, are continuous, therefore computable.
Recursion can be expressed in terms of least fixed-point operator, and
so, in contrast with HOL, function definition never depends on proofs.

The Isabelle formalization of HOLCF is based on axiomatic type classes
\cite{Wenzel} so to deal with complete partial orders in a totally
abstract way. Domain theory offers a good basis for the semantical
analysis of programming languages; nevertheless, it may make proofs
comparatively hard.  After being spared the need to discharge proof
obligations at the defining stage, one has to bear with assumptions
over the continuity of functions while proving theorems. A standard
strategy to get the best out of the two systems is to define as much
as possible in HOL, using HOLCF type constructors to lift types only
when this is needed.


\section{Translations}
\label{sec:Translations}

Translations depend first of all on the expressiveness of the target
logic. Those into FOL based on large-step operational semantics of
Miranda \cite{Thompson95,Thompson89,Thompson95b} and Haskell
\cite{Thompson92} struggle with higher-order features such as
currying.  The Haskell translation to the Agda implementation of
Martin-Loef type theory in \cite{Abel} struggles with Haskell
polymorphism. Higher-order logic helps overcome such obstacles. It
also allows for higher-level approaches based on denotational
semantics, such as proposed in \cite{Huff} to translate Haskell to
HOLCF and in \cite{Pollack} to translate ML to HOL. Denotational
semantics makes it possible to give representations of programs that
are closer to their specification as well as to give proofs that are
relatively more abstract and general.

A shallow embedding into a logical language is one that relies quite
radically on its extra-logical features, possibly taking advantage of
built-in packages provided with the implementation of that language,
particularly with respect to features such as types, classes, and
recursion. On the contrary, a deep embedding relies on the logical
definition of all the relevant notions. This may give a plus in
semantic clarity and, if the logic is expressive enough, in generality
as well. Taking advantage of built-in features, on the other hand, may
help rely on standard proof methods to make theorem proving less
tedious.

The translation of ML in \cite{Pollack} based on the definition of a
class of types with bottom elements in HOL gives an example of the
deep sort. The translation of Haskell to HOLCF proposed in \cite{Huff}
relies on a generic formalization of domain theory and particularly on
the \emph{fixrec} package for recursive functions, created to provide
a friendly syntax covering mutually recursive definitions as well.  In
order to capture type constructor classes --- an important feature of
the Haskell type system --- a deep embedding approach is used there as
well.  Haskell types are translated to terms, relying on a
domain-theoretic modelling of the type system at the object level. The
practical drawback of this approach is that plenty of the automation
built into Isabelle type checking either needs to be reimplemented or
is lost.

The translations of Haskell to HOLCF and HOL that we are presenting
here are based on denotational semantics and on a shallow embedding
approach.  The translation to HOLCF relies on the \emph{fixrec}
package. In contrast with \cite{Huff}, we rely on the similarity
between type systems, translating Haskell types to HOLCF types, in a
comparatively direct way, and Haskell classes to Isabelle axiomatic
classes. The modelling we are relying on is a simplified one, and we
expect operational equivalence between Haskell programs and their
translation to HOLCF up to the level of typeable output.  The
translation covers a significant part of the Prelude syntax, although
it is still incomplete in one important respect --- it does not
include type constructor classes; we have plans for an extention that
should address this aspect.

Conceptually, type classes in Isabelle are quite different from those
in Haskell. Where the former ones are associated with sets of axioms,
the latter ones come with sets of function declarations. Moreover,
Isabelle classes may have at most a single type parameter.  Most
importantly, Isabelle does not allow for type constructor classes. The
last limitation is serious, since it makes hard to cope with essential
Haskell features such as monads and the \emph{do} notation.  In
alternative to the method proposed in \cite{Huff}, we would like to
get around the obstacle by relying on an extension of Isabelle based
on theory morphism (see section~ \ref{sec:Monads}). The AWE system
\cite{AWE} is in fact an implementation of such an extension.

The translation to HOL, similar in respect of shallow approach,
nevertheless is much more limited. An important restriction has to do
with recursion --- only primitive recursive functions are covered.
This limitation appears relatively hard to overcome, given the way
syntax for full recursion is defined for HOL. Moreover, we have to
restrict to total functions. Operational equivalence for a larger
fragment could be obtained using option types, but it is not pursued
here.


\section{Haskell2Isabelle}

The Hets function \emph{Haskell2Isabelle} supports the translation of
simple Haskell programs to HOLCF and, with more restriction, to HOL.
Not all the syntactical features used in the Prelude and main
libraries are covered. Some of the most noticeable limitations are
those related to built-in types, pattern-matching, local definitions,
import and export. There are additional and more substantial
restrictions in the case of HOL, related to termination and recursion.
Each of the translations relies on a \emph{base theory}. These are
Isabelle theory files, respectively \emph{HsHOLCF}, extending HOLCF,
and \emph{HsHOL}, extending HOL, which are included in the Hets
distribution. Each of them provides definitions and axiomatizations of
notions that are used in the corresponding translation --- notably
equality.

Information for the use of Hets may be found in \cite{HetsUG} and a
general outlook in \cite{HetsUG}. The Haskell-to-Isabelle translation
requires essentially GHC, Isabelle 2006 and Programatica --- with
respect to the latter, both analysis and translation functions in Hets
make use of its modules.  The command to run the application is

\emph{hets -v[1--5] -t Haskell2Isabelle[HOLCF | HOL] -o thy filename}

\noindent where options set verbosity, target logic and name of the
output file. The input file (last argument) must be a GHC source
(\emph{hs} extension).  The Haskell program gets analyzed and
translated. The result of a successful execution is an Isabelle theory
file (\emph{thy} extension) depending on the corresponding base
theory.

%The application requires essentially Haskell GHC and Programatica. It
%is run by the command \emph{hets -t ...  [h|hc] file.hs} where the
%options stand for HOL and HOLCF, respectively.

The internal representation of Haskell in Hets (module
\emph{Logic\_Haskell} and particularly \emph{HatAna}) is essentially
the same as in Programatica, whereas the internal representation of
Isabelle (module \emph{Logic\_Isabelle} and particularly
\emph{IsaSign}) is a Haskell reworking of the ML definition of
Isabelle own base logic, extended in order to allow for a
straightforward representation of HOL and HOLCF.

Haskell programs as well as Isabelle theories are internally
represented as Hets theories --- each of them a data-structures formed
by a signature and a set of sentences, fitting a theoretical framework
described in \cite{MossaTh}.  Each translation, defined as composition
of the signature translation with the translation of all sentences,
has the structure of a morphism from theories in the internal
representation of the source language to those in the representation
of the target language.  The distribution module
\emph{Haskell2IsabelleHOLCF} contains the main function, dependent on
the target logic. The module \emph{IsaPrint} provides the essential
functions for the pretty-printing of Isabelle theories.

The following gives a list of reserved names, i.e. the names that are
used in order to either rename or name automatically variables and
constants in the translations.  1) Type variables, in the translation
to HOL: \emph{'vX}; any name terminating with \emph{'XXn} where $n$ is
an integer.  2) Term variables, in both translations: \emph{pXn},
\emph{qXn}, with $n$ integer. 3) Constants, in the translation to HOL:
strings obtained by joining together names of defined functions, using
\emph{\_X} as end sequence and separator.

%Correspondingly, the translation function is defined as the
%composition of signature translation with the translation of all
%sentences. 
%, has the
%structure of a morphism from Hets theories written in the language of
%the source internal representation to those in the language
%representation of the target language.  Internally, the information
%about types is collected into a signature, whereas datatype
%declarations and definitions are associated with sentences.
% For the sake of maintainance, \emph{Haskell2IsabelleHOLCF.hs} is the
% module that contains the main translation function, carried out as
% composition of a signature morphism and a theory morphism.
% \emph{IsaSign.hs} contains the Hets representation of the Isabelle
% base logic and extensions.

% The general design of the translation functions, and particularly
% the two-layered recursion adopted for them, is similar to that used
% for the translation Haskell to Agda in Programatica.


\subsection{HOLCF: Type signature}

The translation to HOLCF keeps into account partiality, i.e. the fact
that a function might be undefined for certain values, either because
definition is missing, or because the program does not terminate.  It
also keeps into account laziness, i. e.  the fact that by default
function values in Haskell are passed by name and evaluated only when
needed.  Essentially, we are following the main lines of the ``crude''
denotational semantics for lazy evaluation in \cite{winskel}, pp.
216--217.  Raising an exception is different from running forever, and
both are different from stopping short of evaluation. However, from
the point of view of the printable output, these behaviours are
similar and can be treated semantically as such, i.e. using one and
the same bottom element.

% In order to ease the translation of some notions, the Isabelle
% theory \emph{HsHOLCF.thy}, that extends HOLCF has been defined and
% included in the Hets distribution.
% It is often convenient to define in Isabelle notions that match the
% translation of some itmes.
% Haskell datatype and function declarations are translated to
% declarations in HOLCF.  Names are translated by a function that
% preserves them, up to avoidance of clashes with HOLCF keywords.

Each type in Isabelle has a sort, defined by the set of the classes of
which it is member.  Haskell type variables are translated to HOLCF
ones, of class \emph{pcpo}. Each built-in type is translated to the
lifting of its corresponding HOL type. Properly covered are Haskell
booleans and unbounded integers, associated respectively to HOL
booleans and integers.
%Rationals need to be
%imported in order to be used in Haskell, so they are not covered by
%the translation at the present stage (import has not been dealt with
%yet).
%and rationals.
Bound integers and floating point numbers would need low-level
modelling, and have not been covered.  Bounded integers in particular
are simply treated as unbounded in the translation. The HOLCF type
constructor \emph{lift} is used to lift HOL types to flat domains.  In
the case of booleans, we can use type \emph{tr}, defined as equal to
$bool \ lift$ in HOLCF. In the case of integers, we use \emph{dInt},
defined in \emph{HsHOLCF} to equal $int \ lift$. The types of Haskell
functions and product are translated, respectively, to HOLCF function
spaces and lazy product --- i.e. such that $\bot = (\bot * \bot) \neq
(\bot*'a) \neq ('a * \bot)$, consistently with lazy evaluation. Type
constructors are translated to corresponding HOLCF ones (noticeably,
parameters precede type constructors in Isabelle syntax). In
particular, lists are translated to the domain \emph{llist} defined in
\emph{HsHOLCF}.  Function declarations are translated to HOLCF ones
(keyword \emph{consts}). Names (for types as well as for terms) are
translated by a function ($t$ here) that preserves them, up to
avoidance of clashes with HOLCF keywords. Translation of types (minus
mutual recursion) may be summed up as follows:

% Haskell datatype and function declarations are translated to
% declarations in HOLCF.  

$$\begin{array}{lcl}
  \lceil a \rceil & = & 'a_{t}::pcpo \\
  \lceil Bool \rceil & = & tr \\
  \lceil Integer \rceil & = & dInt \\
  \lceil a \to b \rceil & = & \lceil a \rceil \to \lceil b \rceil \\
  \lceil (a,b) \rceil & = & \lceil a \rceil * \lceil b \rceil \\
  \lceil [a] \rceil & = & \lceil a \rceil \ llist \\
  \lceil TyCons \ a_{1} \ldots a_{n} \rceil & = & \lceil a_{1} \rceil \ldots \lceil a_{n} \rceil \ TyCons_{t} 
\end{array}$$

\noindent In HOL, datatype declarations define types of class
\emph{type} by keyword \emph{datatype}; in contrast, in HOLCF, they
define types of class \emph{pcpo} (i.e. domains) by keyword
\emph{domain} (so we also may call them \emph{domain declarations}).
Each recursive datatype declaration in Haskell is translated to the
corresponding one in HOLCF. The translation of mutually recursive
datatypes relies on specific Isabelle syntax (keyword
\emph{and}), as in the next example.\\

$data \ AType \ a \ b \ = \ ABase \ a \ | \ AStep \ (AType \ a \ b) \ (BType \ a \ b)  $

$data \ BType \ a \ b \ = \ BBase \ b \ | \ BStep \ (BType \ a \ b) \ (AType \ a \ b) $\\

\noindent This translates to HOLCF as the following.

$$\begin{array}{rr} domain & \ ('a::pcpo, 'b::pcpo) \ BType \ = \ BBase \ (BBase\_1::'b) \ | \\
   & BStep \ (BStep\_1::('a, 'b) \ BType) \ (BStep\_2::('a, 'b) \ AType) \\
   and & \ ('a::pcpo, 'b::pcpo) \ AType \ =  \ ABase \ (ABase\_1::'a) \ | \\
   & AStep \ (AStep\_1::('a, 'b) \ AType) \ (AStep\_2::('a, 'b) \ BType) \\
\end{array}$$

\noindent Notably, domain declarations require an explicit
introduction of destructors. Both translations (to HOL as well as to
HOLCF) take care automatically of the order of datatype declarations
--- this is needed, insofar as, differently from Haskell, Isabelle
requires them to be listed according to their order of dependency.



\subsection{HOLCF: Sentences}

Each Haskell function definition is translated to a corresponding one.
Non-recursive definitions are translated to standard ones (keyword
\emph{defs}), whereas the translation of recursive definitions relies
on the \emph{fixrec} package. Lambda abstraction is translated as
continuous abstraction (\emph{LAM}), and function application as
continuous application (the \emph{dot} operator) --- these notions
coincide with the corresponding, HOL-defined ones, whenever their
arguments are continuous.

% As to the translation of terms containing either values or operators
% of built-in type, it is often the case that we translate them to
% lifted HOL operators. The lifting function is \emph{Def}; $\bot$ is
% used for the undefined case.

Terms of built-in type (boolean and integers) are translated to lifted
HOL values, using the HOLCF-defined lifting function \emph{Def}.  The
bottom element $\bot$ is used for all undefined terms.  The following
operator, defined in \emph{HsHOLCF}, is used to map binary arithmetic
functions to lifted functions over
lifted integers.\\

$fliftbin :: ('a \Rightarrow \ 'b \Rightarrow \ 'c) 
  \Rightarrow ('a \ lift \to \ 'b \ lift \to \ 'c \ lift) $
 
$fliftbin \ f \ == \ LAM \ y x. \ (flift1 \ (\%u. \ flift2 \ (\%v. \ f \ v \ u))) \cdot x \ \cdot y $\\

\noindent Booleans are translated to values of \emph{tr} ---
\emph{TT}, \emph{FF} and $\bot$, and boolean connectives are
translated to the corresponding HOLCF-defined lifted operators.
HOLCF-defined \emph{If then else fi} and \emph{case} syntax are used
to translate, respectively, conditional and case expressions. There
are some restrictions, however, on the latters, due to limitations in
the translation of patterns (see section~\ref{sec:Patterns}); in
particular, the case term should always be a variable, and no nested
patterns are allowed.

The translation of lists and list constructors relies on the
following \emph{HsHOLCF}-defined datatype.\\

$domain \ 'a\ llist \ = \ lNil \ | \ lCons \ (lazy \ lHd ::'a) \ (lazy \ lTl ::'a \ llist) $\\

\noindent Under the given interpretation, a stream as well as an
undefined list function take value $\bot$. This may be regarded as a
semantical weakness.

Haskell allows for local definitions by means of \emph{let} and
\emph{where} expressions. The \emph{let} expressions where the
left-hand side is a variable are translated to similar Isabelle ones;
other \emph{let} expressions (i.e. those containing patterns on the
left hand-side) and the \emph{where} expressions are not covered. The
translation of terms (minus mutual recursion) may be summed up,
essentially, as follows:

$$\begin{array}{lcl} 
  \lceil x::a \rceil & = & x_{t}::\lceil a \rceil \\
  \lceil c \rceil & = & c_{t} \\
  \lceil \backslash x \ \to \ f \rceil & = & LAM \ x_{t}. \ \lceil f \rceil \\
  \lceil (a,b) \rceil & = & (\lceil a \rceil, \lceil b \rceil) \\
  \lceil f \ a_{1} \ldots a_{n} \rceil & = & FIX \ f_{t}. \ f_{t} \ \cdot \ 
  \lceil a \rceil \ldots \cdot \ \lceil a_{n} \rceil \\
  & & \mbox{where } f::\tau, \ f_{t}::\lceil \tau \rceil \\
  \lceil let \ x_{1} \ \dots \ x_{n} \ in \ exp \rceil & = 
         & let \ \lceil x_{1} \rceil \ \dots \ \lceil x_{n} \rceil \ in \ \lceil exp \rceil
% \lceil TyCons \ a_{1} \ldots a_{n} \rceil & = & \lceil a_{1} \rceil
% \ldots \lceil a_{n} \rceil \ TyCons_{t}
\end{array}$$

%In HOLCF, where every type is a domain and every function is
%continuous, 
\noindent In HOLCF all recursive functions can be defined by fixpoint
operator --- a function that, given as argument the defining term
abstracted of the recursive call name, returns the corresponding
recursive function.  Coding this directly turns out to be rather
cumbersome, particularly in the case of mutually recursive functions,
where tuples of defining terms and tupled abstraction would be needed.
In contrast, the \emph{fixrec} package allows us to handle fixpoint
definitions in a way much more similar to ordinary Isabelle recursive
definitions, providing with friendly syntax for mutual recursion,
as well. Continuing the example,\\

\noindent $ fun1 \ :: \ (a \to c) \ \to \ (b \to d) \
\to \ AType \ a \ b \ \to \ AType \ c \ d $ 
$$\begin{array}{lll}
  fun1 & f \ g \ k  & = \ case \ k \ of \\
  & ABase \ x & \to \ ABase \ (f \ x) \\
  & AStep \ x \ y & \to \ AStep \ (fun1 \ f \ g \ x) \ (fun2 \ f \ g \
  y)
\end{array} $$
$ fun2 \ :: \ (a \to c) \ \to \ (b \to d) \ \to \ BType \ a \ b \ \to \ BType \ c \ d $
$$\begin{array}{lll}
  fun2 & f \ g \ k & = \ case \ k \ of \\
  & BBase \ x & \to \ BBase \ (g \ x) \\
  & BStep \ x \ y & \to \ BStep \ (fun2 \ f \ g \ x) \ (fun1 \ f \ g \ y)
\end{array} $$ 

\noindent this code translates to HOLCF as follows:\\

\noindent $consts $ 
$$ \begin{array}{ll} fun1 \ :: & ('a::pcpo \to 'c::pcpo) \ \to \ ('b::pcpo \to 'd::pcpo) \ \to \\
  & ('a::pcpo, 'b::pcpo) \ AType \ \to \ ('c::pcpo, 'd::pcpo) \ AType \\
  fun2 \ :: & ('a::pcpo \to 'c::pcpo) \ \to \ ('b::pcpo \to 
  'd::pcpo) \ \to \\ 
  & ('a::pcpo, 'b::pcpo) \ BType \ \to \ ('c::pcpo,
  'd::pcpo) \ BType 
\end{array}$$

$$\begin{array}{rccl} 
  fixrec & fun1 & = & (LAM f. \ LAM g. \ LAM k. \ case \ k \ of \\
  & & & ABase \cdot pX1 \ => \ ABase \cdot (f \cdot pX1) \ | \\
  & & & AStep \cdot pX1 \cdot pX2 \ => \\
  & & & \ AStep \cdot (fun1 \cdot f \cdot g \cdot pX1) \cdot (fun2 \cdot f \cdot g \cdot pX2)) \\
  and & fun2 & = & (LAM f. \ LAM g. \ LAM k. \ case \ k \ of \\
  & & & BBase \cdot pX1 \ => \ BBase \cdot (g \cdot pX1) \ | \\
  & & & BStep \cdot pX1 \cdot pX2 \ => \\
  & & & \ BStep \cdot (fun2 \cdot f \cdot g \cdot pX1) \cdot (fun1 \cdot f \cdot g \cdot pX2)) \end{array}$$

\noindent The translations take care automatically of the fact that,
in contrast with Haskell, Isabelle requires patterns in case
expressions to follow the order of datatype declarations.

In the Programatica representation of Haskell, class information about
type parameters is given by adding dictionary parameters to normal
ones. These are eliminated by the translation, as class information in
Isabelle may be given by annotating arguments. In particular, each
definition includes the type of the defined function complete with
class annotation, in order to allow for overloading (a detail we
omitted to show for the sake of readability in some of the examples
here).

\subsection{HOL: Type signature}

The translation to HOL is semantically rather crude, it takes into
account neither partiality nor laziness, and so, for soundness, it
requires all functions in the program to be total ones.

An account of partiality could be obtained using the \emph{option}
type constructor to lift types, along lines similar to those followed
in HOLCF with \emph{lift}. Here instead we are just mapping Haskell
types to corresponding, unlifted HOL ones --- so for booleans and
integers. Type variables are of class \emph{type}.  HOL function type,
product and list are used to translate the corresponding Haskell
constructors. The translation of types (minus mutual recursion) may be
summed up as follows.

% Haskell datatype and function declarations are translated to
% declarations in HOLCF.  

$$\begin{array}{lcl}
  \lceil a \rceil & = & 'a_{t}::type \\
  \lceil Bool \rceil & = & bool \\
  \lceil Integer \rceil & = & int \\
  \lceil a \to b \rceil & = & \lceil a \rceil \Rightarrow \lceil b \rceil \\
  \lceil (a,b) \rceil & = & \lceil a \rceil * \lceil b \rceil \\
  \lceil [a] \rceil & = & \lceil a \rceil \ list \\
  \lceil TyCons \ a_{1} \ldots a_{n} \rceil & = & \lceil a_{1} \rceil \ldots \lceil a_{n} \rceil \ TyCons_{t} 
\end{array}$$

\noindent Recursive and mutually recursive data-types declarations are
translated to HOL as datatype declaration.

$$\begin{array}{rll} datatype & \ ('a, 'b) \ BType & = \ BBase \ 'b \ | \\
  & & BStep \ (('a, 'b) \ BType) \ (('a, 'b) \ AType) \\
  and & \ ('a, 'b) \ AType & =  \ ABase \ 'a \ | \\
  & & AStep \ (('a, 'b) \ AType) \ (('a, 'b) \ BType) \\
\end{array}$$

\noindent Metalevel features are essentially shared with the HOLCF
translation.

\subsection{HOL: Sentences}

Non-recursive definitions are treated in an analogous way to the
translation into HOLCF. Standard lambda-abstraction ($\%$) and
function application are used here instead of continuous ones. Partial
functions, and particularly case expressions with incomplete patterns,
are not allowed. The translation of terms (minus recursion and case
expressions) may be summed up as follows:

$$\begin{array}{lcl} 
  \lceil x::a \rceil & = & x_{t}::\lceil a \rceil \\
  \lceil c \rceil & = & c_{t} \\
  \lceil \backslash x \ \to \ f \rceil & = & \% \ x_{t}. \ \lceil f \rceil \\
  \lceil (a,b) \rceil & = & (\lceil a \rceil, \lceil b \rceil) \\
  \lceil f \ a_{1} \ldots a_{n} \rceil & = & f_{t} \ \lceil a \rceil \ldots \  \lceil a_{n} \rceil \\
  & & \mbox{where } f::\tau, \ f_{t}::\lceil \tau \rceil \\
  \lceil let \ x_{1} \ \dots \ x_{n} \ in \ exp \rceil & = 
  & let \ \lceil x_{1} \rceil \ \dots \ \lceil x_{n} \rceil \ in \ \lceil exp \rceil 
% \lceil TyCons \ a_{1} \ldots a_{n} \rceil & = & \lceil a_{1} \rceil
% \ldots \lceil a_{n} \rceil \ TyCons_{t}
\end{array}$$

\noindent Recursive definitions set HOL and HOLCF apart.  In HOL a
distinction is drawn, and syntactically highlighted, between primitive
recursive functions (introduced by keyword \emph{primrec}) and generic
recursive ones (by keyword \emph{recdef}).  Termination is guaranteed
for each of the formers, by the fact that recursion is based on the
datatype structure of one of the parameters. In contrast, termination
is not a trivial matter for the latters. A strictly decreasing measure
must be provided, associated to the parameters of the defined
function. This requires a degree of ingenuity that cannot be easily
dealt with automatically. For this reason, the translation to HOL is
restricted to primitive recursive functions.

Mutual recursion is allowed for under additional restrictions --- more
precisely: 1) all the functions involved are recursive in the first
argument; 2) recursive arguments are of the same type in each
function.  The translation of mutually recursive functions $a
\rightarrow b, \ldots a \rightarrow d$ introduces a new function $a
\rightarrow (b * \ldots * d)$ recursively defined, for each case
pattern, as the product of the values
correspondingly taken by the original, non-recursively defined ones. \\

\noindent $ fun3 \ :: \ AType \ a \ b \to (a \to a) \to AType \ a \ b $
$$\begin{array}{ll}
fun3 \ k \ f & = \ case \ k \ of \\
  & ABase \ a \to ABase \ (f \ a) \\
  & AStep \ a \ b \to AStep \ (fun4 \ a) \ b \\
\end{array}$$
   
\noindent $ fun4 \ :: \ AType \ a \ b \to AType \ a \ b $
$$\begin{array}{ll}
fun4 \ k & = \ case \ k \ of \\ 
  & AStep \ x \ y \to AStep \ (fun3 \ x \ (\backslash z \to z)) \ y \\
   & ABase \ x \to ABase \ x \\
\end{array}$$

\noindent These functions, satisfying the restrictions, will translate
to the following.\\

\noindent $consts$
$$\begin{array}{ll}
fun3 \ :: & ('a::type, 'b::type) AType \Rightarrow ('a::type \Rightarrow 'a::type) \Rightarrow \\ & ('a::type, 'b::type) AType \\
fun4 \ :: & ('a::type, 'b::type) AType \Rightarrow ('a::type \Rightarrow 'a::type) \Rightarrow\\ & ('a::type, 'b::type) AType \\
fun3\_Xfun4\_X :: & ('a::type, 'b::type) AType \Rightarrow \\ & (('aXX1::type \Rightarrow 'aXX1::type) \Rightarrow \\ & ('aXX1::type, 'bXX1::type) AType) * \\
 & (('aXX2::type \Rightarrow 'aXX2::type) \Rightarrow \\ & ('aXX2::type, 'bXX2::type) AType) \\
\end{array}$$

\noindent $defs$
$$\begin{array}{lll}
fun3\_def : & fun3 \ == \ \% k \ f. \ fst \ (( fun3\_Xfun4\_X ::  \\ 
        & \ ('a::type, 'b::type) AType \Rightarrow (('a::type \Rightarrow 'a::type) \\
& \Rightarrow ('a::type, 'b::type) AType) * ((unit \Rightarrow unit) \Rightarrow \\ & (unit, unit) AType) ) \ k) \ f \\
fun4\_def : & fun4 \ == \ \% k \ f. \ snd \ (( fun3\_Xfun4\_X ::  \\ 
 & ('a::type, 'b::type) AType \Rightarrow ((unit \Rightarrow unit) \Rightarrow (unit, unit) AType) * \\ & (('a::type \Rightarrow 'a::type) \Rightarrow ('a::type, 'b::type) AType) ) \ k) \ f \\
\end{array}$$


\noindent $primrec$ 
$$\begin{array}{lll}
fun3\_Xfun4\_X & (ABase \ pX1) \ = & (\% f. \ ABase \ (f \ pX1), \\
                           &        &  \% f. \ ABase \ pX1) \\
fun3\_Xfun4\_X & (AStep \ pX1 \ pX2) & =  \\
  & (\% f. \ AStep \ (snd & (fun3\_Xfun4\_X \ pX1) \ f) \ pX2, \\
  &  \% f. \ AStep \ (fst & (fun3\_Xfun4\_X \ pX1) \ f) \ pX2) \\
\end{array}$$

\noindent Calls of the recursive functions in the non-recursive
definitions are annotated with type where exceeding type variables are
instantiated with the unit type, as required by Isabelle in order to
avoid definitions from which inconsistencies are derivable.


\subsection{Patterns}
\label{sec:Patterns}

Support of patterns in definitions and case expressions is more
restricted in Isabelle than in Haskell. Nested patterns are overall
disallowed. In case expressions, the case term is required to be a
variable. Both of these restrictions apply to our translations. A
further Isabelle limitation concerning case expressions ---
sensitiveness to pattern order --- is dealt with automatically.
Similarly, wildcards --- something unknown to Isabelle --- are dealt
with, as well as, in HOLCF, incomplete patterns.  The exclusion of
nested patterns complicate the translation of some specific ones ---
in fact, guarded expressions and list comprehension are not covered;
their use should be avoided here, using conditional expressions and
\emph{map} instead.

% not allowed in Isabelle with ordinary definitions (keyword
% \emph{Defs}), but it is allowed with the syntax for recursive ones.

Multiple function definitions using top level pattern matching are
translated as definitions based on a single case expression; this is
due to HOL more than to HOLCF. In fact, multiple definitions in
Isabelle are only allowed with the syntax of recursive ones. However,
in HOL primitive recursive definitions, patterns are allowed for only
in one parameter. In order to translate definitions with more patterns
as arguments, without resorting to tuples and to more complex syntax
(\emph{recdef} instead of \emph{primrec}) we translate multiple
definitions by top level pattern matching
as definitions by case construct.\\

\noindent $ctl \ :: \ Bool \to Bool \to Bool \to Bool$ \\
$ctl \ False \ a \ False \ = \ a $\\
$ctl \ True \ a \ False \ = \ False $\\
$ctl \ False \ a \ True  = \ True $\\
$ctl \ True \ a \ True \ = \ a $\\

\noindent This translates to HOL as the following.\\

\noindent $consts \ ctl :: \ bool \Rightarrow bool \Rightarrow bool \Rightarrow bool$\\

\noindent $defs$
$$\begin{array}{llll} 
ctl\_def : & ctl \ == & \% qX1. \% qX2. \% qX3. & case \ qX1 \ of \\
                                       & &  False \Rightarrow case \ qX3 & of \\
                                       & & &  False \Rightarrow qX2 \ | \\
                                       & & & True \Rightarrow True \ | \\
                                       & & True \Rightarrow case \ qX3 & of \\
                                       & & & False \Rightarrow False \ | \\
                                       & & & True \Rightarrow qX2 \\
\end{array}$$

\noindent This example does not go through with the translation to
HOLCF, as booleans there are translated to values of \emph{tr}, which
is not a recursive datatype. The following gives an alternative that
can be handled (a new datatype is used instead of booleans).\\

\noindent $data \ Two \ =  Fx \ | \ Tx $\\
\noindent $ctlx ::  \ Two \to Two \to Two \to Two $ \\
$ctlx \ Fx \ a \ Fx \ = \ a $ \\
$ctlx \ Tx \ a \ Fx \ = \ Fx $ \\
$ctlx \ Fx \ a \ Tx \ = \ Tx $ \\
$ctlx \ Tx \ a \ Tx \ = \ a $\\

\noindent This translates to HOLCF as follows.\\

\noindent $domain \ Two \ = \ Fx \ | \ Tx $

\noindent $consts \ ctlx :: \ Two \to Two \to Two \to Two$\\

\noindent $defs$
$$\begin{array}{llll}
ctlx\_def : & map5 \ == & LAM \ qX1 \ qX2 \ qX3. & case \ qX1 \ of \\
                                             & & Fx \Rightarrow case \ qX3 & of \\
                                             & & & Fx \Rightarrow qX2 \ | \\
                                             & & & Tx \Rightarrow Tx \ | \\
                                             & & Tx \Rightarrow case \ qX3 & of \\
                                             & & & Fx \Rightarrow Fx \ | \\
                                             & & & Tx \Rightarrow qX2 \\
\end{array}$$

\noindent In case expressions as well as in top level pattern matching
definitions, wildcards may be used --- though not in nested position.
Incomplete patterns are translated to HOLCF using $\bot$ as default
value.

\subsection{Classes}

Haskell defined classes are translated to Isabelle as classes with
empty axiomatization.  Isabelle allows classes with no more than one
type parameter, therefore our translations can support only them ---
it might be possible to handle more than one parameter using tuples,
but this would surely involve considerable complications dealing with
conditional instances.

Instance declarations are translated to corresponding ones in
Isabelle. Isabelle instances in general require proofs that class
axioms are satisfied by the types, but as long as there are no axioms
the proofs are trivial and can be handled automatically.  Function
declarations associated with Haskell classes are translated as
independent function declarations with appropriate class annotation.
Function definitions associated with instance declarations are
translated as overloaded function definitions,
relying on class annotation of the typed variables.\\

\noindent $class ClassA \ a \ where$

$abase :: \ a \to Bool $
 
$ astep :: \ a \to Bool $\\

\noindent $instance \ (ClassA \ a, \ ClassA \ b) \Rightarrow ClassA \ (AType \ a \ b) \ where $
$$\begin{array}{ll}
abase \ x \ = & case \ x \ of \\
           &  ABase \ u \ \to \ True \\
            & \_ \ \to \ False \\
\end{array}$$

\noindent This code translates to HOLCF as follows.\\

\noindent $axclass \ ClassA < pcpo$

\noindent $instance \ AType::({pcpo, ClassA}, {pcpo, ClassA}) \ ClassA$ 

$by \ intro\_classes$\\

\noindent $consts$
$$\begin{array}{ll}
abase :: & 'a::\{ClassA, pcpo\} \to tr \\
astep :: & 'a::\{ClassA, pcpo\} \to tr \\
default\_abase :: & 'a::\{ClassA, pcpo\} \to tr \\
default\_astep :: & 'a::\{ClassA, pcpo\} \to tr \\
\end{array}$$

\noindent $defs$
$$\begin{array}{rl}
AType\_abase\_def : & \\
 abase :: & ('a::\{ClassA, pcpo\}, 'b::\{ClassA, pcpo\}) \ AType \to tr \\
                                   == & LAM x. \ case \ x \ of \\
                                  &  ABase \cdot pX1 \Rightarrow TT \ | \\
                                  &  AStep \cdot pX2 \cdot pX1 \Rightarrow FF \\
AType\_astep\_def : & \\
   astep :: & ('a::\{ClassA, pcpo\}, 'b::\{ClassA, pcpo\}) \ AType \to tr \\
                         & == default\_astep \\
\end{array}$$ 

\noindent Similarly, it translates to HOL.\\

\noindent $axclass \ ClassA < type$\\

\noindent $instance \ AType::(\{type, ClassA\}, \{type, ClassA\}) \ ClassA$ 

$by \ intro\_classes$\\

\noindent $consts$
$$\begin{array}{ll}
abase :: & 'a::\{ClassA, type\} \Rightarrow bool\\
astep :: & 'a::\{ClassA, type\} \Rightarrow bool\\
default\_abase :: & 'a::\{ClassA, type\} \Rightarrow bool\\
default\_astep :: & 'a::\{ClassA, type\} \Rightarrow bool\\
\end{array}$$

\noindent $defs$
$$\begin{array}{rl}
AType\_abase\_def : & \\
   abase :: & ('a::\{ClassA, type\}, 'b::\{ClassA, type\}) \ AType \Rightarrow bool \\
                                   == & \% x. \ case \ x \ of \\
                                   &  ABase \ pX1 \Rightarrow True \ | \\
                                   &  AStep \ pX2 \ pX1 \Rightarrow False \\
AType\_astep\_def : & \\
      astep :: & ('a::\{ClassA, type\}, 'b::\{ClassA, type\}) \ AType \Rightarrow bool \\
      == & default\_astep\\
\end{array}$$

\subsection{Equality}

At the moment equality is the only covered built-in class.  The
axiomatizations provided, respectively, in \emph{HsHOLCF} and
\emph{HsHOL} are based on the abstract definition of the equality and
inequality functions \cite{HaskellRep}. In both theories, \emph{Eq} is
declared as a subclass of \emph{type} --- in \emph{HsHOLCF} this is
done in order to instantiate it with lifted types.\\

\noindent $consts$ 
$$ \begin{array}{ll}
hEq :: & ('a::Eq) lift \ \to \ 'a \ lift \ \to \ tr \\
hNEq :: & ('a::Eq) lift \ \to \ 'a \ lift \ \to \ tr\\
\end{array}$$

\noindent $axioms$ 
$$\begin{array}{ll}
axEq: & ALL x. (hEq \cdot p \cdot q \ = \ Def x) \ = \\
 & (hNEq \cdot p \cdot q \ = \ Def (\sim x)) \\
\end{array}$$

\noindent The instantiation of equality (consequently, of inequality) for
boolean (and similarly for integer) is obtained by lifting HOL
equality so that $\bot$ is returned whenever one of the argument is
undefined.\\

\noindent $tr\_hEq\_def: \ hEq \ == \ fliftbin \ (\% (a::bool) \ b. \ a \ = \ b)$\\


%ax1: & hEq \cdot p \cdot p \ == \ Def \ true \\
%ax2: & [| \ hEq \cdot q \cdot r \ = \ Def \ true; \ hEq \cdot p \cdot q \ = \ Def \ true \ |] \\
% & ==> \ hEq \cdot p \cdot r \ = \ Def \ true \\

\noindent In \emph{HsHOL} the axiomatization reflects the
restriction to terminating programs.\\

\noindent $consts$ 
$$\begin{array}{ll}
hEq :: & ('a::Eq) \ \Rightarrow \ 'a \ \Rightarrow \ bool \\
hNEq :: & ('a::Eq) \ \Rightarrow \ 'a \ \Rightarrow \ bool \\
\end{array}$$

\noindent $axioms$ 
$$\begin{array}{ll}
 axEq: & hEq \ p \ q \ == \ \sim hNEq \ p \ q \\
\end{array}$$

\noindent The instantiation of \emph{hEq} for boolean and integer is
simply taken to be HOL equality.

% ax1: & hEq \ p \ p \\
% ax2: & [| \ hEq \ q \ r; \ hEq \ p \ q \ |] ==> hEq \ p \ r \\

\subsection{Monads}
\label{sec:Monads}

Isabelle does not allow for classes of type constructors, hence a
problem in representing monads. We could deal with this problem
relying on an axiomatization of monads that allows for the
representation of monadic types as an axiomatic class, as presented in
\cite{Lueth}. Monadic types should be translated to newly defined
types that satisfy monadic axioms. This would involve defining a
theory morphism, as an instantiation of type variables in the theory
of monads. We are planning to rely on AWE \cite{AWE}, an
implementation of theory morphism on top of Isabelle base logic that
may be used to extend HOL as well.


\section{Conclusion}

The following is a list of features that are covered by our
translations.

\begin{itemize} 
\item predefined types: boolean, integer.
\item predefined type constructors: function, cartesian product, list.
\item declaration of recursive datatype, including mutually recursive ones.
\item predefined functions: equality, booelan constructors,
  connectives, list constructors, head and tail list functions,
  arithmetic operators.
\item non-recursive functions, including conditional, \emph{case} and
  \emph{let} and expressions (with restriction related to use of
  patterns).
\item use of incomplete patterns (in HOLCF) and of wildcards in case
  expressions.
\item total primitive recursive functions (in HOL)
and partial recursive ones (in HOLCF), including mutual ones (with
restrictions in the HOL case).
\item single-parameter class and instance declarations.
% \item monad declarations and do notation.
\end{itemize}


The shallow embedding approach makes it possible to take the most out
of the automation currently available on Isabelle, especially in HOL.
Further work should include extending the translation to cover the
whole of the Haskell Prelude. We would also be interested in carrying
out an extension to cover P-logic \cite{KiebPl}, a specification
formalism for Haskell programs included in the Programatica toolset.

%This translation, due to the character of the P-logic
%typing system, could be more easily carried out relying on some form
%of universal quantification on type variables, or else further relying
%on parametrisation.

\bibliographystyle{alpha}
\bibliography{case}

\end{document}











In our example,
translation gives the
following.\\






\noindent $consts$
$$ \begin{array}{ll} 
fun1 & :: \ ('a::type \Rightarrow 'c::type) \ \Rightarrow \ ('b::type \Rightarrow 'd::type) \ \Rightarrow \\ 
    & ('a::type, 'b::type) \ Prelude\_AType \ \Rightarrow \\ & ('c::type, 'd::type) \ Prelude\_AType \\
fun1\_Xfun2\_X & :: \ ('a::type \Rightarrow 'c::type) \ \Rightarrow \\ 
 & (('bXX1::type \Rightarrow 'dXX1::type) \ \Rightarrow \\ & ('aXX1::type, 'bXX1::type) \ Prelude\_AType \ \Rightarrow \\ & ('cXX1::type, 'dXX1::type) \ Prelude\_AType) * \\ 
 &  (('bXX2::type \Rightarrow 'dXX2::type) \ \Rightarrow \\ & ('aXX2::type, 'bXX2::type) \ Prelude\_BType \ \Rightarrow \\ & ('cXX2::type, 'dXX2::type) \ Prelude\_BType) \\
fun2 & :: \ ('a::type \Rightarrow 'c::type) \ \Rightarrow \ ('b::type \Rightarrow 'd::type) \ \Rightarrow \\ 
  & ('a::type, 'b::type) \ Prelude\_BType \ \Rightarrow \\ & ('c::type, 'd::type) \ Prelude\_BType \\
\end{array}$$

\noindent $defs$
$$ \begin{array}{ll}
fun1\_def : & fun1 \ == \ \% f. \% g. \% k. \ fst (fun1\_Xfun2\_X \ f) \ g \ k \\
fun2\_def : & fun2 \ == \ \% f. \% g. \% k. \ snd (fun1\_Xfun2\_X \ f) \ g \ k
\end{array}$$

\noindent $primrec$
$$ \begin{array}{ll}
 fun1\_Xfun2\_X \ (ABase \ pX1) & = \ (\% g. \% k. \ ABase \ (f \ pX1)) \\
         fun1\_Xfun2\_X \ (BBase \ pX1) & = \ (\% g. \% k. \ BBase \ (g \ pX1)) \\
         fun1\_Xfun2\_X \ (AStep \ pX1 \ pX2) & = \ (\% g. \% k. \ AStep \\ 
         & (fst \ (fun1\_Xfun2\_X \ f) \ g \ pX1) \\
         & (snd \ (fun1\_Xfun2\_X \ f) \ g \ pX2)) \\
         fun1\_Xfun2\_X \ (BStep \ pX1 \ pX2) & = \ (\% g. \% k. \ BStep \\ 
         & (snd \ (fun1\_Xfun2\_X \ f) \ g \ pX1) \\
         & (fst \ (fun1\_Xfun2\_X \ f) \ g \ pX2)) \\
\end{array}$$




\subsection{Equality: HOLCF}



\subsection{Equality: HOL}




...

...

Haskell classes and instances are translated to Isabelle,
respectively, as classes with empty axiomatisation and as instances
with trivial instantiation proof.  The translation supports only
single-parameter classes --- in principle, more than one parameter
could be handled using tuples.  

Built-in classes --- Eq and Ord in particular --- are translated to
newly defined classes in Isabelle; the corresponding axiomatisation
however here is not provided !!!

...

Functions declarations associated to classes are translated to
Isabelle as independent function declarations with appropriate class
annotation. Function definitions associated to instances are
translated as overloaded function definitions, relying on class
annotation of typed variables.

...


% ...xxx...!!! check out for stuff to include in Classes ---

Non-recursive definitions are translated to ordinary definitions (keyword
\emph{defs}), whereas recursive ones require specific keywords. The
type of the function, inclusive of class annotation, is included in
every definitions in order to allow for overloading.
%For values of
%built-in type, the lifting function is \emph{Def}; the undefined case
%collapses on $\bot$.

...


...

In HOLCF, where every type is a domain and every function is
continuous, all recursive functions can be defined by fixpoint
operator.  This is essentially a function that, given as argument the
function definition body abstracted of the recursive call name,
returns the corresponding recursive function. Coding this directly
would be rather cumbersome, particularly in the case of mutually
recursive functions, where tuples of definition bodies and tupled
abstraction are needed. We can instead rely on the \emph{recdef}
package, which allows to handle fixpoint definitions like ordinary
recursive ones, offering nice syntax to deal with mutual recursion as
well.

...

\subsection{HOL: Type signature}

The present translation to HOL is very basic, and does take into
account neither partiality nor laziness. Essentially, it left to the
Haskell programmer to check that all the functions are total ones.

An account of partiality -- although possibly not aof laziness ---
could be obtained using the option type constructor to lift types,
along a similar line to lift. Here however we are presenting just a
plain version, mapping Haskell types in the corresponding unlifted HOL
types. All variables belong to the class type. Recursive and mutually
recursive data-types declarations are translated to HOL datatype
declaration.

...

\subsection{HOL: Function definitions}

Non-recursive definitions and let expressions are treated similarly as
in the translation to HOLCF.  Recursive defitions is the topic that
really sets HOL and HOLCF apart.  In HOL a distinction has to be drawn
between primitive recursive functions and generic recursive ones. In
the former, termination is guaranteed by the fact that recursive
definitions are strictly associated with the datatype structure of a
parameter. In the latter, termination needs to be proved. This must be
done, in a \emph{recdef} definition, by providing a measure for the
function parameter that can be proved to be strictly decreasing for
each occurrence of a recursive call in the definition. This requires a
degree of ingenuity that cannot be generally associated with an
automated translation.

For this reason, the translation of recursive functions to HOL is
restricted to primitive recursive ones. Mutually recursive functions
are allowed under additional restrictions.  These are: all the
functions should be recursive in the first argument; this should be of
the same type for each of them; the constructors should be listed in
the same order in each of the case expressions; the variable names in
the case patterns are expected to be maintained.

The translation of mutually recursive functions $a \rightarrow b, \ldots a
\rightarrow d$ introduces a new function $a \rightarrow (b * \ldots * d)$
defined, for each case pattern, as the product of the values
correspondingly taken by the original functions.

...

\subsection{Pattern matching}

Support of pattern-matching in Isabelle is more restricted than in
Haskell. Nested patterns are disallowed; case expressions are
sensitive to the order of the constructors, which should be the same
as in the datatype declaration. In particular, the case variable si
required to be a variable. Translation of pattern-matching is
potentially laborious. For this reason guarded expressions, list
comprehension and nested pattern-matching have been left out; they
should be replaced by the programmer, using conditional expressions
and map functions.

Function definition by top level pattern matching is not allowed in
Isabelle with ordinary definitions, but it is with those that are
explicitly declared to be recursive ones. However, particularly in HOL
primitive recursive definitions patterns are allowed in one parameter
only. In order to translate function definitions with patterns in more
than one parameter, without resorting to using tuples and more complex
definitions (\emph{recdef} instead of \emph{primrec}), our
translations turn a multiple definition by top level pattern matching
into a definition by case construct.

...

\subsection{Classes}


Haskell classes and instances are translated to Isabelle,
respectively, as classes with empty axiomatisation and as instances
with trivial instantiation proof.  The translation supports only
single-parameter classes --- in principle, more than one parameter
could be handled using tuples.

Built-in classes --- Eq and Ord in particular --- are translated to
newly defined classes in Isabelle; the corresponding axiomatisation
however here is not provided !!!

...

Functions declarations associated to classes are translated to
Isabelle as independent function declarations with appropriate class
annotation. Function definitions associated to instances are
translated as overloaded function definitions, relying on class
annotation of typed variables.

...



\section*{Haskell2Isabelle}

The tool is designed to support the translation of simple but
realistic Haskell programs to HOLCF and, with some restriction, to
HOL.  Not all the syntactical features of Haskell are covered yet,
although most of those used in the Prelude are. The most noticeable
limitations are related to built-in types, pattern-matching and local
definitions.  There are additional and more substantial restrictions
in the case of HOL, related to termination and to recursion.

The application can be installed as part of Hets. It requires Haskell
GHC and Programatica. It is run by the command \emph{hets -t ...
  [h|hc] file.hs} where the options stand for HOL and HOLCF,
respectively.

For the sake of maintainance, \emph{Haskell2IsabelleHOLCF.hs} is the
module that contains the main translation function, carried out as
composition of a signature morphism and a theory morphism.
\emph{IsaSign.hs} contains the Hets representation of the Isabelle
base logic and extensions. 

The general design of the translation functions, and particularly the
two-layered recursion adopted for them, is similar to that used for
the translation Haskell to Agda in Programatica.

\subsection{Monads}

Isabelle does not allow for classes of type constructors, hence a
problem in representing monads. We can deal with this problem,
however, relying on an axiomatisation of monads that allows for the
presentation of monadic types as an axiomatic class, as provided in
\cite{}. Therefore, monadic types can be translated to newly defined
types that are constrained to satisfy the monadic axioms. This
constraining, carried out as an instantiation of type variables in the
theory of monads, takes the form of a theory morphism, and can be
carried out automatically by AWE, an implementation of parametrisation
and theory morphism on top of Isabelle basic logic, that may also be
used to extend HOL and HOLCF. The do notation can then be translated
by unpacking it into monadic operators. Attempting to translate
monadic notation without support AWE support will result in an error
message.

\section*{Conclusion}


The following is a list of constructs that are covered by all the
translations.

\begin{itemize} 
\item predefined types: boolean, integer.
\item predifed type constructors: funtion, cartesian product, list.
\item declaration of recursive datatype, including mutually recursive ones.
\item predefined functions: equality, booelan constructors,
  connectives, list constructors, head and tail list functions,
  arithmetic operations.
\item non-recursive functions, including conditional, 
let and case expressions.
\item use of incomplete patterns (HOLCF) and of wildcards in case
  expressions.
\item total primitive recursive functions (HOL)
and partial recursive ones (HOLCF), including mutual ones (with
restrictions in the HOL case).
\item class and instance declarations. 
\item monad declarations and do notation.
\end{itemize}


The shallow embedding approach used by our translations should allow
to take as much advantage as possible of the automation currently
available on Isabelle, particularly in HOL. 

Further work should include extending the translation to cover the
whole of the Haskell Prelude. We hope that this will make it possible
to support the verification of functional robotics programs.

A further extension, in line with the work on Programtica, would be
the translation of P-logic. This translation, due to the character of
the P-logic typing system, could be more easily carried out relying on
some form of universal quantification on type variables, or else
further relying on parametrisation.

\bibliographystyle{alpha}
\bibliography{case}

\end{document}









\subsection{}

The translation of types is 

Our general strategy for translating signatures is to map Haskell
types to Isabelle defined types.


 the fact that Haskell expressions are
evaluated lazily, however it does not preserve The translation to HOL
does not, and moreover requires that all the functions defined in the
Haskell program are total ones...

All types have a sort in Isabelle, defined by the set of the classes
of which it is a member. In HOL, all types belong to the class type;
in HOLCF, they belong to class pcpo (pointed complete poset).

All Haskell function declarations are translated to Isabelle ones.
Type variables are assigned default sort type in HOL and pcpo in
HOLCF.  Names get translated by a function that preserves them, up to
avoidance of clashes with Isabelle keywords.

The only built-in types that are properly covered are booleans,
unbounded integers and rationals. These get translated, in the case of
HOL, to Isabelle booleans, integers and rationals, respectively. Bound
integers and floating point numbers would require low-level modelling,
and have not been dealt with; anyway, bounded integers are accepted
and simply treated as unbounded by the application.

In the HOLCF translation, all built-in types get lifted to the
corresponding domains (pcpos), which by definion include the undefined
value, by using the lift HOLCF type constructor. In particular, type
boolean get translated to type tr (short for bool lift), and similarly
for types integer and rational.

Recursive data-types declarations can be translated to datatype
declarations in HOL. In HOLCF they can be translated to domain
declarations. The two structure are quite similar, except that in
domains, which are pcpo, all the parameters are required to be pcpos
as well. Morevover, Isabelle domain declarations require to introduce
names for destructors. Mutually recursive Haskell datatypes can be
translated to both logics, relying on the specific Isabelle syntax for
them.


A type class in Isabelle is conceptually quite different from one in
Haskell. The former is associated to a set of axioms, whereas the
latter is associated to a set of function declarations. Isabelle
classes may have at most a type parameter.  Moreover, Isabelle does
not allow for type constructor classes. The last limitation is the
most serious of the three, since it makes quite hard to cope with
monads.  We can get around this obstacle relying on a method for the
axiomatic encoding of monads in HOL that uses parametrisation.
Parametrisation based on theory morphism has been implemented, as a
conservative extension of various Isabelle logics, in AWE, a tool
designed to support proof reuse \cite{AWE}.









...

The \emph{lift} type constructor, in particular, lifts each HOL type
to a flat domain, by adding a bottom element to them. We are going to
use it to lift the basic types. For the product we are going to use
non-strict product.

...

Its
formalisation on Isabele relies on axiomatic type classes \cite{},
with \emph{pcpo} (pointed complete partial order) as the default
class, in which the type of continuous functions can be represented as
function spaces, allowing for an elegant interpretation of recursion
by fixpoint operator. HOL types may be lifted to \emph{pcpo} by simply
adding a bottom element to them. This is what the type constructor
\emph{lift} does, thus providing also a way of handling partiality ---
undefined cases can be mapped to the bottom element. HOLCF has also a
type constructor \emph{u} that allows to deal with laziness, by
guaranteeing $\bot \neq \smath{up} \$ (\bot)$.
  

\subsection{}


In the case of HOLCF, the logic provides a computational
interpretation of function types as domains. In HOL there is no such
facility, therefore either appropriate types for the translation are
introduced --- as in \cite{}, or considerable semantical restrictions
must be accepted --- this is also the case for one of the translation
to FOL proposed in \cite{}.

Depending on the expressiveness of the logic, the translation of the
programming language may be carried out at different levels of depth.
In general, the deeper is the embedding, the less it relies on
meta-level features of the target language. This may be a plus, either
from the point of view of semantic clarity, or from that of
expressiveness. Taking advantage of meta-level feaures, on the other
hand, may be useful in order to make the theorem proving less tedious,
by allowing more constant use of generic proof methods. It may also
ultimately affect positively the theorem prover economy, by
reinforcing the bias toward the development of proof methods that,
however specific, can be more easily integrated with generic ones.

% a point made in \cite{}. for example by allowing a straightforward
% operational interpretation ---

The translations of Haskell to HOLCF and HOL that we are presenting in
the following justify themselves in terms of denotational semantics.
The translation to HOLCF differs from that proposed in \cite{} in the
treatment of types. We translate Haskell types directly into HOLCF
ones, whereas they use HOLCF terms to embed Haskell types. We assume
that the HOLCF types are similar enough to the those of Haskell, in
order to guarantee the semantical correctness of our translations.
Moreover, we rely on axiomatic classes to translate Haskell types
classes.

Our approach has the advantage of giving a shallower embedding,
however it has the drawback of not bein complete --- notably, type
constructor classes are missing. We plan to handle in future this
problem (see section~\ref{sec:Monads}) by relying on an extension of
Isabelle based on morphism between theories. The translation into HOL
is essentialy only a first experiment, it handles types very
simplistically, relying on drastic semantical restrictions for
correcteness. Under those restrictions, however, it gives expressions
that are comparatively quite simple to work with.

A type class in Isabelle is conceptually quite different from one in
Haskell. The former is associated to a set of axioms, whereas the
latter is associated to a set of function declarations. Isabelle
classes may have at most a type parameter.  Moreover, Isabelle does
not allow for type constructor classes. The last limitation is the
most serious of the three, since it makes quite hard to cope with
monads.  We can get around this obstacle relying on a method for the
axiomatic encoding of monads in HOL that uses parametrisation.
Parametrisation based on theory morphism has been implemented, as a
conservative extension of various Isabelle logics, in AWE, a tool
designed to support proof reuse \cite{AWE}.



\subsection{Types}

HOLCF allows has types for computable functions --- defined as sets of
continuous functions of a type (!! fixpoints!!). It has a type
constructor \emph{lift} that lifts HOL types to \emph{pcpo}s, thus
providing a way of handling partiality --- undefined cases can be
mapped to the bottom element. It has also a type constructor \emph{u}
that allows to deal with laziness, by guaranteeing $\bot \neq
\smath{u}(\bot)$.



that can be used to embed Haskell function types.


The translation to HOLCF differs from that to HOL insofar as only the
first one relies on a denotational semantics for the full language,
keeping into account partiality and lazy evaluation, and associating
recursive functions to corresponding fixed points, using on the Fixrec
extension of HOLCF. The translation to HOL is more restrictive; it
does not keep into account any form of partiality, and it restricts
translation of recursive functions to the primitive recursive ones.
It would have been possible to give an account of partiality in HOL by
using option types. However, we chose to keep the HOL translation as
simple as possible for didactic purpose.



\subsection{maybe}

The translations of Haskell that we are presenting in the following,
we present translations of Haskell into some of the Isabelle logics.
These translations are implemented as part of Hets, and are based on a
shallow embedding approach quite different from that used in
Programatica. In our translations Haskell types are mapped to Isabelle
types. We are assuming that the type systems of HOL and HOLCF are
similar enough to the type system of Haskell, in order to guarantee
the semantical correctness of our translations. Moreover, we rely on
axiomatic classes to translate Haskell types classes.

A type class in Isabelle is conceptually quite different from one in
Haskell. The former is associated to a set of axioms, whereas the
latter is associated to a set of function declarations. Isabelle
classes may have at most a type parameter.  Moreover, Isabelle does
not allow for type constructor classes. The last limitation is the
most serious of the three, since it makes quite hard to cope with
monads.  We can get around this obstacle relying on a method for the
axiomatic encoding of monads in HOL that uses parametrisation.
Parametrisation based on theory morphism has been implemented, as a
conservative extension of various Isabelle logics, in AWE, a tool
designed to support proof reuse \cite{AWE}.

The translation to HOLCF differs from that to HOL insofar as only the
first one relies on a denotational semantics for the full language,
keeping into account partiality and lazy evaluation, and associating
recursive functions to corresponding fixed points, using on the Fixrec
extension of HOLCF. The translation to HOL is more restrictive; it
does not keep into account any form of partiality, and it restricts
translation of recursive functions to the primitive recursive ones.
It would have been possible to give an account of partiality in HOL by
using option types. However, we chose to keep the HOL translation as
simple as possible for didactic purpose.



\subsection{Type signature}

Our general strategy for translating signatures is to map Haskell
types to Isabelle defined types. The translation to HOLCF keeps into
account the fact that Haskell expressions are evaluated lazily. The
translation to HOL does not, and moreover requires that all the
functions defined in the Haskell program are total ones...

All types have a sort in Isabelle, defined by the set of the classes
of which it is a member. In HOL, all types belong to the class type;
in HOLCF, they belong to class pcpo (pointed complete poset).

All Haskell function declarations are translated to Isabelle ones.
Type variables are assigned default sort type in HOL and pcpo in
HOLCF.  Names get translated by a function that preserves them, up to
avoidance of clashes with Isabelle keywords.

The only built-in types that are properly covered are booleans,
unbounded integers and rationals. These get translated, in the case of
HOL, to Isabelle booleans, integers and rationals, respectively. Bound
integers and floating point numbers would require low-level modelling,
and have not been dealt with; anyway, bounded integers are accepted
and simply treated as unbounded by the application.

In the HOLCF translation, all built-in types get lifted to the
corresponding domains (pcpos), which by definion include the undefined
value, by using the lift HOLCF type constructor. In particular, type
boolean get translated to type tr (short for bool lift), and similarly
for types integer and rational.

Recursive data-types declarations can be translated to datatype
declarations in HOL. In HOLCF they can be translated to domain
declarations. The two structure are quite similar, except that in
domains, which are pcpo, all the parameters are required to be pcpos
as well. Morevover, Isabelle domain declarations require to introduce
names for destructors. Mutually recursive Haskell datatypes can be
translated to both logics, relying on the specific Isabelle syntax for
them.









we present translations of Haskell into some of the Isabelle logics.
These translations are implemented as part of Hets, and are based on a
shallow embedding approach quite different from that used in
Programatica. In our translations Haskell types are mapped to Isabelle
types. We are assuming that the type systems of HOL and HOLCF are
similar enough to the type system of Haskell, in order to guarantee
the semantical correctness of our translations. Moreover, we rely on
axiomatic classes to translate Haskell types classes.


In the following, we present translations of Haskell into some of the
Isabelle logics. These translations are implemented as part of Hets,
and are based on a shallow embedding approach quite different from
that used in Programatica. In our translations Haskell types are
mapped to Isabelle types. We are assuming that the type systems of HOL
and HOLCF are similar enough to the type system of Haskell, in order
to guarantee the semantical correctness of our translations. Moreover,
we rely on axiomatic classes to translate Haskell types classes.




 are those presented in \cite{}.



as well as in \cite{}

translation of Haskell
into FOL \cite{} and that into

In \cite{} Haskell is embedded in Martin-Loef
type theory, implemented on Agda.

 is indeed a general difference between a lower level
approach based on operational semantics, followed in \cite{agda} and
\cite{miranda}, and a higher-level one, based essentially on
denotational semantics, followed in \cite{}


The correctness of such embeddings
always rests, ultimately, on the denotational semantics of Haskell,
and on the possibility to encode it in a logic under consideration.


\cite{Thompson95}.  Indeed, Haskell --- a strongly typed, purely
functional language with lazy evaluation, based on a system of
polymorphic types extended with type constructor classes, where monads
allow for side effects and pseudo-imperative code \cite{Hudak} --- has
already been considered several times as a target for verification
\cite{...}.

All the above-mentioned works provide embeddings of Haskell, either
partial or total, into programming logic implemented on theorem
provers. The correctness of such embeddings always rests, ultimately,
on the denotational semantics of Haskell, and on the possibility to
encode it in a logic under consideration.

The tools, the logic and the ways in which the encoding can be done
may differ quite deeply. In \cite{} Haskell is embedded in Martin-Loef
type theory, implemented on Agda. All the other examples rely on
Isabelle, a generic theorem-prover written in SML that allows for the
formalisation of a variety of logics, including FOL, HOL and HOLCF
\cite{Paulson94isa}. The embedding in \cite{} rely on FOL, the
Isabelle implementation of classical first-order logic.

All the other examples \cite{...} rely on higher-order logic. HOL is
the implementation of classical higher-order logic based on simply
typed lambda calculus, extended with axiomatic type classes.  HOLCF
\cite{holcf} is the extension of HOL with a formalisation of the
theory of computable functions (LCF) based on axiomatic type classes.




%  In
% other words, we need to represent the program in a mechanised logic in
% which the requirements can be expressed, as well.
% The representation can take different forms. It is possible to
% associate the state-machine behavioural description to models in a
% logic. Another possibility is to represent the program in terms of a
% computational logic based on the operational semantics of the
% language. 
% In both cases however, a significant part of the work has to
% be carried out before the actual verification may take place.

In order to verify a program, we need to associate its code to a
logical representation that allows for requirements to be expressed
and for proofs to be carried out. There are different ways to do this.
The way we carry this out here is to embed Haskell into an expressive
higher-order logic, together with providing an implementation of the
embedding that allows for an automated translation of Haskell. From
the theoretical point of view, the correctness of the embdedding
depends on the denotational semantics of the language.  From the
practical point of view, once the program code can be translated
automatically to the logic, all the burden of the verification can
rest on the theorem proving.

Depending on the expressiveness of the logic, the translation of the
progrmaming language may be carried out at different levels of depth.
In general, the deeper is the embedding, the less it relies on
meta-level features of the target language. This may be a plus from
the point of view of semantic clarity. Taking advantage of meta-level
feaures, on the other hand, may be useful in order to make the theorem
proving less tedious.

Isabelle is a generic theorem-prover written in SML that allows for
the formalisation of a variety of logics \cite{Paulson94isa}.
Isabelle-HOL is the implementation of classical higher-order logic
based on simply typed lambda calculus, extended with axiomatic type
classes. Isabelle-HOLCF \cite{holcf} is the extension of HOL with a
formalisation of the theory of computable functions (LCF) based on
axiomatic type classes.

Hets \cite{Hets} is a parsing, static analysis and proof management
tool designed to interface with various language specific tools, in
order to support heterogeneous specification. In particular, for the
parsing and static analysis of Haskell programs Hets relies on
Programatica \cite{PTeam}, a tool that has its own proof management,
including a translation of Haskell to Isabelle-HOLCF.

The translations of Haskell that we are presenting in the following,
we present translations of Haskell into some of the Isabelle logics.
These translations are implemented as part of Hets, and are based on a
shallow embedding approach quite different from that used in
Programatica. In our translations Haskell types are mapped to Isabelle
types. We are assuming that the type systems of HOL and HOLCF are
similar enough to the type system of Haskell, in order to guarantee
the semantical correctness of our translations. Moreover, we rely on
axiomatic classes to translate Haskell types classes.

A type class in Isabelle is conceptually quite different from one in
Haskell. The former is associated to a set of axioms, whereas the
latter is associated to a set of function declarations. Isabelle
classes may have at most a type parameter.  Moreover, Isabelle does
not allow for type constructor classes. The last limitation is the
most serious of the three, since it makes quite hard to cope with
monads.  We can get around this obstacle relying on a method for the
axiomatic encoding of monads in HOL that uses parametrisation.
Parametrisation based on theory morphism has been implemented, as a
conservative extension of various Isabelle logics, in AWE, a tool
designed to support proof reuse \cite{AWE}.

The translation to HOLCF differs from that to HOL insofar as only the
first one relies on a denotational semantics for the full language,
keeping into account partiality and lazy evaluation, and associating
recursive functions to corresponding fixed points, using on the Fixrec
extension of HOLCF. The translation to HOL is more restrictive; it
does not keep into account any form of partiality, and it restricts
translation of recursive functions to the primitive recursive ones.
It would have been possible to give an account of partiality in HOL by
using option types. However, we chose to keep the HOL translation as
simple as possible for didactic purpose.

These translations are on the overall quite different from other ones
based on deep embeddings, particularly from the Programatica
translation of Haskell into HOLCF \cite{Huff}, where the type system
is modelled at the logical level, as well as from translations of
Miranda and Haskell into FOL (Isabelle classical first-order logic)
where higher-order features have to be dealt with less directly
\cite{Thompson95,Thompson92}.



\section*{The hs2thy tool}

(suggested change: from h2hf to hs2thy)
The tool is designed to support the translation of simple but
realistic Haskell programs to HOLCF and, with some restriction, to
HOL.  Not all the syntactical features of Haskell are covered yet,
although most of those used in the Prelude are. The most noticeable
limitations are related to built-in types, pattern-matching and local
definitions.  There are additional and more substantial restrictions
in the case of HOL, related to termination and to recursion.

The application can be installed as part of Hets. It requires Haskell
GHC and Programatica. It is run by the command \emph{h2hf
  [h|hc|mh|mhc] file.hs} where the options stand, respectively, for
HOL, HOLCF, HOL extended with AWE and HOLCF extended with AWE.

For the sake of maintainance, \emph{Haskell2IsabelleHOLCF.hs} is the
module that contains the main translation function, carried out as
theory translation after a preliminary step of signature translation,
and \emph{IsaSign.hs} contains the internal representation of Isabelle
logics.

\subsection{Type signature}

Our general strategy for translating signatures is to map Haskell
types to Isabelle defined types. The translation to HOLCF keeps into
account the fact that Haskell expressions are evaluated lazily. The
translation to HOL does not, and moreover requires that all the
functions defined in the Haskell program are total ones...

All types have a sort in Isabelle, defined by the set of the classes
of which it is a member. In HOL, all types belong to the class type;
in HOLCF, they belong to class pcpo (pointed complete poset).

All Haskell function declarations are translated to Isabelle ones.
Type variables are assigned default sort type in HOL and pcpo in
HOLCF.  Names get translated by a function that preserves them, up to
avoidance of clashes with Isabelle keywords.

The only built-in types that are properly covered are booleans,
unbounded integers and rationals. These get translated, in the case of
HOL, to Isabelle booleans, integers and rationals, respectively. Bound
integers and floating point numbers would require low-level modelling,
and have not been dealt with; anyway, bounded integers are accepted
and simply treated as unbounded by the application.

In the HOLCF translation, all built-in types get lifted to the
corresponding domains (pcpos), which by definion include the undefined
value, by using the lift HOLCF type constructor. In particular, type
boolean get translated to type tr (short for bool lift), and similarly
for types integer and rational.

Recursive data-types declarations can be translated to datatype
declarations in HOL. In HOLCF they can be translated to domain
declarations. The two structure are quite similar, except that in
domains, which are pcpo, all the parameters are required to be pcpos
as well. Morevover, Isabelle domain declarations require to introduce
names for destructors. Mutually recursive Haskell datatypes can be
translated to both logics, relying on the specific Isabelle syntax for
them.


\subsection{Function definitions}

Haskell function definitions are translated to corresponding Isabelle
ones. Non-recursive definitions are translated to ordinary definitions
(keyword \emph{defs}), whereas recursive ones require specific
keywords. The type of the function, inclusive of class annotation, is
included in each of its definitions in order to allow for overloading.
In the translation to HOLCF, when the Haskell value is of built-in
type, the translated value has the lifted corresponding type --- this
is either the case of a lifted value (where \emph{Def} is the lifting
function) or the case of the undefined value (\emph{UU}).

As to local definitions, they can be introduced in Haskell by let and
where expressions. Let expressions are translated to Isabelle let
expressions, whereas where expressions are not covered.


\subsection{Recursive definitions}

Recursive defitions is the topic that really sets HOL and HOLCF apart.
In HOLCF, where every type is a domain and every function is
continuous, all recursive functions can be defined by fixpoint
operator.  This is essentially a function that, given as argument the
function definition body abstracted of the recursive call name,
returns the corresponding recursive function. Coding this directly
would be rather cumbersome, particularly in the case of mutually
recursive functions, where tuples of definition bodies and tupled
abstraction are needed. We can instead rely on the \emph{recdef}
package, which allows to handle fixpoint definitions like ordinary
recursive ones, offering nice syntax to deal with mutual recursion as
well.

The case of HOL is more compplex. There a distinction must be drawn
between primitive recursive functions and generic recursive ones. In
the formers, termination is guaranteed by the fact that recursive
definitions are strictly associated with the datatype structure of a
parameter. In the latters, termination needs to be proved. This can be
done, in a \emph{recdef} definition, by providing a measure for the
function parameter that can be proved to be strictly decreasing for
each occurrence of a recursive call in the definition. This requires a
generally unbounded amount of ingenuity and cannot be carried out by a
simple translation.

For this reason, the translation of recursive functions to HOL is
restricted to primitive recursive ones. Mutually recursive functions
are allowed under additional restrictions.  These are: all the
functions should be recursive in the first argument, this should be of
the same type for each of them, the constructors should be listed in
the same order in each of the case expressions, and also the variable
names in the case patterns are expected to be maintained.

The translation of mutually recursive functions $a \rightarrow b, \ldots a
\rightarrow d$ introduces a new function $a \rightarrow (b * \ldots * d)$
defined, for each case pattern, as the product of the values
correspondingly taken by the original functions.


\subsection{Pattern matching}

Support of pattern-matching in Isabelle is more restricted than in
Haskell. Nested patterns are disallowed; case expressions are
sensitive to the order of the constructors, which should be the same
as in the datatype declaration. In particular, the case variable si
required to be a variable. Translation of pattern-matching is
potentially laborious. For this reason guarded expressions, list
comprehension and nested pattern-matching have been left out; they can
be replaced without loss, anyway, using conditional expressions and
map functions.

Function definition by top level pattern matching is not allowed in
Isabelle with ordinary definitions, but it is with those that are
explicitly declared to be recursive ones. However, particularly in HOL
primitive recursive definitions patterns are allowed in one parameter
only. In order to translate function definitions with patterns in more
than one parameter, without resorting to using tuples and more complex
definitions (\emph{recdef} instead of \emph{primrec}), our
translations turn a multiple definition by top level pattern matching
into a definition by case construct.

\subsection{Classes}

Haskell classes and instances are translated to Isabelle,
respectively, as classes with empty axiomatisation and as instances
with trivial instantiation proof.  The translation supports only
single-parameter classes --- in principle, more than one parameter
could be handled using tuples.  

Built-in classes --- Eq and Ord in particular --- are translated to
newly defined classes in Isabelle; the corresponding axiomatisation
however here is not provided.

Functions declarations associated to classes are translated to
Isabelle as independent function declarations with appropriate class
annotation. Function definitions associated to instances are
translated as overloaded function definitions, relying on class
annotation of typed variables.


\subsection{Monads}
\label{sec:Monads}

Isabelle does not allow for classes of type constructors, hence a
problem in representing monads. We can deal with this problem,
however, relying on an axiomatisation of monads that allows for the
presentation of monadic types as an axiomatic class, as provided in
\cite{}. Therefore, monadic types can be translated to newly defined
types that are constrained to satisfy the monadic axioms. This
constraining, carried out as an instantiation of type variables in the
theory of monads, takes the form of a theory morphism, and can be
carried out automatically by AWE, an implementation of parametrisation
and theory morphism on top of Isabelle basic logic, that may also be
used to extend HOL and HOLCF. The do notation can then be translated
by unpacking it into monadic operators. Attempting to translate
monadic notation without support AWE support will result in an error
message.


\section*{Conclusion}


The following is a list of constructs that are covered by all the
translations.

\begin{itemize} 
\item predefined types: boolean, integer.
\item predifed type constructors: funtion, cartesian product, list.
\item declaration of recursive datatype, including mutually recursive ones.
\item predefined functions: equality, booelan constructors,
  connectives, list constructors, head and tail list functions,
  arithmetic operations.
\item non-recursive functions, including conditional, 
let and case expressions.
\item use of incomplete patterns (HOLCF) and of wildcards in case
  expressions.
\item total primitive recursive functions (HOL)
and partial recursive ones (HOLCF), including mutual ones (with
restrictions in the HOL case).
\item class and instance declarations. 
\item monad declarations and do notation.
\end{itemize}


The shallow embedding approach used by our translations should allow
to take as much advantage as possible of the automation currently
available on Isabelle, particularly in HOL. 

Further work should include extending the translation to cover the
whole of the Haskell Prelude. We hope that this will make it possible
to support the verification of functional robotics programs.

A further extension, in line with the work on Programtica, would be
the translation of P-logic. This translation, due to the character of
the P-logic typing system, could be more easily carried out relying on
some form of universal quantification on type variables, or else
further relying on parametrisation.


\bibliographystyle{alpha}
\bibliography{case}

\end{document}


%Moreover, being such languages
%as Haskell, ML and Miranda closely modelled on typed lambda-calculus,
%their translations to specification formalisms, particularly to
%higher-order logics, turn out to be comparatively straighforward to be
%defined in the main lines --- although not so to be carried out in
%detail. 

%, due to the richness in syntax, the specific way evaluation is
%implemented, as well as to issues associated with classes (Haskell),
%overloading and side effects.

%All functions are total in HOL. Partiality may be dealt with by
%lifting types through the \emph{option} type constructor. HOL has an
%implementation of recursive functions based on Tarski fixed-point
%theorem. Defining a recursive function requires giving a specific
%measure and proving monotonicity about it. 

% HOL has an implementation of recursive functions, based on Tarski
% fixed-point theorem. Their definition in general requires specific
% measure to be given each time, for monotonicity to be proved.
% Moreover, partial functions are needed to programs that might not
% terminate, but in HOL all functions are total. Partiality may be
% dealt with by lifting types through \emph{option}, a type
% constructor in HOL. This inevitably complicates case expressions.

%There are different general approaches depending on the style of the
%semantics. The expressiveness of the logical language may also make a
%big difference.  











The Haskell classes Eq and Ord are translated to corresponding classes
declared, respectively, in HsHOL.thy and HsHOLCF.thy. These classes
for the moment are taken to be absolutely generic ones (with empty
axiomatisation).


\subsection{Isabelle-HOL}

Isabelle-HOL as an implementation of classical higher-order logic
supports the definition of datatypes and recursive functions.
However, it distinguishes between primitive recursion (keyword
emph{primrec}), in which termination is trivially guaranteed by
decrease in the complexity of a parameter, and recursion (keyword
emph{recdef}) for which an appropriate specific measure must be
provided in order to get the termination proof to go through.
Moreover, HOL does not have a notion of undefined suitable for
computations --- therefore all functions have to be total.

This translation of recursive functions is restricted to total
primitive recursive functions that are recursive in their first
datatype argument. The Haskell definition should be a case expression
in which all the constructors are considered in their datatype
declaration order, save for the use of a wildcard. 
%Wildcards cannot be used for constructor arguments.
%Moreover, the
%case variable should not be used in the specific case expression (just
%replace it with the specific case pattern).

Mutually recursive functions are allowed under additional
restrictions.  All the functions should be recursive in the same type,
the constructors should be listed in the same order in each of the
case expressions, and also the variable names in the case patterns are
expected to be maintained.

The translation of mutually recursive functions $a \rightarrow b, \ldots a
\rightarrow d$ introduces a new function $a \rightarrow (b * \ldots * d)$
defined, for each case pattern, as the product of the values
correspondingly taken by the original functions.

\subsection{Isabelle-HOLCF} 

Isabelle-HOLCF is the implementation of the theory of computable
functions on top of higher-order logic, relying on axiomatic classes.
In comparison with HOL, the handling of partial functions is more
natural, and the presence of the fixpoint operator allows for a
general treatment of recursive functions. However, types in HOLCF can
be interpreted as pointed complete partial orders (pcpo), and this
requires all standard types to be lifted.

In the present translation, the type constructor \emph{lift} is used
to lift types, in association with the corresponding lifting function
\emph{Def} and with the undefined value \emph{UU}.  Datatypes are
translated to domains --- complete with destructors from declaration.
The class \emph{pcpo} is added to the sort of each type variable.

Partial recursive functions (defined by case expressions), including
mutually recursive ones, are covered without any essential
restrictions, relying on the recdef package.

\end{document}




 as well as an
operational, but is quite deep.

, insofar as
it builds into HOL a specific domain class for the interpretation of


Higher-level approaches relying on denotational
semantics 

 propose level
approaches based on large-step operational semantics, even if with
some complication when dealing with higher-order programming languages
such as those presented in \cite{} --- Miranda in FOL; \cite{} ---
Haskell in FOL; \cite{} --- Haskell in the Agda implementation of
Martin-Loef type theory. More expressiveness is usually required by
high-level approaches based on denotational semantics, such as those
presented in \cite{} --- Haskell in HOLCF and HOL; \cite{} --- ML in
HOL.

The main one, is which Depending on the
expressiveness of the logic, the translation of the programming
language may be carried out at different levels of depth.  In general,
the deeper is the embedding, the less it relies on meta-level features
of the target language. This may be a plus, either from the point of
view of semantic clarity, or from that of expressiveness. Taking
advantage of meta-level feaures, on the other hand, may be useful in
order to make the theorem proving less tedious, by allowing more
constant use of generic proof methods.  It may also ultimately affect
positively the theorem prover economy, by reinforcing the bias toward
the development of proof methods that, however specific, can be more
easily integrated with generic ones.


Some significantly different examples of functional programming
languages embedded into the logic of a theorem prover can be found in
\cite{} (Miranda on Isabelle), \cite{} (ML on Isabelle), \cite{}
(Haskell on Isabelle), \cite{} (Haskell on Agda). Example of lower
level approaches are given in \cite{}, where Haskell is translated
into the Agda implementation of Martin-Loef theory based on
operational rules and relies on operational semantics for correctness,
is followed in \cite{}, where Haskell is embedded into the Agda
implementation of Martin-Loef theory, as well as in the embeddings of
Miranda \cite{} and Haskell \cite{} into FOL (classical first-order
logic, until the middle of the nineties still probably the best
supported logic on Isabelle). Higher-level approaches, such as
presented in \cite{} (embedding ML into HOL), and in \cite{}
(embedding Haskell into HOLCF and into HOL), rely ultimately on
denotational semantics for correctenss. This, together with the
expressiveness allowed by higher-order logic for the formulation of
requirements, gives the considerable advantage of enabling proofs that
are more abstract and closer to mathematical intuition.

\subsection{}



method for the
axiomatic encoding of monads in HOL that uses parametrisation.
Parametrisation based on theory morphism has been implemented, as a
conservative extension of various Isabelle logics, in AWE, a tool
designed to support proof reuse \cite{AWE}.

it is not complete, and cannot be trivially completed ---
notably, type constructor classes have not been considered, since
Isabelle does not have them. However, we are expecting to deal with
this problem by relying on an extension of Isabelle based on theory
morphism (see section \ref{}) .
