
\documentclass{article}
\title{Creating ATermConvertible-Instances with \emph{genRules} and \emph{DrIFT}} 
\author{Felix Reckers}
\begin{document}

\maketitle

The tool \emph{genRules} provides the ability to create automatically rules, which can 
be proceeded by \emph{DrIFT}. The directory \emph{ATC} includes the files with ATermConvertible
instances, which are necessary to read datatypes from and to write them to files in the 
Shared-ATerm-format.\\ 
To create a new ATermConvertible instance for the datatypes in one file or more, you can invoke 
\emph{genRules} in the directory \emph{utils}. It will generate a file with the specified 
rules for \emph{DrIFT}. Then you can invoke \emph{DrIFT} on that file to create the instances.
\emph{genRules} will generate a file with a \verb|.der.hs|-suffix, it will contain all necessary 
imports and the \emph{DrIFT}-rules for each datatype. If the input is one file, the name of the 
output file will be the same as the input plus the special suffix. If there is more than one 
input file, the output file will have the name of the first of them. Every specific logic 
instance is saved in the directory which represents that logic: CASL, HasCASL, Haskell and CspCASL.
These files begin with an
\emph{ATC\_} followed by the logic name. All the other ATC-files are saved in the directory \emph{ATC}.
\section*{Invoking \emph{genRules}}
The following options are available:
\begin{description}
\item[-r:] This option must be declared, it specifies the rule for generation. 
In this case it must be \emph{ShATermConvertible}.
\item[-o:] Specifies the output directory, commonly the output directory is \emph{ATC}. 
If the output directory is the same as the directory from the used file(s) it is supposed 
that the instance is a specific logic instance. 
\item[-x:] Excludes the specified datatype(s). That means, that there will no rules be
 generated for that datatype(s).
\item[-h:] Specifies a header file.
\end{description}
\section*{Header files}
Header files must have the suffix \verb|.header.hs| and the name of the file they 
are created for. They must be in the \emph{ATC}-directory.\\
A header file will be parsed and every instance in that file will be omited
at the rule-generation. You can also include rules for omitting any datatype(s) 
from the parsed file. The syntax for omitting a datatype is: 
\begin{verbatim}
{-| Exclude: Datatype |-}
\end{verbatim}
where "Datatype" is the name of the datatype, which should be omited. In a header file can be 
haskell-comments, they will not be considered. \\
Every logic specific ATC-instance, although it was generated from more than one file, has only one 
header file, which has the logic name (i.e., CASL.header.hs).\\
Header files are necessary, because \emph{DrIFT} can't generate all instances automatically.

\section*{Automatic file-generation with \emph{make}}
To generate a file automatically before compiling \emph{hets}, you must include the file name
in the variable \emph{genrule\_files} in the Makefile. All files in that variable will be seperated 
into files which are from a specific logic directory (CASL, HasCASL, CspCASL, Haskell) and those which 
are not. The files from every logic will be packed in the specific logic ATC-file (look above).
So every instance of a file which is not in a logic directory, will be put in the ATC-directory. 
This prospective filenames must be also added to the variable \emph{gendrifted\_files}. For example,
for the file name \emph{Commom/Named.hs} in the variable \emph{genrule\_files}, there must also be 
the file name \emph{ATC/Named.hs} in the variable \emph{gendrifted\_files}.\\
It must be noted, that the order in the variable \emph{gendrifted\_files} is significant. So if 
an instance file A imports an other instance file B, B must stand before A in the variable.

\end{document}
