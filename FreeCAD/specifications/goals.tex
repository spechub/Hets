% Ligatur-Problem (z.b. bei specific wird aus dem fi eine ligatur
% gemacht, danach nicht mehr suchbar im acrobat-reader) wird mit
% diesem Eintrag behoben. Toll!
%%%%%%%%%%%%%%%%%%%%%%%%%%
\RequirePackage{ifpdf}
\ifpdf
\RequirePackage{cmap}% Improve PDF text search.
\fi
%%%%%%%%%%%%%%%%%%%%%%%%%%
\documentclass[11pt,draft,a4paper]{article}%{scrartcl}%{report}


\usepackage{xspace}
\usepackage{amsmath}


\newcommand{\freecad}{FreeCAD\xspace}
\newcommand{\opencasc}{Open CASCADE\xspace}
\newcommand{\hets}{Hets\xspace}
\newcommand{\bash}[1]{{\tt #1}}
\newcommand{\haskell}[1]{{\tt #1}}
\newcommand{\setsep}[2]{\{#1~|~#2\}}
\newcommand{\leftsem}{[\![} %% 
\newcommand{\rightsem}{]\!]} %% 
\newcommand{\setsem}[1]{\leftsem #1 \rightsem} %% set semantics


\begin{document}


\title{Internship Goals for Robert Savu}

\author{Ewaryst Schulz}

\maketitle


\section{Reading \freecad documents}
Given a document containing a \freecad design (typically a *.fcstd-file) we want to import it into \hets. This requires the following:
\begin{description}
  \item[Abstract Syntax:] an appropriate representation in \hets for \freecad designs,
    i.e., the \freecad abstract syntax (mostly done)
  \item[Semantics:] \freecad terms should have a semantics in the sense of 3D Pointsets (see section \ref{sec:semantics})
  \item[Import:] an import method translating \freecad documents into this
    representation (partly done, has to be integrated into one method)
  \item[\hets Integration:] a complete integration of the \freecad logic into the \hets
    logic-graph (partly done, signature and simple static analysis missing)
  \item[\hets tool:] a hook into the hets program to open \freecad documents from there, i.e., \bash{hets -g test.fcstd} should work
  \item[Pretty printing] for basic \freecad specs, i.e.,
    \haskell{Pretty}-instances for the \freecad abstract syntax
\end{description}

\subsection{Semantics}
\label{sec:semantics}
We first give the semantics of some base objects such as rectangles,
boxes, and cylinders. The goal is to specify the semantics for each of
the base objects used in the abstract syntax of \freecad and also for
the transformation by rotations and translations as well as for
compound objects such as \haskell{Cut}, \haskell{Common},
\haskell{Fusion}, etc..

\begin{figure}[htb]
    \centering
    \framebox{\begin{minipage}{\textwidth}
\begin{align}
\setsem{Rectangle(w, l)} & = \text{The set consisting of the four sides of the }\notag\\
 & \qquad \text{rectangle in the x-y-plane}\notag\\
 & = \setsep{(x,y,0)}{x\in[0,l], y\in\{0,w\}}\notag\\
 & \quad \cup \setsep{(x,y,0)}{x\in\{0,l\}, y\in[0,w]}\\
\setsem{Box(h, l, w)} & = \text{The solid bounded by the faces of the box}\notag\\
 & = [0,l]\times[0,w]\times[0,h]\\
\setsem{Cylinder(a, h, r)} & = \text{The ``pac-man'' cylinder along z-axis}\notag\\
 & = circle\times[0,h]\\
 \text{where}~ circle &=\setsep{(\rho\cdot cos(\alpha),\rho\cdot sin(\alpha))}{\alpha \in [0,a], \rho \in [0,r]}\notag\\
\setsem{Sphere(a_1, a_2, a_3, r)} & = \setsep{(x,y,z)}{x + y + z \leq r} \cap \notag\\
& \setsep{(\rho\cdot cos(\alpha),\rho\cdot sin(\alpha),z)}{z \in [sin(a_1),sin(a_2)]. \rho \in [0, \infty], \alpha \in [0, a_3]}\\
\setsem{Cone(a, r_1, r_2, h)} & = \setsep{(\rho \cdot cos(\alpha), \rho \cdot sin(\alpha), z)}{\rho \leq r_1 + {{z}\over{h}} \cdot (r_2 - r_1). z \in [0,h],\alpha \in [0, a]} \\
\setsem{Torus(a_1, a_2, a_3, r_1, r_2)} & = \{(x,y,z)\} \cup \{(x_2,y_2,z_2)\}\notag\\
& \text{where:}\notag\\
& circle = \setsep{(\rho\cdot cos(\alpha), \rho\cdot sin(\alpha))}{\alpha \in [a_1, a_2], \rho \in [0,r_2]}\notag\\
& (rad, z) \in \setsep{(m,n)}{(-(m-r_1), n)\in circle}\notag\\
& x = rad\cdot cos (ang) \notag\\
& y = rad\cdot sin (ang) \notag\\
& ang \in [0, a_3] \notag\\
& \text{if } (a_1 \geq -180) \text{or} (a_2 \leq 180) \text {then} \notag\\
& (x_2,y_2,z_2 - r_2 \cdot sin(a_1)) \in \notag\\
&\setsem{Cone(a_3, r_1 - r_2 \cdot cos(a_1), r_1 - r_2 \cdot cos(a_2), sin(a_2) - sin(a_1))}\\
\setsem{Circle(sa, ea, r)} & = \setsep{(x,y,z)}{x = r\cdot cos(\alpha), y = r \cdot sin(\alpha), z = 0, \alpha \in [sa, ea]}\\
\end{align}
\end{minipage}}
    \caption{Set semantics for some baseobjects}\label{fig:semantics}
\end{figure}

\end{document}
