\documentclass{llncs}
\usepackage{graphicx}
\usepackage{mathpartir}
\usepackage{amsmath,amssymb}
\usepackage{stmaryrd}

\newcommand{\hearts}{\heartsuit}
\newcommand{\prems}{\mathit{prems}}
\newcommand{\eval}{\mathit{eval}}
\newcommand{\sem}[1]{[\![#1]\!]}
\newcommand{\ind}[1]{\llparenthesis #1 \rrparenthesis}


\newcommand{\mi}[1]{\mathit{#1}}

% defs.tex ---
% for WADT99
% copied from Tarlecki's definitions for frocos'98
% modified from moving defs.tex

\newcommand{\ASP}{\mathit{ASP}}
\newcommand{\SP}{\mathit{SP}}
\newcommand{\UDD}{\mathit{UDD}}
\newcommand{\UT}{\mathit{T}}
\newcommand{\UN}{\mathit{U}}
\newcommand{\UDecl}{\mathit{Dcl}}
\newcommand{\UDefn}{\mathit{Dfn}}

\newcommand{\parspec}[3]{#1{\stackrel{#2}{\longrightarrow}}#3}

\newcommand{\archspec}[2]{\textbf{arch spec }#1\textbf{ result }#2}
\newcommand{\decl}[2]{#1\colon#2}
\newcommand{\dfn}[2]{#1=#2}
\newcommand{\amlg}[4]{#1\textbf{ with }#2\textbf{ and }#3\textbf{ with }#4}
\newcommand{\appl}[3]{#1\lbrack#2\textbf{ fit }#3\rbrack}
\newcommand{\gris}{\mathrel{::=}}


\newcommand{\Mset}{\mathcal{M}}
\newcommand{\Fset}{\mathcal{F}}

\newcommand{\Dgm}{\mathit{Diag}}
\newcommand{\Shape}[1]{\mathit{Shape}(#1)}
\newcommand{\Nodes}[1]{\mathit{Nodes}(#1)}
\newcommand{\Edges}[1]{\mathit{Edges}(#1)}
%\newcommand{\restrict}[2]{\reduct{#1}{#2}}

\newcommand{\fm}[2]{\langle#1\rangle_{#2}}

\newcommand{\incl}[2]{\iota_{{#1}\subseteq{#2}}}

\newcommand{\st}{\mathit{st}}
\newcommand{\stC}{\mathit{C}_{\st}}
\newcommand{\stB}{\mathit{B}_{\st}}
\newcommand{\stP}{\mathit{P}_{\st}}
\newcommand{\emptystC}{\stC^\emptyset}
\newcommand{\emptyUC}{\UC^\emptyset}
\newcommand{\UC}{\mathcal{C}}
\newcommand{\UEv}{\mathit{UEv}}
\newcommand{\Uset}{\mathcal{V}}
\newcommand{\UU}{\mathit{V}}

\newcommand{\EstC}{\mathcal{C}_{\st}}
\newcommand{\emptyEstC}{\EstC^\emptyset}
\newcommand{\EstB}{\mathcal{B}_{\st}}
\newcommand{\SD}{D}

\newcommand{\CstC}{\mathfrak{C}_{\st}}

\newcommand{\getstctx}{\mathit{ctx}}

\newcommand{\env}{\mathit{E}}

\newcommand{\lambdaexp}[2]{\lambda#1\cdot#2}
\newcommand{\reduct}[2]{#1|_{#2}}

\newlength{\croutw}
\newlength{\crouth}
\newcommand{\crossout}[1]%
        {\settowidth{\croutw}{$#1$}\settoheight{\crouth}{$#1$}#1%
        \hspace{-1.0\croutw}\raisebox{0.3\crouth}{\rule{\croutw}{0.1ex}}}

\newcommand{\commentout}[1]{\ignorespaces}
\newcommand{\underscore}{\rule{3mm}{0.005in}}

\newcommand{\extaticsem}[3]{#1 \vdash #2 \extsmb #3}
\newcommand{\extsmb}{\mathrel{\rhd\hspace{-0.5em}\rhd}}

\newcommand{\rulesection}[1]{\[\fbox{$#1$}\]}
\newcommand{\infrule}[2]{\frac{#1}{#2}}
\newcommand{\staticsem}[3]{#1 \vdash #2 \rhd #3}
\newcommand{\modelsem}[3]{#1 \vdash #2 \Rightarrow #3}
\newcommand{\staticsembreak}[3]{\begin{array}{r}#1 \vdash #2 \qquad\\ 
                                                \rhd #3 \end{array}}
\newcommand{\modelsembreak}[3]{\begin{array}{r}#1 \vdash #2 \qquad\\
                                                 \Rightarrow #3 \end{array}}



% Defs from Till 


%\newenvironment{proof}{\begin{trivlist}\item[\textbf{Proof. 
%}]}{\end{trivlist}}

%\newcommand{\qed} {\hfill{$\Box$}}

\newcommand{\ttsize}{\footnotesize }
\newcommand{\NT}[1]{{\ttsize\texttt{#1}}}

%\newcommand{\gram}[1]{{\texttt{#1}}}
%\newenvironment{grammar}
% {\texonly{\footnotesize}
%  \begin{example}}{\end{example}\normalsize}
%\newenvironment{slgrammar}
% {\texonly{\footnotesize}
%  \begin{example}\texorhtml{\slshape}{\it}}{\end{example}\normalsize}


\newcommand{\cofidirectory}{}


%symbols etc
\newcommand{\parrightarrow}{\p\rightarrow}
\newcommand{\p}[1]{\mathrel{\ooalign{\hfil$\mapstochar\mkern 
5mu$\hfil\cr$#1$}}}
%\newsavebox{\pararrbox}
%\savebox{\pararrbox}{$\ \rightarrow\!\!\!\!\!\!{\raisebox{.15cm}{\tiny
%p}}\ \ $}
%\newcommand{\classicalpararrow}{{\usebox{\pararrbox}}}
\newcommand{\totrightarrow}{\rightarrow}


\newcommand{\X}{CASL}
%\newcommand{\congr}{\equiv}
%\newcommand{\TSX}{{T_{\Sigma}(X)}}
\newcommand{\LSX}{{W_{\Sigma}(X)}}

\newcommand{\Powfin}{\mathcal{P}_\omega}
\newcommand{\gen}[1]{\mid_{#1}}
\newcommand{\cogen}[1]{\mid^{#1}}
\newcommand{\induce}[2]{\mid^{#1}_{#2}}
\newcommand{\inducefrom}[1]{\!\mid_{#1}}
\newcommand{\induceto}[1]{\mid^{#1}}

%\makeindex

%\newcommand{\Mod}{\mathbf{Mod}}
\newcommand{\sen}{\mathbf{Sen}}

%\newcommand{\Sen}{\mathbf{Sen}}

%\newcommand{\Sign}{\mathbf{Sign}}
%\newcommand{\Set}{\mathbf{Set}}
%\newcommand{\ttsize}{\footnotesize }
%\newcommand{\NT}[1]{{\ttsize\texttt{#1}}}
\newcommand{\PF}{\mathit{PF}}
\newcommand{\TF}{\mathit{TF}}

\newcommand{\SY}{\mathit SY}
\newcommand{\SYs}{\mathit SYs}
\newcommand{\RSY}{\mathit RSY}
\newcommand{\RSYs}{\mathit RSYs}
\newcommand{\Sym}{\mathit{Sym}}
\newcommand{\RawSym}{\mathit{RawSym}}
\newcommand{\RawSymMap}{\mathit{RawSymMap}}
\newcommand{\IDAsRawSym}{\mathit{IDAsRawSym}}
\newcommand{\SymAsRawSym}{\mathit{SymAsRawSym}}

%\newcommand{\dom}{\mathit{dom}}
%\newcommand{\graph}{\mathit{graph}}
%% incompatible with xypic -- dirk


%\newenvironment{grammar}
% {\ttsize\begin{alltt}}{\end{alltt}}



\newcommand{\TSX}{{T_{\Sigma^\#}(X)}}

\newcommand{\bit}{\begin{itemize}}
\newcommand{\eit}{\end{itemize}}
\newcommand{\Equivalent}{\Leftrightarrow}
%\newcommand{\Implies}{\Rightarrow}
\newcommand{\logand}{\wedge}
\newcommand{\logor}{\vee}
%\newcommand{\union}{\cup}
\newcommand{\intersection}{\cap}
\newcommand{\comp}{\circ}
\newcommand{\suchthat}{\mid}
\newcommand{\congr}{\equiv}
\newcommand{\disjoint}{\uplus}
\newcommand{\buildset}[1]{\{#1\}}
\newcommand{\Si}{\Sigma}
\newcommand{\al}{\alpha}
\newcommand{\pp}[2]{#1_1\commadots #1_{#2}}
%\newcommand{\map}[2]{\colon#1\!\longrightarrow\!#2}
\newcommand{\mmap}[2]{#1\!\longrightarrow\!#2}
\newcommand{\anddots}{\logand\,\cdots\,\logand}
\newcommand{\lldots}{\,\ldots\,}
\newcommand{\ccdots}{\,\cdots\,}
\newcommand{\commadots}{,\lldots,}
\newcommand{\sigspec}[1]{\langle #1 \rangle}
\newcommand{\eq}{\stackrel{\mbox{\scriptsize e}}{=}}
\newcommand{\weq}{\stackrel{\mbox{\scriptsize w}}{=}}

\newcommand{\mBox}[1]{\, \mbox{#1} \,}

\newcommand{\engtwocase}[3]{
\left\{
\begin{array}{ll}
  #1,&\mBox{if }#2\\
  #3,&\mBox{otherwise}
\end{array}\right.}

\newcommand{\threecase}[5]{
\left\{
\begin{array}{ll}
  #1,&\mbox{if }#2\\
  #3,&\mbox{if }#4\\
  #5,&\mbox{otherwise}
\end{array}\right.}

\newcommand{\threefullcase}[6]{
\left\{
\begin{array}{ll}
  #1,&\mbox{if }#2\\
  #3,&\mbox{if }#4\\
  #5,&\mbox{if }#6
\end{array}\right.}

\newcommand{\delete}[1]{$\overline{\mbox{#1}}$}
\newcommand{\new}[1]{{\it\tiny #1}}

\newcommand{\mbs}[1]{\mbox{$#1$}}




\newcommand{\ASL}[1]{[\![#1]\!]}

\newcommand{\sym}{\mathit{sym}}
\newcommand{\totqual}{\mathsf{t}}
\newcommand{\parqual}{\mathsf{p}}
\newcommand{\ws}{\mathit{ws}}
\newcommand{\FinSet}[1]{\mathit{FinSet}(#1)}
\newcommand{\RawSymOf}{\mathit{RawSymOf}}
\newcommand{\Gram}[1]{{\textup{\texttt{#1}}}}

\newcommand{\implicitqualkind}{\mathit{implicit}}
\newcommand{\sortqualkind}{\mathit{sort}}
\newcommand{\funqualkind}{\mathit{fun}}
\newcommand{\predqualkind}{\mathit{pred}}
\newcommand{\SymKind}{\mathit{SymKind}}

%\newcommand{\Sort}{\mathit{Sort}}
\newcommand{\QualFunName}{\mathit{QualFunName}}
\newcommand{\QualPredName}{\mathit{QualPredName}}



% Lutz's definitions


\hfuzz1pc
\sloppy
%\frenchspacing


%\spnewtheorem{thm}{Theorem}[section]{\bfseries}{\itshape}
%\spnewtheorem{cor}[theorem]{Corollary}{\bfseries}{\itshape}
%\spnewtheorem{lem}[theorem]{Lemma}{\bfseries}{\itshape}
%\spnewtheorem{prop}[theorem]{Proposition}{\bfseries}{\itshape}
%\spnewtheorem{defn}[theorem]{Definition}{\bfseries}{\upshape}
%\spnewtheorem{rem}[theorem]{Remark}{\bfseries}{\upshape}
%\spnewtheorem{expl}[theorem]{Example}{\bfseries}{\upshape}
%\spnewtheorem{thmdefn}[theorem]{Theorem and Definition}{\bfseries}{\itshape}
%\spnewtheorem{propdefn}[theorem]{Proposition and Definition}{\bfseries}{\itshape}
%\spnewtheorem{assumption}[theorem]{Assumption}{\bfseries}{\upshape}
%\spnewtheorem{algorithm}[theorem]{Algorithm}{\bfseries}{\upshape}

%% Other environments
%\newenvironment{mycd}
%       {\begin{displaymath}\renewcommand{\arraystretch}{0.5}\begin{CD}}
%       {\end{CD}\end{displaymath}}
\newenvironment{rmenumerate}%
        {\begin{enumerate}\renewcommand{\labelenumi}{(\roman{enumi})}}%
        {\end{enumerate}}
\newenvironment{alenumerate}%
        {\begin{enumerate}\renewcommand{\labelenumi}{(\alph{enumi})}}%
        {\end{enumerate}}
\newenvironment{Alenumerate}%
        {\begin{enumerate}\renewcommand{\labelenumi}{\Alph{enumi})}}%
        {\end{enumerate}}
\newenvironment{numenumerate}%
        {\begin{enumerate}\renewcommand{\labelenumi}{\arabic{enumi})}}%
        {\end{enumerate}}
%\newenvironment{defn}%
%       {\begin{definition}\em}%
%       {\end{definition}}
%\newenvironment{rem}%
%       {\begin{remark}\em}%
%       {\end{remark}}
%\renewenvironment{example}%
%       {\begin{expl}\em}%
%       {\end{expl}}
%\newenvironment{texteqnarray}%
%       {\setlength{\itemsep}{0pt}%
%       \setlength{\topsep}{0pt}%
%       \begin{trivlist}\item[]\begin{displaymath}%
%       \renewcommand{\arraystretch}{1.2}\begin{array}{rcll}}%
%       {\end{array}\end{displaymath}\end{trivlist}}
%\newenvironment{acknowledge}%
%       {\begin{list}{}{\setlength{\leftmargin}{0pt}%
%                       \setlength{\topsep}{1cm}}% 
%       \item{\bf Acknowledgements:}}%
%       {\end{list}}

%\newcommand{\qed}{\hfill\openbox}
%\newenvironment{proof}
%       {\begin{trivlist}\item{\sc Proof: }}
%       {\qed\end{trivlist}}



%% Math definitions
\newcommand{\Cat}{\mathbf}
\newcommand{\Cls}{\mathcal}
\newcommand{\Opname}{\mathrm}

\newcommand{\Ob}{{\Opname{Ob}\,}}
\newcommand{\Mor}{{\Opname{Mor}\,}}
\newcommand{\Epi}{{\Opname{Epi}\,}}
\newcommand{\Mono}{{\Opname{Mono}\,}}
\newcommand{\Iso}{{\Opname{Iso}\,}}
\newcommand{\Ident}{{\Opname{Ident}\,}}
\newcommand{\Op}{{op}}
\newcommand{\colim}{{\Opname{colim}\,}}

\newcommand{\lrule}[3]{(#1)\;\;\infrule{#2}{#3}}
\newcommand{\lrrule}[3]{(#1)\;\;\infrule{\underline{#2}}{#3}}
\newcommand{\laxiom}[2]{(#1)\;\;#2}

\newcommand{\BA}{{\Cat A}}
%\newcommand{\BB}{{\Cat B}}
\newcommand{\BC}{{\Cat C}}
\newcommand{\BD}{{\Cat D}}
\newcommand{\BI}{{\Cat I}}
\newcommand{\BK}{{\Cat K}}
\newcommand{\BL}{{\Cat L}}
\newcommand{\BM}{{\Cat M}}
\newcommand{\BN}{{\Cat N}}
\newcommand{\BS}{{\Cat S}}
\newcommand{\BW}{{\Cat W}}
\newcommand{\Btwo}{{\Cat 2}}

\newcommand{\CC}{{\Cls C}}
\newcommand{\CE}{{\Cls E}}
%\newcommand{\CK}{{\Cls K}}
% \newcommand{\CL}{{\Cls L}}
\newcommand{\CM}{{\Cls M}}
\newcommand{\CO}{{\Cls O}}
\newcommand{\CP}{{\Cls P}}
\newcommand{\CS}{{\Cls S}}
\newcommand{\CT}{{\Cls T}}
\newcommand{\eps}{\varepsilon}

\newcommand{\integers}{{\mathbb Z}}
\newcommand{\tensor}{\otimes}

%\newcommand{\openbox}{\leavevmode
%  \hbox to.77778em{%
%  \hfil\vrule
%  \vbox to.675em{\hrule width.6em\vfil\hrule}%
%  \vrule\hfil}}
\newcommand{\mystrut}[1]{\rule[#1]{0cm}{0.1cm}}



\newcommand{\red}{\vdash}
\newcommand{\idred}{\red}
\newcommand{\step}{\red_c}
% \newcommand{\stepeq}{\succeq}
\newcommand{\downto}{\succeq}
\newcommand{\upto}{\preceq}
\newcommand{\moremonicthan}{\Rightarrow}
\newcommand{\into}{\hookrightarrow}
\newcommand{\impl}{\Rightarrow}
\newcommand{\id}{{id}}
\newcommand{\ev}{{ev}}
\newcommand{\Hom}{{hom}}
\newcommand{\Dom}{{dom\, }}
\newcommand{\Cod}{{cod}}
\newcommand{\adj}{\dashv}
\newcommand{\powerset}{{\mathcal P}}
\newcommand{\finpowerset}{{\powerset_\omega}}
\newcommand{\kVec}{\mbox{$k$-$\Cat{Vec}$}}
\newcommand{\termObj}{1}

\newcommand{\map}[2]{:#1\to #2}
\newcommand{\restr}[2]{{#1}|_{#2}}
\newcommand{\wordbrace}[1]{\underbrace{\hspace{1.5cm}}_{\displaystyle{#1}}}
\newcommand{\forget}[1]{|_{#1}}
\newcommand{\argument}{\_\!\_}%{\underline{\;\;}}
%\newcommand{\substack}[2]{{{#1} \atop {#2}}}

%% Category Names
\newcommand{\CASLsign}{\Cat{CASLsign}}
\newcommand{\enrCASLsign}{\Cat{enrCASLsign}}
\newcommand{\refCASLsign}{\Cat{refCASLsign}}
%\newcommand{\Mod}{\Cat{Mod}\,}
% the \, caused bad effect when copmbined with indices etc.
\newcommand{\EnrMod}{\Cat{Mod}_e}
\newcommand{\CAT}{\Cat{CAT}}
\newcommand{\Sen}{\Cat{Sen}}
\newcommand{\Sign}{\Cat{Sign}}
\newcommand{\SignVar}{\Cat{Var}}
\newcommand{\EnrSign}{\Cat{EnrSign}}
%\newcommand{\Set}{\Cat{Set}}
\newcommand{\CLS}{\Cat{CLS}}

\newcommand{\SmallSpecName}[1]{\textsc{\small #1}}
%% Institution Names
\newcommand{\EnrInst}{EnrPCFOL}
\newcommand{\CASLInst}{SubPCFOL}
\newcommand{\PCFOLInst}{PCFOL}

%% Words
\newcommand{\Word}{\mathbf}
\newcommand{\Ba}{{\Word{a}}}
\newcommand{\Bb}{{\Word{b}}}
\newcommand{\Bc}{{\Word{c}}}
\newcommand{\Bf}{{\Word{f}}}
\newcommand{\Bg}{{\Word{g}}}
\newcommand{\Bh}{{\Word{h}}}
\newcommand{\Bl}{{\Word{l}}}
\newcommand{\Br}{{\Word{r}}}



%% For HasCASL:

\newcommand{\HasCASL}{{\sc HasCasl}\xspace}
\newcommand{\ML}{{\textsf{ML}}\xspace}
\newcommand{\BaseTypes}{B}
\newcommand{\SimpleTypes}{T}
\newcommand{\TotalFunType}[2]{#1\to #2}
\newcommand{\sbullet}{\mbox{ $\scriptstyle\bullet$ }}
\newcommand{\lambdaTotal}[2]{\lambda\, #1\sbullet\!\! !\, #2}
\newcommand{\PartialFunType}[2]{#1\to ?#2}
\newcommand{\lambdaPartial}[2]{\lambda\, #1\sbullet #2}
\newcommand{\lambdaType}[2]{\lambda\, #1\sbullet #2}
\newcommand{\NonStrictFunType}[2]{\PartialFunType{\Maybe{#1}}{#2}}
\newcommand{\lambdaNonStrict}[2]{\lambda\, #1\sbullet #2}
\newcommand{\substTerm}[3]{{#1[#3/#2]}}
\newcommand{\PredType}[1]{{\textsf{pred}(#1)}}
\newcommand{\ProdType}[2]{#1 * #2}
\newcommand{\pMonType}[2]{#1 -\!\!\mu\!\!\to\hspace{-1pt}? #2}
\newcommand{\tMonType}[2]{#1 -\!\!\mu\!\!\to #2}
\newcommand{\pContType}{Pcont}
\newcommand{\tContType}{Tcont}
\newcommand{\projOp}[1]{{pr_#1}}
\newcommand{\fst}{{\textsf{fst}}}
\newcommand{\snd}{{\textsf{snd}}}
\newcommand{\UnitType}{{\textsf{unit}}}
\newcommand{\UnitOp}{{()}}
\newcommand{\Maybe}[1]{?#1}
\newcommand{\TypeSchemes}{{TS}}
\newcommand{\TypeContext}{\mathcal{C}}
\newcommand{\UniversalType}[2]{{\forall #1\sbullet #2}}
\newcommand{\applyPoly}[2]{{#1[#2]}}
%\newcommand{\lambdaPoly}[2]{\Lambda #1:\TypeKind.\, #2}
\newcommand{\ConstrainedType}[2]{{#1 \Rightarrow #2}}
\newcommand{\Constraint}{C}
\newcommand{\TypeKind}{\textsf{type}}
\newcommand{\TypeVars}{{TV}}
\newcommand{\TypeOps}{{TO}}
\newcommand{\PseudoTypes}{{PT}}
\newcommand{\SubstType}[3]{{#1[#3/#2]}}
\newcommand{\varco}[1]{{v^+(#1)}}
\newcommand{\varcontra}[1]{{v^-(#1)}}
\newcommand{\Covar}{\textsf{covariant}}
\newcommand{\Contravar}{\textsf{contravariant}}
\newcommand{\sigSorts}{S}
\newcommand{\sigClasses}{C}
\newcommand{\sigConstrs}{T}
\newcommand{\sigAlias}{{A}}
\newcommand{\sigExpand}{{\varepsilon}}
\newcommand{\unalias}{{\bar\sigExpand}}
\newcommand{\sigSub}{\leq}
\newcommand{\extSub}{\leq_*}
\newcommand{\elPred}[2]{{#1\in #2}}
\newcommand{\downCast}[2]{{#1\textsf{ as }#2}}
\newcommand{\upCast}[2]{{#1:#2}}
\newcommand{\sigOps}{O}
\newcommand{\IndVars}{V}
\newcommand{\ifthenelse}[3]{{\textsf{if }#1\textsf{ then }#2
                        \textsf{  else }#3}}
\newcommand{\BT}{B}
\newcommand{\emb}{{emb}}

\newcommand{\Context}{\Gamma}
\newcommand{\ContextTerm}[3]{{#1 \rhd #2: #3}}
\newcommand{\ContextEq}[2]{{#1 \rhd #2}}
\newcommand{\ContextEntails}[3]{{#2 \entails_{#1} #3}}
\newcommand{\ContextImpl}[3]{{#2 \impl_{#1} #3}}
\newcommand{\exeq}{\stackrel{e}{=}}
\newcommand{\IsDef}{\operatorname{def}}

\newcommand{\isFormula}[1]{{#1\in\textsf{Prop}}}

\newcommand{\Model}{M}
\newcommand{\Sem}[1]{{[\![#1]\!]}}
\newcommand{\extendsTo}{\preceq}
\newcommand{\True}{\top}
\newcommand{\False}{\bot}
\newcommand{\uexists}{\exists !}
\newcommand{\iotaTerm}[2]{\iota #1\sbullet #2}
%\newcommand{\conj}{\wedge}
\newcommand{\disj}{\vee}
\newcommand{\modimpl}{\to}
\newcommand{\modiff}{\leftrightarrow}
\newcommand{\intEq}{eq}
\newcommand{\dOrder}[3]{#2 \le\![#1]\,#3}
\newcommand{\resTerm}[2]{#1\ res\ #2}
\newcommand{\mono}[2]{\operatorname{mono}_{#1}#2}
%\newcommand{\entails}{\vdash}

%% CASL constructs

%% HasCASL constructs
\newcommand{\Class}{\textbf{class}}
\newcommand{\Instance}{\textbf{instance}}
\newcommand{\Program}{\textbf{program}}
\newcommand{\Internal}{\textbf{internal}}
\providecommand{\pfun}{\mathrel{\rightarrow?}}
\newcommand{\iimpl}{\Rightarrow}
\newcommand{\iconj}{\wedge}

%% For Monads:
\newcommand{\BBT}{\mathbb{T}}
\newcommand{\DO}{\operatorname{do}}
\newcommand{\Let}{\operatorname{let}}
\newcommand{\retOp}{\operatorname{ret}}
\newcommand{\retOpHC}{\mathit{ret}}
\newcommand{\ret}[1]{\retOp #1}
\newcommand{\letTerm}[2]{\DO\, #1; \ #2}
\newcommand{\leteq}{\leftarrow}
\newcommand{\se}[2]{\mathsf{se}(#1,#2)}
\newcommand{\Pfin}{\mathcal P_{\mathit{fin}}}

%% Program logic
\newcommand{\nec}{\square\,}
\newcommand{\gbox}{\square\hspace{-6.5pt}\raisebox{2pt}{\tiny{G}}\;}
\newcommand{\pbox}[1]{[#1]\,}
\newcommand{\pdiamond}[1]{<\hspace{-3pt}#1\hspace{-3pt}>\!}
%\newcommand{\lbox}{\square\,}
\newcommand{\lbox}{\Box}
\newcommand{\ldiamond}{\Diamond}
\newcommand{\pmodal}[2]{[#1]_G\,#2}
\newcommand{\Timpl}{\Rightarrow_T}
\newcommand{\Tconj}{\wedge}
\newcommand{\HTriple}[3]{\{#1\}~#2~\{#3\}}
\newcommand{\ContextHTriple}[4]{#1\rhd\HTriple{#2}{#3}{#4}}
\newcommand{\DummyCHT}[4]{\HTriple{#2}{#3}{#4}}
\newcommand{\sef}{\mathit{sef}}
\newcommand{\dsef}{\mathit{dsef}}

%% For polymorphism
\newcommand{\extInst}[1]{\mathrm{Ext}(#1)}
%\newcommand{\gextInst}[1]{\mathrm{gExt}(#1)}
\newcommand{\Poly}[1]{\mathrm{Poly}(#1)}
%% For institutions


\newcommand{\ModFunctor}{\Cat{Mod}\,}


\newcommand{\la}{\langle}
\newcommand{\ra}{\rangle}


\newcommand{\si}{\sigma}

\newcommand{\ISig}{\mathit{Sig}}
\newcommand{\IMod}{\mathit{Mod}}
\newcommand{\IAx}{\mathit{Ax}}

\newenvironment{specsem}%
        {\smallskip\par\noindent\qquad$\begin{array}{@{}l@{{}={}}l}}%
        {\end{array}$\par\noindent}


%% Width of parboxes for figures
\newlength{\myboxwidth}
\setlength{\myboxwidth}{\textwidth}
\addtolength{\myboxwidth}{-20pt}
\newenvironment{myfigure}{\begin{figure}\begin{center}
	\setlength{\fboxsep}{10pt}}% 
	{\end{center}\end{figure}}

%% Coalgebraic Modal Logic
\newcommand{\Lang}{\mathcal{L}}	
\newcommand{\FLang}{\mathcal{F}}	
\newcommand{\pls}{\Lambda}
\newcommand{\pl}[3]{#1\in #2(#3)}
\newcommand{\polypl}[3]{#1\in #2 #3}
\newcommand{\negpl}[3]{#1\notin #2(#3)}
\newcommand{\plbox}[1]{[#1]}
%\newcommand{\pldiamond}[1]{<\hspace{-3pt}#1\hspace{-3pt}>\!}
\newcommand{\pldiamond}[1]{\langle#1\rangle}
\newcommand{\gldiamond}[1]{\Diamond_{#1}}
\newcommand{\glbox}[1]{\square_{#1}}

\newcommand{\HMLBox}[1]{\square\hspace{-5.5pt}\raisebox{2pt}{$\scriptscriptstyle{#1}$}\;}
\newcommand{\HMLDiamondb}[1]{\Diamond\hspace{-5pt}\raisebox{1.5pt}{$\scriptscriptstyle{#1}$}\;}
\newcommand{\HMLDiamonda}[1]{\Diamond\hspace{-5.3pt}\raisebox{1.9pt}{$\scriptscriptstyle{#1}$}\;}
\newcommand{\pldbox}[2]{[#1]_{#2}}
\newcommand{\plempty}{\plbox{\emptyset}}
\newcommand{\plcan}{\square}
\newcommand{\FC}{{\mathfrak C}}
\newcommand{\negcl}[1]{\mathit{cl}(#1)}
\newcommand{\Nat}{{\mathbb{N}}}
\newcommand{\Int}{{\mathbb{Z}}}
\newcommand{\Rat}{{\mathbb{Q}}}
\newcommand{\Real}{{\mathbb{R}}}
\newcommand{\List}{\mathsf{list}}
\newcommand{\PDist}{D_\omega}%% Provisionary!!
%\newcommand{\Id}{\mathit{Id}}
\newcommand{\SFun}{\mathcal{SP}}
\newcommand{\plcomp}{\circledast}
\newcommand{\contrapower}{2^{\argument}}
\newcommand{\trans}[1]{#1^\flat}
\newcommand{\boolcl}[2]{\mathrm{bcl}_{#1}(#2)}
\newcommand{\Bag}{\mathcal{B}}
\newcommand{\Baginfty}{\mathcal{B}_\infty}
%\newcommand{\Ab}{\mathcal{B}_{\mathbb{Z}}}
\newcommand{\Prop}{\mathsf{Prop}}
\newcommand{\Up}{\mathsf{Up}}
\newcommand{\Lit}{\mathsf{Lit}}
\newcommand{\nneg}{\sim}
\newcommand{\satisfies}{\vDash}
\newcommand{\gsatisfies}{\satisfies_g}
\newcommand{\nsatisfies}{\nvDash}
\newcommand{\PSPACE}{\mathit{PSPACE}}
\newcommand{\Rules}{\mathcal{R}}
\newcommand{\RulesC}{\Rules_C}
\newcommand{\Frame}[1]{2^{2^{#1}}}
\newcommand{\UpP}{\mathsf{Up}\mathcal{P}}
\newcommand{\one}{\mathbb{1}}
%\newcommand{\smaller}[2]{#1_{<#2}}
\newcommand{\total}{\nu}
\newcommand{\sgn}{\mathit{sgn}}
%\newcommand{\RedClause}{\Clause^r}
\newcommand{\size}{\mathit{size}}
\newcommand{\arity}{\mathit{ar}}

\newcommand{\Sorts}{\mathcal{S}}
\newcommand{\MSorts}{\Sorts_0}
\newcommand{\ModOp}{L}
\newcommand{\contrapow}{\mathcal{Q}}
\newcommand{\profto}{\stackrel{\bullet}{\to}}
\newcommand{\Struct}{\mathcal{M}}
\newcommand{\Id}{\mathit{Id}}
\newcommand{\Det}{\mathsf{Det}}
\newcommand{\hDet}{\mathsf{hDet}}
\newcommand{\FF}{\mathfrak{F}}
\newcommand{\idOp}{\plbox{\iota}}
\newcommand{\Sig}{\mathit{Sig}}

\newcommand{\NEXP}{\mi{NEXPTIME}}
\newcommand{\EXP}{\mi{EXPTIME}}
\newcommand{\NP}{\mi{NP}}
\newcommand{\at}{\mi{at}}
\newcommand{\Max}{\mi{max}}
\newcommand{\muprod}{\textstyle\prod^\mu}
\newcommand{\tsum}{\textstyle\sum}
\DeclareMathOperator{\mge}{\ge}
\DeclareMathOperator{\meq}{=}
\DeclareMathOperator{\mle}{\le}
\DeclareMathOperator{\fixarrow}{\uparrow}
\newcommand{\Stable}{S}
\newcommand{\Zero}{Z}
\newcommand{\Regular}{U}
\newcommand{\last}{\lambda}
\newcommand{\fix}[2]{#1\!\fixarrow #2}
\newcommand{\msf}{\mathsf}

\newcommand{\CKCMi}{\CK\!+\!\CMi\xspace}
\newcommand{\CK}{\mathit{CK}}
\newcommand{\CMi}{\mathit{CMi}}
\newcommand{\Ax}{\mathcal{A}}

%\newcommand{\PLentails}{\entails_{\mi{PL}}}
%\newcommand{\ModSig}{\Lambda}

\newcommand{\modelsCA}{\models}
\newcommand{\Th}{\mathsf{Th}}
\newcommand{\rank}{\operatorname{rank}}
\newcommand{\Space}{\mathcal{S}}
\newcommand{\osder}[2]{\entails_{#1}^{#2}}
\newcommand{\gentails}[1]{\entails^{#1}_g}

\sloppy
% The following is enclosed to allow easy detection of differences in
% ascii coding.
% Upper-case    A B C D E F G H I J K L M N O P Q R S T U V W X Y Z
% Lower-case    a b c d e f g h i j k l m n o p q r s t u v w x y z
% Digits        0 1 2 3 4 5 6 7 8 9
% Exclamation   !           Double quote "          Hash (number) #
% Dollar        $           Percent      %          Ampersand     &
% Acute accent  '           Left paren   (          Right paren   )
% Asterisk      *           Plus         +          Comma         ,
% Minus         -           Point        .          Solidus       /
% Colon         :           Semicolon    ;          Less than     <
% Equals        =3D           Greater than >          Question mark ?
% At            @           Left bracket [          Backslash     \
% Right bracket ]           Circumflex   ^          Underscore    _
% Grave accent  `           Left brace   {          Vertical bar  |
% Right brace   }           Tilde        ~

% A couple of exemplary definitions:

%\newcommand{\Nat}{{\mathbb N}}
%\newcommand{\Real}{{\mathbb R}}
\newcommand{\COLOSS}{{\textrm CoLoSS}}
\def\lastname{Hausmann}
\begin{document}

\begin{frontmatter}
  \title{Optimizing Coalgebraic Modal Logic Reasoning}
  \author{Daniel Hausmann\thanks{Work forms part of DFG-project \emph{Generic Algorithms and Complexity
      Bounds in Coalgebraic Modal Logic} (SCHR 1118/5-1)}}
  \institute{Department of Computer Science, Universit\"at Bremen\\\email{hausmann@informatik.uni-bremen.de}}
\maketitle
\end{frontmatter}

\begin{abstract} 
The framework provided by coalgebraic modal logics offers
broadly applicable coalgebraic semantics and an ensuing general
treatment of modal sequent and tableau calculi while covering a wide
variety of logics ranging from graded and probabilistic modal logic
to coalition logic and conditional logics. Here we discuss generic optimisation
strategies that may be employed to improve the performance of global
caching algorithms that decide the satisfiability of coalgebraic modal logics.
Specifically, we discuss and show the admissability of generalisations of such
established strategies as propositional and modal \emph{simplification},
\emph{dependency directed backtracking}, \emph{semantic branching} and
complex \emph{heuristics} for coalgebraic modal logics. As a more advanced consideration, the flattened
representation of the involved proof graph by a \emph{proof formula} is shown to be
sound and complete; this separation of \emph{proof structure} from the actual content of
the proof does not only enhance the performance of the propagation
process, it also allows for further optimisation techniques such as
\emph{proof graph simplification} to be applied. The relevance of the
theoretic results is exhibited by detailled benchmarking of the examplary
implementation, the \emph{Coalgebraic Logic Satisfiability Solver}
(\emph{CoLoSS}), against established testsets of formulas.
\end{abstract}


\section{Introduction}\label{intro}
 
\ldots

\section{Coalgebraic Sequent Calculus}

\subsection{Notation}

Given an endofunctor on sets $T$, a $T$-coalgebra $\mathcal{C}$ consists of a
carrier-set $C$ and a function $\gamma:C\rightarrow T(C)$.

The semantics of modal operators $\hearts\in\Lambda$ is then defined by means
of \emph{predicate liftings} $\sem{\hearts}:2^n\rightarrow2\circ T^{op}$ where
$2:\mathbf{Set}\rightarrow\mathbf{Set}^{op}$ denotes the contravariant powerset functor.

Formulas are of the shape
\begin{quote}
$\mathcal{F}(\Lambda)\ni A_1,\ldots,A_n\mathrel{\mathop:}=p\mid \neg A_1 \mid A_1\wedge A_2\mid A_1\vee A_2 \mid \hearts(A_1,\ldots,A_n)$
\end{quote}
where $\hearts\in\Lambda$ is an $n$-ary modal operator.

The semantics with respect to a model coalgebra $M=(C,\gamma,\pi)$ (where $\pi$ denotes a
valuation of propositional atoms) is fixed as follows:
\begin{quote}
$\sem{\hearts(A_1,\ldots,A_n)}_M=\gamma^{-1}\circ\sem{\hearts}_C(\sem{A_1}_M,\ldots,\sem{A_n}_M)$
\end{quote}

A ($\Lambda$-)\emph{sequent} $\Gamma$ is a set of formulas (from $\mathcal{F}(\Lambda)$).

A \emph{sequent rule} consists of its \emph{premise} $\Sigma$ (a set of sequents)
and its \emph{conclusion} $\Gamma$ (one sequent). A rule $R=(\Sigma=\{\Gamma_1,\ldots,\Gamma_n\},\Gamma)$ 
is usually presented as follows:
\begin{quote}
\centerline{\inferrule{\Gamma_1 \\ \ldots \\ \Gamma_n}{\Gamma}}
\end{quote}

A rule application of rule $R$ to sequent $\Gamma$ is an instantiation of $R$
s.t. the conclusion of the rule is equal to $\Gamma$.

A formula $\phi$ is provable if there is a model coalgebra $M=(C,\gamma,\pi)$ with a
state $x\in C$ s.t. $\sem{\phi}\ni x$. A sequent $\Gamma=\{\phi_1,\ldots,\phi_n\}$ is provable
iff $\bigvee_{i=1}^n\phi_i$ is provable.
A premise is said to be provable iff all its sequents are provable.

\subsection{A generic sequent calculus for coalgebraic modal logics}

A sound and complete sequent calculus (w.r.t. the according coalgebraic logic) is obtained if
the propositional rules from Figures~\ref{fig:bpropRules} and~\ref{fig:lpropRules} are utilized together
with the appropriate modal rule(s). Some examplary modal rules for different coalgebraic modal logics are
depicted in Figure~\ref{fig:modalRules} while the modal rules of some conditional logics are
introduced in Figure~\ref{fig:modalRules2}.

\begin{footnotesize}
\begin{figure}[!h]
  \begin{center}
\begin{tabular}{| l |}
\hline
(T)\inferrule{ }{\Gamma, \top}\qquad
(At)\inferrule{ }{\Gamma, p, \neg p}\qquad
($\neg\vee$)\inferrule{\Gamma,A \\ \Gamma,B}{\Gamma, \neg (\neg A\vee \neg B)} \\
\hline
 \end{tabular}
  \end{center}
  \caption{Branching or finishing propositional sequent rules}
  \label{fig:bpropRules}
\end{figure}
\end{footnotesize}

\begin{footnotesize}
\begin{figure}[!h]
  \begin{center}
\begin{tabular}{| l |}
\hline
($\neg\neg$)\inferrule{\Gamma, A}{\Gamma, \neg\neg A}\qquad
($\vee$)\inferrule{\Gamma,A,B}{\Gamma, (A\vee B)} \\
\hline
 \end{tabular}
  \end{center}
  \caption{Linear propositional sequent rules}
  \label{fig:lpropRules}
\end{figure}
\end{footnotesize}

Take note that the rules from Figure~\ref{fig:lpropRules} may be integrated into
the normalisation process (see 3.1) due to their linearity.

\begin{footnotesize}
\begin{figure}[!h]
  \begin{center}
\begin{tabular}{| l | l |}
\hline
Modal Logic (Feature) & Modal Rule(s)\\
\hline
\textbf{M} ($\Box_\mathbf{M}$) & \inferrule{A\rightarrow B}{\Gamma,\Box_\mathbf{M} A\rightarrow\Box_\mathbf{M} B}  \\
\hline
\textbf{K} ($\Box_\mathbf{K}$) & \inferrule{\bigwedge\limits_{i=1}^n A_i\rightarrow B}
                                           {\Gamma,\bigwedge\limits_{i=1}^n\Box_\mathbf{K} A_i\rightarrow\Box_\mathbf{K} B}  \\
\hline
\textbf{KD} ($\Box_\mathbf{KD}$) & \inferrule{\bigwedge\limits_{i=1}^n A_i\rightarrow B}
                                           {\Gamma,\bigwedge\limits_{i=1}^n\Box_\mathbf{KD} A_i\rightarrow\Box_\mathbf{KD} B} \qquad
                                   \inferrule{\neg\bigwedge\limits_{i=1}^n A_i}
                                           {\Gamma,\neg\bigwedge\limits_{i=1}^n\Box_\mathbf{KD} A_i}  \\
\hline
Hennessy-Milner-Logic ($\Box^a_\mathbf{HM}$) & \inferrule{\bigwedge\limits_{i=1}^n A_i\rightarrow B}
                                                         {\Gamma,\bigwedge\limits_{i=1}^n \Box^a_\mathbf{HM} A_i\rightarrow\Box^a_\mathbf{HM} B}  \\
\hline
Coalition Logic ($\Box^C_\mathbf{C}$) & \inferrule{\bigwedge\limits_{i=1}^n A_i\rightarrow B\vee\bigvee\limits_{j=1}^m C_j}
                                                  {\Gamma,\bigwedge\limits_{i=1}^n\Box^{C_i}_\mathbf{C} A_i\rightarrow
                                                   \Box^D_\mathbf{C} B\vee\bigvee\limits_{j=1}^m \Box^N_\mathbf{C} C_j}  \\
                                             & $m,n\geq 0$,\\ & $C_i$ pairwise disjoint subsets of $D$.\\
\hline
Graded Modal Logic ($\Diamond^i_\mathbf{G}$) & \inferrule{\sum\limits_{i=1}^n A_i\leq\sum\limits_{j=1}^m B_j}
                                                   {\Gamma,\bigwedge\limits_{i=1}^n\Diamond^{k_i}_\mathbf{G}
                                                    A_i\rightarrow\bigvee\limits_{j=1}^m\Diamond^{l_j}_\mathbf{G} B_j}  \\
                                           & $n,m\geq 0$,\\
                                           & $\sum\nolimits_{i=1}^n(k_i + 1)\geq 1 + \sum\nolimits_{j=1}^m l_j$.\\
\hline
Probabilistic Modal Logic ($\Diamond^p_\mathbf{P}$) & \inferrule{\sum\limits_{i=1}^n A_i + u \leq\sum\limits_{j=1}^m B_j}
                                                   {\Gamma,\bigwedge\limits_{i=1}^n\Diamond^{p_i}_\mathbf{P}
                                                    A_i\rightarrow\bigvee\limits_{j=1}^m\Diamond^{q_j}_\mathbf{P} B_j}  \\
                                           & $m,n\geq 0$, $m+n\geq 1$, $u\in \mathbb{Z}$,\\
                                           & $\sum\nolimits_{i=1}^n p_i + u \geq \sum\nolimits_{j=1}^m q_j$ and\\
                                           &  $m=0\rightarrow \sum\nolimits_{i=1}^n p_i + u > 0$.\\
\hline
 \end{tabular}
  \end{center}
  \caption{One-step complete, resolution closed modal rules for some coalgebraic modal logics}
  \label{fig:modalRules}
\end{figure}
\end{footnotesize}

We introduce the following abbreviations in order to allow for a swift presentation of the modal rules
of propositional and graded modal logic:
Given a formula $\phi_i$ and $r_i\in\mathbb{Z}$ for all $i\in I$ as well as $k\in\mathbb{Z}$,
\begin{quote}
$\sum\limits_{i\in I}r_i\phi_i\geq k\equiv \bigwedge\limits_{J\subseteq I, r(J)<k}(
    \bigwedge\limits_{j\in J}\phi_j\rightarrow\bigvee\limits_{j\notin J}\phi_j)$,
\end{quote}
where $r(J)=\sum\nolimits_{j\in J}r_j$. Furthermore, we use $\sum a_i\leq\sum b_j$ as an abbreviation
for $\sum b_j - \sum a_i \geq 0$.

\begin{footnotesize}
\begin{figure}[!h]
  \begin{center}
\begin{tabular}{| l | l |}
\hline
Modal Logic (Feature) & Modal Rule(s)\\
\hline
\textbf{CK} ($\Rightarrow_\mathbf{CK}$) & \inferrule{ A_0\leftrightarrow\ldots \leftrightarrow A_n \\ \neg B_1,\ldots, \neg B_n,B_0 }
                      { \Gamma,\bigwedge\limits_{i=1}^n (A_i\Rightarrow_\mathbf{CK} B_i)\rightarrow (A_0\Rightarrow_\mathbf{CK} B_0)}  \\
\hline
\textbf{CK+CEM} ($\Rightarrow_\mathbf{CK_{CEM}}$) & \inferrule{ A_0\leftrightarrow\ldots \leftrightarrow A_n \\ B_0,\ldots,B_k,\neg B_{k+1},\ldots \neg B_n}
                      { \Gamma,\bigwedge\limits_{i=k+1}^ n (A_i\Rightarrow_\mathbf{CK_{CEM}} B_i)\rightarrow
                               \bigvee\limits_{j=0}^k(A_j\Rightarrow_\mathbf{CK_{CEM}} B_j)}  \\
\hline
\textbf{CK+CM} ($\Rightarrow_\mathbf{CK_{CM}}$) &  \vspace{-20pt}\begin{tabular}{ c l }
                                                               $A_i=A_j$ & for $i,j\in l(v),v\in G$ initial\\
                                                               $\bigcup\nolimits_{i\in l(v)} \{\neg B_i,\neg A_i\},A_j$ &
                                                               for $v\in G$ and $j\in l(w)$\\
                                                               & for some successor $w$ of $v$ \\
                                                               $\{\neg B_i\mid i\in I\},D$\\
                                                               $\bigcup\nolimits_{i\in I} \{\neg B_i,\neg A_i\},C$ \\
                                                               $\neg C, A_i$ & for $i\in I$ \\
                                                               \end{tabular}\\ 
                                                    & \inferrule{\empty}{\hspace{28pt}\Gamma,\bigwedge\limits_{i\in I}(A_i\Rightarrow_\mathbf{CK_{CM}} B_i)
                                                                 \rightarrow (C\Rightarrow_\mathbf{CK_{CM}} D) \hspace{28pt}}\\
                                                    & for an $I$-compatibility graph $G$   \\
\hline
System S ($\Rightarrow_{SysS}$) & \inferrule { \Delta_M(\nu(M)) \text{ for each } M\in\mathfrak{S}_{\Gamma_0}}
                                                            { \Gamma,\underbrace {\bigwedge\limits_{i\in I}(A_i\Rightarrow_{SysS} B_i)\rightarrow
                                                              (A_0\Rightarrow_{SysS} B_0)}_{\equiv:\Gamma_0}}\\
                                               & where $\mathfrak{S}_{\Gamma_0}$ denotes all S-structures $(S,\preceq)$ over $\Gamma_0$,\\
                                               & $\Delta_M([i])\equiv\bigcup_{k\simeq i}\{\neg A_k, \neg B_k\},
                                                 \{A_j\mid i \prec j\vee j\notin S\}$,  \\
                                               & $\Delta_M([0])\equiv\neg A_0, \bigcup_{0\neq k\simeq 0}\{\neg A_k, \neg B_k\},
                                                 B_0,\{A_j\mid j\notin S\}$.\\
\hline
 \end{tabular}
  \end{center}
  \caption{One-step complete, resolution closed modal rules for some conditional coalgebraic logics}
  \label{fig:modalRules2}
\end{figure}
\end{footnotesize}

Given a sequent $\Gamma$, a proof tree $t=(s,p,r_1,r_2,x)$ is a two-kinded tree with sequents $s$
and premises $p$ as nodes and a set $x$ of unexpanded sequents. The relation
$r_1:s\rightarrow \mathcal{P}(p)$ assigns to each sequent the set of its successor premises,
whereas the relation $r_2:p\rightarrow \mathcal{P}(s)$ returns for each premise the set of sequents
out of which it consists. A proof tree has its root at the sequent node
$\Gamma$.

For all expanded sequents $\Gamma\notin x$, $r_1(\Gamma)=\{\Sigma \mid (\Sigma,\Gamma) \text{ is an instance of a rule } R\}$.
For all premises $\Sigma$, $r_2(\Sigma)=\{\Gamma\mid \Sigma\in\Gamma\}$.

Unexpanded sequents $\Gamma\in x$ (i.e. sequents to which no rule has been applied
yet), are called \emph{open sequents}. A proof tree with $x=\emptyset$ is
called \emph{fully expanded}.

Any sequent to which either the rule (T) or the rule (At) may be applied
is called a \emph{successful leaf} of the proof tree.

Any sequent to which no rule may be applied is called an \emph{unsuccessful leaf}.

A proof tree with root $\Gamma$ is successful if
\begin{enumerate}
\item it is a successful leaf, or
\item there is \emph{any} $\Sigma\in r_1(\Gamma)$, s.t. \emph{all} $\Gamma'\in r_2(\Sigma)$ have a successful proof tree.
\end{enumerate}

For the sake of readability, we will usually show \emph{simplified proof trees} where only one rule application
is considered for each sequent (such that only one -- the interesting one -- of the possible premises of each sequent is shown).

It was shown in Schr\"oder, Pattinson, 2008, that this generic sequent calculus enables the construction of a PSPACE
decision procedure for the provability of the according coalgebraic logic -- as long as the used modal rules are one-step
complete and resolution closed as well as... [ask Lutz for name of property]

\section{Optimisations}

An optimisation technique for provability proofs of coalgebraic modal formulas is
a coalgebra homomorphism, mapping states of a witness coalgebra to states of an optimized
witness coalgebra (given a formula $\phi$, coalgebra $(A,\alpha)$ is a witness of $\phi$, whenever
there is a state $a\in A$ s.t. $\phi\in a$). In general, the idea is to map bigger coalgebras
to smaller coalgebras, thusly reducing the amount of necessary proof work.

In general, we are interested in finding as small as possible proofs for a given formula.
\emph{Simplification by tautologies} allows us to cut of short successful branches or to shorten
long successful branches in the proof tree (without the need to completely explore them).
\emph{Semantic branching} allows to ensure that two or more sub-proofs (which are induced by a
conjunction or a modal operator) are strictly disjoint in order to avoid repeated treatment of a
sepcific formula. The technique of \emph{controlled backtracking} allows us to cut of all
branches in the proof tree which contain formulas that have already been shown to lead to success.
\emph{Heuristics} try to choose the shortest path through the proof tree that leads to success.
Finally, the use of \emph{caching} techniques turns (parts of) the proof tree into a proof graph,
thusly eliminating the need for duplicate proofs while paying the price of possibly exponential
storage needs.

The most important expectation with regards to any optimisation is of course that it must respect the soundness and
completeness of the algorithm. Furthermore any optimisation involves some computations and might hence
sometimes \emph{decrease} the efficiency of the underlaying algorithm. Thus another important property
of optimisations is, that its gain should `usually` be larger than the invested effort.


We describe the optimisation techniques (all of which being adoptations or generalisations of the
techniques from Horrocks,Patel-Schneider, except the restriction to maximal applications of modal rules -
Horrocks,Patel-Schneider considered only logics for which this restriction is admissable) in more detail:

\subsection{Normalisation}

First of all, we fix the syntactic form in which we assume all formulas to be. This is
achieved by the process of \emph{normalisation} of formulas which is in fact not an
optimisation by itself, but rather a necessary condition for the usage of
optimisations such as simplification by tautologies.

\begin{footnotesize}
\begin{figure}[!h]
  \begin{center}
\begin{tabular}{| l | l |}
\hline
Syntactic construct & Normalisation \\
\hline
$\bot$ & $\neg\top$ \\
$\neg\neg\phi$ & $\phi$ \\
$\phi\wedge\psi$ & $\neg(\neg\phi\vee\neg\psi)$ \\
$\phi\rightarrow\psi$ & $\neg\phi\vee \psi$ \\
$\phi\leftrightarrow\psi$ & $(\phi\rightarrow\psi)\wedge(\psi\rightarrow\phi)$ \\
$\phi\vee\psi$ & $\cup\{\phi,\psi\}$ \\
$\cup\{\phi,(\cup\Gamma)\}$ & $\cup(\{\phi\}\cup\Gamma)$ \\
\hline
 \end{tabular}
  \end{center}
  \caption{Normalisation rules}
  \label{fig:normalisation}
\end{figure}
\end{footnotesize}

The input formula is assumed to already contain only the standard modalities
for each feature. Thus we normalize to truth, negation and
generalized disjunction (where the collection of all disjuncts is treated as a set
in order to remove duplicates and ordering).

\subsection{Simplification by Tautologies}

Optimisation by simplifaction rules makes use of specific (propositional or modal)
tautologies in order to deterministically rewrite any appearing formulas in a proof.
The application of simplification does not invoke any branching and is thus a
linear process.

However, only tautologies with very few (and hence quickly checked for) assumptions
should be used for simplification. Checking the assumptions of tautologies may involve
the checking of syntactical equivalence of formulas. This task may be optimized by
computing unique hashes for formulas and by then only checking for equality of
the hash-values.

It is possible to distinguish two kinds of rules for the optimisation by simplification:
Generic propositional rules which are the same for any coalgebraic modal logic
and more specific modal simplification rules which may differ between any two concrete
modal logics:

\subsubsection{Propositional Simplification}

The following is a non-exhaustive set of prositional simplification rules on sequents:

\begin{footnotesize}
\begin{figure}[!h]
  \begin{center}
\begin{tabular}{| l | l |}
\hline
Operator & Propositional Tautology \\
\hline
$\vee$ & $\cup\{\top,...\}\rightarrow \top$ \\
&  $\cup\{\neg\top,\phi_1,\ldots,\phi_n\}\rightarrow \cup\{\phi_1,\ldots,\phi_n\}$ \\
&  $\cup\{\phi,\neg\phi,\ldots\}\rightarrow \top$ \\
\hline
$\vee\wedge$ & $\cup\{\neg\phi,\neg(\neg\phi\vee\neg\psi)\}\rightarrow\psi$\\
\hline
 \end{tabular}
  \end{center}
  \caption{Propositional tautologies used for simplification}
  \label{fig:propTauts}
\end{figure}
\end{footnotesize}

The last propositional tautology represents boolean constraint propagation (BCP).

\subsubsection{Modal Simplification}

A modal feature may or may not allow for reasonable simplification:

\begin{footnotesize}
\begin{figure}[!h]
  \begin{center}
\begin{tabular}{| l | l |}
\hline
Feature & Modal Tautology \\
\hline
$\Box_\mathbf{M}$ & $\Box_\mathbf{M}\top\rightarrow \top$ \\
\hline
$\Box_\mathbf{K}$ &  $\Box_\mathbf{K}\top\rightarrow \top$ \\
 & ($\Box_\mathbf{K}\phi\wedge\Box_\mathbf{K}\psi\rightarrow \Box_\mathbf{K}(\phi\wedge\psi)$) \\
\hline
$\Box_\mathbf{KD}$ & $\Box_\mathbf{KD}\top\rightarrow \top$ \\
 & ($\Box_\mathbf{KD}\phi\wedge\Box_\mathbf{KD}\psi\rightarrow \Box_\mathbf{KD}(\phi\wedge\psi)$) \\
 & $\neg\Box_\mathbf{KD}\bot\rightarrow \top$ \\
\hline
$\Box^a_\mathbf{HM}$ &  $\Box^a_\mathbf{HM}\top\rightarrow \top$ \\
 & ($\Box^a_\mathbf{HM}\phi\wedge\Box^a_\mathbf{HM}\psi\rightarrow \Box^a_\mathbf{HM}(\phi\wedge\psi)$) \\
\hline
$\Box^C_\mathbf{C}$ & $\Box^C_\mathbf{C}\top\rightarrow \top$ \\
\hline
$\Diamond^i_\mathbf{G}$ &  \\
\hline
$\Diamond^p_\mathbf{P}$ &  \\
\hline
$\Rightarrow_\mathbf{CK}$ & $(\phi\Rightarrow_\mathbf{CK}\top)\rightarrow\top$ \\
\hline
$\Rightarrow_\mathbf{CK_{CEM}}$ & $(\phi\Rightarrow_\mathbf{CK_{CEM}}\top)\rightarrow\top$ \\
\hline
$\Rightarrow_\mathbf{CK_{CM}}$ & $(\phi\Rightarrow_\mathbf{CK_{CM}}\top)\rightarrow\top$ \\
                               & $((\top\Rightarrow_\mathbf{CK_{CM}}\bot)\rightarrow(\phi\Rightarrow_\mathbf{CK_{CM}}\psi))\rightarrow \top$ \\
                               & $((\top\Rightarrow_\mathbf{CK_{CM}}\phi)\rightarrow(\top\Rightarrow_\mathbf{CK_{CM}}\top))\rightarrow \top$ \\
                               & $((\phi\Rightarrow_\mathbf{CK_{CM}}\bot)\rightarrow(\bot\Rightarrow_\mathbf{CK_{CM}}\psi))\rightarrow \top$ \\
                               & $((\bot\Rightarrow_\mathbf{CK_{CM}}\phi)\rightarrow(\bot\Rightarrow_\mathbf{CK_{CM}}\phi))\rightarrow \top$ \\

 \hline
$\Rightarrow_{SysS}$ &  \\
\hline
 \end{tabular}
  \end{center}
  \caption{Modal tautologies for simplification}
  \label{fig:modTauts}
\end{figure}
\end{footnotesize}

\subsection{Semantic Branching}

The technique of semantic branching allows us to make sure that created
sub-proofs are strictly disjoint. This prevents duplicate proofs a formulas
which might otherwise appear in both sub-proofs. This technique may however
enlarge the proof tree.

\subsubsection{Propositional Semantic Branching}

In the propositional case, there is just one branching rule, the conjunction rule.
A semantic version of this rule is shown in Figure~\ref{fig:synSemConj}.
The justification for replacing the standard conjunction rule by the rule for
semantic conjunction branching is simple:

\begin{lemma}
A sequent is provable by syntactic propositional branching iff it is
provable by semantic propositional branching.
\begin{proof}
This fact follows from the propositional tautology
\begin{quote}
$(A\wedge B)\leftrightarrow (A\wedge (A\rightarrow B))$.
\end{quote}
\end{proof}
\end{lemma}

\begin{footnotesize}
\begin{figure}[!h]
  \begin{center}
\begin{tabular}{| c | c |}
\hline
Syntactic branching & Semantic branching \\
\hline
\inferrule{ \Gamma,A \\ \Gamma,B } { \Gamma,A \wedge B} & \inferrule{ \Gamma,A \\ \Gamma,\neg A,B } { \Gamma,A \wedge B}\\
\hline
 \end{tabular}
  \end{center}
  \caption{Syntactic vs. Semantic branching on conjunctions}
  \label{fig:synSemConj}
\end{figure}
\end{footnotesize}

\begin{example}
Consider the sequent $\Gamma=\{A\wedge B, C\wedge A\}$ and let $A$ be provable,
but let $A$ induce a lengthy proof tree. The use of 
syntactic branching leads to the following redundant proof tree:
\begin{quote}
\centerline{
\inferrule{\inferrule {\inferrule{\inferrule{\checkmark}{\ldots}}{A}\\\inferrule{\inferrule{\checkmark}{\ldots}}{A,C}}{A,A\wedge C} \\ 
                       \inferrule {\inferrule{\inferrule{\checkmark}{\ldots}}{B,A}\\\inferrule{\ldots}{B,C}}{B,A\wedge C}}{A\wedge B, A\wedge C}}
\end{quote}
Semantic branching on the other hand may lead to a much smaller proof tree (since $A$ has
to be shown only once):
\begin{quote}
\centerline{
\inferrule{\inferrule {\inferrule{\inferrule{\checkmark}{\ldots}}{A}\\ \inferrule{\checkmark}{A,\neg A, C}}{A,A\wedge C} \\ 
                       \inferrule {\inferrule{\checkmark}{\neg A,B,A}\\\inferrule{\ldots}{\neg A,B,C}}{\neg A,B,A\wedge C}}{A\wedge B, A\wedge C}}
\end{quote}

\end{example}

\subsubsection{Modal Semantic Branching}

The general principle that can be extracted from propositional semantic branching
is a follows:

A syntactic branching rule
\begin{quote}
\centerline{\inferrule{\Gamma_1\\\ldots\\\Gamma_i\\\ldots\\\Gamma_n}{\Gamma}}
\end{quote}
may be (repeatedly) replaced by a `more semantic` branching rule
\begin{quote}
\centerline{\inferrule{\Gamma_1\\\ldots\\ \Gamma'_i\\\ldots\\\Gamma_n}{\Gamma}}
\end{quote}
where $\Gamma'_i=\bigwedge_{t\in T}\Gamma_t\rightarrow\Gamma_i$ for some
$T\subseteq\{1,\ldots,n\}\setminus\{i\}$.

The justification for this generalized semantic branching is as simple as in the propositional
case:

\begin{lemma}
A sequent is provable by syntactic modal branching iff it is
provable by semantic modal branching.
\begin{proof}
This fact follows from the propositional tautology
\begin{quote}
$(\bigwedge_{t\in T}\Gamma_t\wedge \Gamma_i) \leftrightarrow 
 (\bigwedge_{t\in T}\Gamma_t\wedge(\bigwedge_{t\in T}\Gamma_t\rightarrow \Gamma_i))$.
\end{quote}
\end{proof}
\end{lemma}

\begin{footnotesize}
\begin{figure}[!h]
  \begin{center}
\begin{tabular}{| l | l |}
\hline
Feature & Semantic Branching\\
\hline
$\diamond^i_\mathbf{G}$ &   \\
\hline
\ldots& \ldots \\
\hline
$\Rightarrow_\mathbf{CK}$ & \inferrule{ A_0\leftrightarrow\ldots \leftrightarrow A_n \\ \neg(A_0\leftrightarrow\ldots \leftrightarrow A_n),\neg B_1,\ldots, \neg B_n,B_0 }
                      { \Gamma,\bigwedge\limits_{i=1}^n(A_i\Rightarrow_\mathbf{CK} B_i)\rightarrow (A_0\Rightarrow_\mathbf{CK} B_0)} \\
\hline
 \end{tabular}
  \end{center}
  \caption{Specific semantic branching for examplary modal logics}
  \label{fig:semBranch}
\end{figure}
\end{footnotesize}

\subsection{Controlled Backtracking}

The technique of \emph{controlled backtracking} (also known as dependency directed
backtracking) is in fact a special instance of caching. When this optimisation is
employed, each sequent $\Gamma$ depends on a set $\sharp(\Gamma)$ of
sequents (namely on those sequents which introduced any formula $\phi\in\Gamma$ into the proof tree).
Whenever a sequent turns out to be provable, its provability depends
on possibly several sequents. These sequents are called the respective
\emph{dependency set} of $\{\top\}$, denoted by $\sharp(\{\top\})$ (these sequents are 
also known as the \emph{promotors} of the successful leaf).

We have yet to formally define dependency sets, but we will rely on the following
crucial fact:

\begin{lemma} Given a dependency set $\sharp(\{\top\})$ of a succesfull leaf, any
sequent $\Gamma$ s.t. $\Gamma\in \sharp(\{\top\})$, is provable.
\label{fact:promtrs}
\end{lemma}

Due to Lemma~\ref{fact:promtrs}, it will
not be necessary to treat any open sequents between the current successful node in the
proof graph and the first node which is contained in the according dependency set.

To achieve this, a set of \emph{current} dependency sequents is used, and whenever a
sequent is provable, the current set of dependency sequents is defined as the set of
dependency sequents of $\{\top\}$. Now all those open sequents which
appear during the following backtracking through the proof tree and which
are created by a sequent which is \emph{not} in the current set of dependency states,
may be directly set to $provable$.

If however a node in the proof graph which is contained in the current set of dependency
states is reached, the current set of dependency states is set to $\emptyset$ and the
proof may continue as usual.

\begin{definition} Depending on the shape of a proof tree, $\sharp(\Gamma)$ is defined as follows: 
\begin{itemize}
\item If $\Gamma$ is the root of the proof tree, $\sharp(\Gamma)=\Gamma$.
\item If $\Gamma$ is obtained from $\Gamma'$ by means of normalisation or simplification,
      then $\sharp(\Gamma):=\sharp(\Gamma')$.
\item If $\Gamma$ is the only sequent in the premise of a rule application of a non-branching rule to
$\Gamma'$, then $\sharp(\Gamma):=\sharp(\Gamma')$
\item If $\Gamma_1,\ldots,\Gamma_n$ are $n$ sequents from the premise of the rule
application of a branching rule to $\Gamma'$,
the dependancy set of any formula $\phi\in \bigcup_{i\in\{1,\ldots,n\}} \Gamma_i$ is
defined as follows: if $\forall i\in\{1,\ldots,n\}. \phi\in \Gamma_i$, then $\sharp(\{\phi\}):=\sharp(\Gamma)$,
else $\sharp(\{\phi\}):=\Gamma$.
\item Finally, $\sharp(\Gamma_1\cup \Gamma_2):=\sharp(\Gamma_1)\cup\sharp(\Gamma_2)$
\end{itemize}
\end{definition}

\begin{footnotesize}
\begin{figure}[!h]
  \begin{center}
\begin{tabular}{| l | l | l |}
\hline
Feature & Modal Rule & Dependencies\\
\hline
$\Box_\mathbf{K}$ & \inferrule{ \overbrace{\bigwedge\nolimits_{i=1}^n A_i\rightarrow B}^{\equiv:\Gamma_1} }
                      { \Gamma, \underbrace{\bigwedge\nolimits_{i=1}^n \Box_\mathbf{K} A_i\rightarrow \Box_\mathbf{K} B}
                        _{\equiv:\Gamma'}} & $\sharp(\Gamma_1):=\sharp(\Gamma')$ \\
\hline
$\Rightarrow_\mathbf{CK}$ & \inferrule{ \overbrace{A_0\leftrightarrow\ldots \leftrightarrow A_n}^{\equiv:\Gamma_1} \\ 
                                        \overbrace{\neg B_1,\ldots \neg B_n,B_0}^{\equiv:\Gamma_2}  }
                      { \Gamma, \underbrace {\bigwedge\nolimits_{i=1}^n (A_i\Rightarrow_\mathbf{CK} B_i)\rightarrow
                        (A_0\Rightarrow_\mathbf{CK} B_0)}_{\equiv:\Gamma'}} & $\sharp(\Gamma_1,\Gamma_2):=\Gamma'$ \\
\hline
 \end{tabular}
  \end{center}
  \caption{Propagation of dependencies over some examplary modal features}
  \label{fig:depProp}
\end{figure}
\end{footnotesize}

Inituitively, a single application of for instance the modal rule of \textbf{K} 
does not \emph{choose} any subformulas of its conclusionary sequent to generate new sequents
(this is already a consequence of the fact that this rule does not branch).
Hence any introduced formulas will depend on those formulas on which the conclusionary sequent
already depended.
On the other hand, the modal rule of \textbf{CK} selects some subformulas (the antecedents)
from its conclusionary sequent to be in the first sequent of the premise and others
(the consequents) to be in the second sequent of the premise, thus a choice is made and the
introduced formulas will directly depend on the conclusionary sequent.

The following Lemmas establish that dependency directed backtracking as defined above
is a sound optimisation technique for any coalgebraic model logic:

\begin{lemma}
Assume an application of any (possibly branching) rule:
\begin{quote}
\centerline{\inferrule{\Gamma_1\\\ldots\\\Gamma_n}{\Gamma}}
\end{quote}
Assume a sequent $\Gamma'$ such that $\Gamma\notin\sharp(\Gamma')$,
but $\Gamma'\in \Gamma_i$ for any $0<i\leq n$.
Then provability of $\Gamma'$ implies the provability of all sequents of
the premise of the rule.
\begin{proof} Let $\Gamma'$ be provable. Since $\Gamma\notin\sharp(\Gamma')$, we have a non-selecting
rule application (w.r.t. $\Gamma'$). For non-branching rules, the implication is trivial. For a
branching rule, it follows directly from the definition of $\sharp(\_)$ and the fact that one
$\Gamma_j$ contains $\Gamma'$, that all $\Gamma_j$ contain $\Gamma'$.
Hence all the sequents in the premise of the rule are provable.
\end{proof}
\label{lemma:provabilityPropagates}
\end{lemma}

%%\begin{lemma}
%%Assume an application of any (possibly branching) rule:
%%\begin{quote}
%%\inferrule{\Gamma_1\\\ldots\\\Gamma_n}{\Gamma}
%%\end{quote}
%%Assume a formula $\phi$ and its dependency set $\sharp(\{\phi\})$, s.t.
%%$\Gamma\notin\sharp(\{\phi\})$. If $\phi$ is contained in any
%%sequent of the premise, it is contained in all sequents of the premise.
%%\begin{proof}
%%By case distinction on the definition of $\sharp(\Gamma)$.
%%\end{proof}
%%\label{lemma:depAnyAll}
%%\end{lemma}

%%A promoting set of formulas is any subset of appearing conclusionary sequents $\Gamma$ 
%%of a rule application $\Gamma'_1 \ldots \Gamma'_n / \Gamma$ s.t. $\sharp(\Gamma)=\sharp(\Gamma'_i)$.
%%i.e. it is any subset of any $\Gamma$ as long as $\Gamma\notin\sharp(\{\top\})$



%%\begin{fact}
%%A sequent that contains a set of promoting formulas, is provable.
%%\label{fact:containProv}
%%\end{fact}

\begin{lemma}
Let the following be a (not yet fully explored) proof tree for $\Gamma$ (where $\Gamma^i_j$ denotes
the sequent number $j$ on level $i$; in case that $j>1$ and $i>1$, $\Gamma^i_j$ is assumed to be an
open sequent):
\begin{quote}
\centerline{\inferrule{\inferrule{\inferrule{\inferrule{\inferrule{\checkmark}{\Gamma^{n}_{1}}}{\ldots}}{\Gamma^{2}_{1}}
                                           \\\ldots\\\Gamma^{2}_{i}}{\Gamma^{1}_{1}} \\ \ldots \\ \Gamma^{1}_{j}}{\Gamma}}
\end{quote}
Further, let $\sharp(\{\top\})$ denote the dependency set of the successful leaf. If for all $\Gamma^i_1$
with $0<i\leq n$, $\sharp(\{\top\})\not\subseteq\Gamma^i_1$,
and if $\sharp(\{\top\})\subseteq\Gamma$ (i.e. $\Gamma$ is the first sequent contained in the dependency
set of the successful leaf), then $\Gamma^1_1$ is provable (since any $\Gamma^i_j$ with $i>1$ is provable).

\begin{proof}
Lemma~\ref{lemma:provabilityPropagates} ensures, that the provability of the successful leaf
propagates to all the open sequents $\Gamma^i_j$ for $i,j>1$ (since $\sharp(\{\top\})\not\subseteq\Gamma^{i-1}_j$).
Thus we are not able to find any sequent of level $>1$ that is not provable.

%%We try to find any $\Gamma^i_j$ which is not provable. Since $\Gamma^n_1$ is provable, any
%%open sequent that could be not provable must have been created by a branching rule from
%%the sequent $\Gamma^{i-1}_1$. However, since $\sharp(\{\top\})\not\subseteq\Gamma^{i-1}_1$ and since
%%$\Gamma^i_1$ contains a promoting set of formulas, 
%%a promoting set of formulas is by Lemma~\ref{lemma:depAnyAll} also contained in $\Gamma^i_j$,
%%thus $\Gamma^i_j$ is by Fact~\ref{fact:containProv} provable.
%%This shows that any $\Gamma^i_j$ with $i>1$ and in fact also $\Gamma^1_1$ is provable.
\end{proof}
\end{lemma}

To summarize, once we found a successful leaf, it is not necessary to treat the whole sub-proof
which is induced by $\Gamma^1_1$, since $\Gamma^1_1$ is always provable in the above situation.

\subsection{Maximal Application of Modal Rules}

A single modal sequent rule representation usually describes in fact a set of modal rules, each of the rules
treating a different combination of positive and/or negative literals of the respective
feature. Given a sequent $\Gamma$ to match, any such modal rule could be applied to $\Gamma$,
as long as the conclusion of the rule is equal to $\Gamma$. It may be observed however
that it often suffices to consider \emph{maximal} applications of the modal rules. An application
$(\Gamma_1, \Gamma'_1)$ of a modal rule to $\Gamma_1\subseteq \Gamma$ is maximal if there is no
formula $\phi$ in $\Gamma\setminus\Gamma_1$ s.t. the conclusion of any the modal rule would still
be of shape $\Gamma_1\cup\{\phi\}$.

Given a set of modal rules $\mathcal{R}_M$, the restriction to maximal applications of rules
from $\mathcal{R}_M$ is admissable whenever the provability of the premise of any non-maximal
application of a rule from $\mathcal{R}_M$ implies that there is a maximal application of
a rule $\mathcal{R}_M$ with a provable premise.

Such a restriction is admissable (and in fact highly recommended) for many modal logics (see Table~\ref{fig:admisRes})

A negative example is the feature of \textbf{CK} (or similarly, the feature of
\textbf{CKCEM}):

\begin{example}
Let us consider the examplary sequent $\Gamma=\{\neg(A_1\Rightarrow B_0),\neg(A_2\Rightarrow B_0)
,(A_0\Rightarrow B_0)\}$ and assume that $A_0\leftrightarrow A_1$ but $A_1\nleftrightarrow A_2$. 
The only maximal application of the modal rule \textbf{CK} would
lead to three new sequents: 

\begin{quote}
\centerline{\inferrule{ \inferrule {\checkmark} {A_0\leftrightarrow A_1} \\ 
            \inferrule {\lightning} {A_1\leftrightarrow A_2} \\
            \inferrule {\checkmark} {\neg B_0,B_0} }
          { \neg(A_1\Rightarrow B_0),\neg(A_2\Rightarrow B_0),(A_0\Rightarrow B_0)}}
\end{quote}

The second of these sequents is not provable by assumption, thus $\phi$ can not be shown to be provable.

However, using a non-maximal application of the modal rule \textbf{CK} to
the subsequent $\Gamma'=\{\neg(A_1\Rightarrow B_0),(A_0\Rightarrow B_0)\}$ of $\Gamma$, the provability of $\phi$ may be
shown: 

\begin{quote}
\centerline{\inferrule{ \inferrule {\checkmark} {A_0\leftrightarrow A_1} \\ 
            \inferrule {\checkmark} {\neg B_0,B_0} }
          { \neg(A_1\Rightarrow B_0),\neg(A_2\Rightarrow B_0),(A_0\Rightarrow B_0)}}
\end{quote}
\end{example}

Even though the restriction to maximal applications of the modal rule of \textbf{CK} is not admissable,
it is possible to obtain a nearly equivalent optimisation of the proofs for \textbf{CK} by the use of
intelligent modal heuristics (see 3.8).

\begin{footnotesize}
\begin{figure}[!h]
  \begin{center}
\begin{tabular}{| l | l |}
\hline
Modal Features & Admissability\\
\hline
$\Box_{\mathbf{K}}, \Box_{\mathbf{KD}}, \Box_{\mathbf{M}}, \Box_{\mathbf{HM}}^a,
  \diamond^i_{\mathbf{G}}, \diamond^p_{\mathbf{P}}, \Rightarrow_{\mathbf{CK_{CM}}}, \Rightarrow_{SysS}$ & Yes\\
\hline
$\Rightarrow_{\mathbf{CK}}, \Rightarrow_{\mathbf{CK_{CEM}}}$ & No\\
\hline
 \end{tabular}
  \end{center}
  \caption{Admissability of restriction to maximal application of modal rules}
  \label{fig:admisRes}
\end{figure}
\end{footnotesize}

\subsection{Proof-tree Trimming}

Due to the two-kinded nature of a proof tree and the alternating existential and universal
provability requirements between the two kinds of nodes, a proof tree may be trimmed,
once enough information about the provability of specific sequents has been obtained.

If there are several possible premises for one sequent and the provability of one of the premises
has already been shown, the other premises for the same sequent need not be explored any more.
Similary, if there are several sequents in a premise, as soon as one of the
sequents turns out to be not provable the whole premise is not provable any more
and the other sequents of the premise need not be explored.

In more detail, the following operations are valid trimmings of a proof tree:
\begin{enumerate}
\item Removal of non-provable premises,
\item Removal of all premises that are parallel to a provable premise ($r_1(\Gamma)=
      \{\Sigma_1,\ldots,\Sigma_i,\ldots,\Sigma_n\}$ and $\Sigma_i$ provable $\rightsquigarrow$
      $r_1(\Gamma)=\{\Sigma_i\}$),
\item Removal of provable sequents,
\item Removal of all sequents that are parallel to a non-provable sequent ($r_2(\Sigma)=
      \{\Gamma_1,\ldots,\Gamma_i,\ldots,\Gamma_n\}$ and $\Gamma_i$ non-provable $\rightsquigarrow$
      $r_2(\Sigma)=\{\Gamma_i\}$).
\end{enumerate}
The resulting tree is no complete proof tree any more (since several branches may have been
cut of). However, the following fact establishes that trimming is a sound optimisation
technique:

\begin{lemma} A sequent $\Gamma$ has a successful proof tree iff it has a successful
trimmed proof tree.
\begin{proof}
The branches which are cut off do not interfere with the provability of $\Gamma$. See 4.3
for more details.
\end{proof}
\end{lemma}

\subsection{Caching}

Using the optimisation technique of caching, it is possible to store the satisfiability
status of subformulas that appear during the course of a proof. This prevents us from
treating the same subformula more than once, thus saving redundant proof-work.

The cost of this optimisation technique is an increase in the use of memory during
the proof. Especially in the case of logics whose decision procedure is in PSPACE,
the use of caching may in theory negate the property of optimal complexity of the used
algorithm (bringing those logics in theory to EXPTIME).

Caching is a bit troublesome in combination with dependency directed
backtracking since a formula may depend on several sequents if
it appears at different positions in the proof. However, it is sufficient
to simply use the union of the respective dependency sets whenever a formula
appears at least twice.

\subsection{Heuristics}

A heuristic guides the search in the proof tree and chooses 

\begin{enumerate}
\item a premise from the set of all open premises,
\item a meta-disjunction within this premise and
\item one sequent within the selected meta-disjunction.
\end{enumerate}

To this end, the heuristic assigns a priority to each open premise,
each open meta disjunction and each open sequent.

The sequent with the highest overall priority will be the next to be explored.

A reasonable heuristic helps (if possible) in finding a shorter way to show
provability of a formula.

\subsubsection{Propositional expansion first}

It is a reasonable tactic of expansion to first completely treat all
propositional formulas in a sequent before considering the modal
literals. This may be achieved by assigning high priority to
those sequents, meta-disjunctions and premises which are introduced
by the conjunction rule. Such sequents, meta-disjunctions and premises
which are generated by an application of a modal rule are given a low priority.

\subsubsection{Classic Heuristics}

On top of this preference of propositional expansion, one might add any
other heuristic, such as for instance one of the following two classic
heuristics (or even a combination of both):

\begin{itemize}
\item \emph{most appearing first}: Assign the highest priority to open sequents
that appear in the highest number of formulas. The priority of meta-disjunctions
is defined to be the priority of the contained sequent. The priority of a premise
is the maximum of the priorities of the contained meta-disjunctions.
\item \emph{oldest first}: \ldots
\end{itemize}


\subsubsection{Modal Heuristics}

Here is a specific modal heuristic for \textbf{CK} (a similar heuristic
if suitable for \textbf{CKCEM}). Note, that this heuristic is only applicable
if caching is enabled:

Within any premise of a modal rule, the short sequents of the form $A\leftrightarrow B$ are
given the highest priority (in combination with caching, this leads to a complete partition
of modal antecedents into equivalence classes).

Once this is done, we give high priority (e.g. 1) to \emph{maximal} (w.r.t the cached equivalence
classes) premises and priority of 0 to non-maximal premises (i.e. we do not treat non-maximal
premises at all). As long as not all significant equivalences have yet been shown or refuted,
the priority of a premise stays at a normal value (e.g. 0.5).

In combination with caching (of logical equivalences), this heuristic implements the
optimizations of \textbf{CK} and \textbf{CKCEM} proposed in Hausmann,Schr\"oder 2009.

\section{Global Caching Algorithm}

This is an adoptation of the Algorithm from Pattinson et.al.

Key differences are 

\begin{enumerate}
\item We treat provability of formulas here instead of satisfiability. However, a formula is 
satisfiable whenever its negation is not provable.
\item The introduction of meta-disjunctions.
\item $X$ is a structured set now. This allows heuristics to first choose a premise to treat,
then choose a meta-disjunction from this premise and finally choose a sequent from the chosen
meta-disjunction.
\item The propagation is optimized now. Instead of computing the fixpoints of all the
undecided sequents every time, classes of coherent sequents are stored. The propagation
checks only 'whether the new sequents decide any of the classes'.
\end{enumerate}

\subsection{The proof graph and transitions}

A sequent is a set of formulas. A meta-disjunction is a set of sequents.
A premises is a set of meta-disjunctions.

A graph is a tuple $(A,E,U,X,L_1,L_2)$ where $A,E,U\subseteq Seq$ are sets of sequents,
$X\subseteq Prems$ is a set of premises.

Given a sequent $\Gamma$, we define the set of all possible rule applications to $\Gamma$,
$AR(\Gamma)=\{(\Gamma,\Sigma)\in\mathcal{R} \}$, the set of all obtainable premises
$AP(\Gamma)=\{\Sigma\mid (\Gamma,\Sigma)\in AR(\Gamma)\}$ and the set of all selections
$AS(\Gamma)=\{(\Sigma,\Gamma')\mid \Sigma\in AP(\Gamma)\wedge\Gamma'\in\Sigma\}$.

We fix two effectively computable transitions between graphs:

\begin{itemize}
\item \emph{Expand:} Given a sequent $\Gamma$ that is contained in any meta-disjunction
from any premise in $X$, this transition leads from a graph $g$ to the graph $g'$ that is
obtained from $g$ by expanding $\Gamma$. To this end, $X$ is extended by all the premises
from $AR(\Gamma)$ and $\Gamma$ is removed from $X$ ($X'=(X\cup AP(\Gamma))\setminus \Gamma$).
Furthermore, $L_1$ is extended by all possible rule applications for $\Gamma$
($L_1'=L_1 \cup AR(\Gamma)$). Finally, $L_2$ is extended by all possible selections
for rule applications for $\Gamma$ ($L_2'=L_2\cup AS(\Gamma)$).

In conclusion, this procedure applies all applicable rules to $\Gamma$, marks all resulting 
premises as not yet expanded, marks $\Gamma$ as expanded and stores the implied transitions in
$L_1$ and $L_2$.

For a fixed sequent $\Gamma$ and a graph $g$, we denote the expansion of $\Gamma$ in $g$ by
$exp(g,\Gamma)$. If $g'=exp(g,\Gamma)$, we also write $g\stackrel{\Gamma}\rightarrow_E g'$.

\item \emph{Propagate:} This procedure evaluates the given graph as far as yet possible: The sets
$X$, $L_1$ and $L_2$ remain unchanged. The set of provable sequents however is extended by the least
fixpoint of a function $M^L$ ($A'=A\cup\mu(M^L)$). The set of not provable sequents is extended by the
greatest fixpoint of a function $W^L$ ($E'=E\cup\nu(W^L)$). Finally, those sequents that are known to be
either provable or not provable are removed from the set of undecided sequents
($U'=U\setminus(A'\cup E')$).\\

The functions which are used for the fixpoint computations are defined as follows:\\

\begin{quote}
$M^L(X)=\{\Gamma\in U\mid\forall(\Gamma,\Sigma)\in L_1.\exists (\Sigma,\epsilon)\in L_2.\forall \Gamma'\in\epsilon. \Gamma'\in X\cup A\}$\\
$W^L(X)=\{\Gamma\in U\mid\exists(\Gamma,\Sigma)\in L_1.\forall (\Sigma,\epsilon)\in L_2.\exists \Gamma'\in\epsilon. \Gamma'\in X\cup E\}$\\
\end{quote}

Given a graph $g$, we denote the graph obtained by propagating in $g$ by
$prp(g)$. If $g'=prp(g)$, we also write $g\rightarrow_P g'$.

\end{itemize}

\subsection{The Algorithm}

In order to show the provability of a formula $\phi$, we start off with the sequent
$\Gamma_0=\{\phi\}$:

\begin{example}
(Show provability of $\Gamma_0$)
\begin{enumerate}
\item Set $g=(A=\{\emptyset\}$, $E=\emptyset$, $U=\{\Gamma_0\}$, $X=AP(\Gamma_0)$,
$L_1=AR(\Gamma_0)$, $L_2=AS(\Gamma_0))$.

\item If $X=\emptyset$, return the current graph as result.
\item Otherwise select any premise
$\Sigma$ from $X$, select any meta-disjunction $\epsilon$ from $\Sigma$ and select any sequent
$\Gamma$ from $\epsilon$. Compute $g'=exp(g,\Gamma)$ s.t.
$g\stackrel{\Gamma}\rightarrow_E g'$. (Optionally: Compute $g''=prp(g')$ s.t.
$g'\rightarrow_P g''$; If $\Gamma_0\in A''\cup E''$, break and return the current
graph as result.)
\item Set the current graph to $g'$ (or $g''$ if a propagation step took place). Continue with 2.
\end{enumerate}
After the loop has finished, propagate one final time and then
let $A$ ($E$) be the set of provable (not provable) sequents of the resulting
graph. If $\Gamma_0\in A$, return \verb|True|, if $\Gamma_0\in E$, return
\verb|False|, otherwise return \verb|Undecided|.
\end{example}

This algorithm has been shown to be
\begin{description}
\item{a) }sound and complete w.r.t. provability of formulas in the logic that is
represented by the utilized set of rules $\mathcal{R}$,
\item{b) }of complexity EXPTIME, if the underlaying ruleset allows for it.
\end{description}

\subsection{Improving the algorithm}

Due to the inherent inefficiency of computing the propagation of $A$
($E$) through the proof graph, we devise the following more efficient method:

\begin{definition}
A \emph{position} $p\in Pos=\{c_1\ldots c_n\mid c_1,\ldots,c_n\in\mathbb{N}\}\cup\{[]\}$ is a finite
list of natural numbers.

A \emph{proof formula} is defined as follows (for $j,p\in Pos$):
\begin{quote}
$\mathcal{PF} \ni \psi_1,\ldots,\psi_n := a_j \mid \wedge \{\psi_1,\ldots,\psi_n\} \mid \vee \{\psi_1,\ldots,\psi_n\} \mid l_p$
\end{quote}

The subformula $\phi|_p$ of a proof formula $\phi$ at position $p=c_1\ldots c_n$ is defined
recursively. If $n=0$, then $\phi|_p=\phi$. If $n=1$ (i.e. $p=c_1$), then $p_2=[]$, otherwise
(i.e. $n>1$, $p=c_1c_2\ldots c_n$), $p_2=c_2\ldots c_n$.
\begin{itemize}
\item If $n>0$, $a_j|_p$ and $l_i|_p$ are undefined.
\item Let $\phi=\wedge \{\psi_1,\ldots,\psi_n\}$ or $\phi=\vee \{\psi_1,\ldots,\psi_n\}$. If $0<c_1\leq n$,
$\phi|_p=\psi_{c_1}|_{p_2}$; if $c_1>n$, $\phi|_p$ is undefined.
\end{itemize}

The partial \emph{indexing function} $\ind{\_}:Seq + Prem \rightarrow Pos$
maps sequents $\Gamma\in Seq$ or premises $\Sigma\in Prem$ from the defined
subset of its domain to a position.

We also introduce sets of provable positions $A_p\subseteq Pos$ and non-provable
positions $E_p\subseteq Pos$ as well as the set $tcs$ of positions that have been
expanded since the last propagation step and that are relevant to the propagation.
\end{definition}

When constructing the proof graph $g$ for a formula $\phi$, we build up a proof formula
$pf(g)$ in parallel, beginning with $pf(g_0)=a_{\ind{\{\phi\}}}$, $\ind{\_}$ undefined except
for $\ind{\{\phi\}}=[]$ and $A_p=E_p=tcs=\emptyset$.

\noindent\textbf{Expanding the proof formula:} 
Say we expand sequent $\Gamma$, s.t. $g\stackrel{\Gamma}\rightarrow_E g'$ for $g'=exp(g,\Gamma)=(A',E',U',X',L_1',L_2')$.
This means that there are $n$ premises to $\Gamma$ (i.e. $|AP(\Gamma)|=n$) and for each $\Sigma_i \in AP(\Gamma)$,
$(\Gamma,\Sigma_i)\in L_1'$. Also, for each $\Sigma_i \in AP(\Gamma)$, there is an $m_i$ s.t.
$|\{\Gamma_j\mid (\Sigma_i,\Gamma_j)\in AS(\Gamma)\}|=m_i$ and each such $(\Sigma,\Gamma_j)$ is contained in $L_2'$.
In other words, $\Gamma$ has $n$ premises, each premise consists of $m_i$ sequents and the according
transitions are stored in $L_1'$ and $L_2'$.

The according changes to the proof formula and the indexing function are as follows:

\noindent First we expand the sequent $\Gamma$:
\begin{itemize}
\item Replace $a_{\ind{\Gamma}}$ with $\vee\{b_1,\ldots,b_n\}$.
\item For $0<i\leq n$, if $\ind{\Sigma_i}$ is defined (i.e. the premise was encountered before),
replace $b_i$ by $l_{\ind{\Sigma_i}}$. If $\ind{\Sigma_i}\in A_p\cup E_p$, add $\ind{\Gamma}$ to $tcs$.
\item For each remaining $b_i$ (i.e. for each new premise), set $\ind{\Sigma_i}=\ind{\Gamma}i$.
\item \emph{Optional:} If $n=1$, then $\vee\{b_1\}=b_1$ and $\ind{\Sigma_1}=\ind{\Gamma}$.
\item If $n=0$, then add $\ind{\Gamma}$ to $tcs$.
\end{itemize}

\noindent Then we expand all the premises $\Sigma_i$:
\begin{itemize}
\item Replace any $b_i$ with $\wedge\{a_1,\ldots,a_{m_i}\}$.
\item For $0<j\leq m_i$, if $\ind{\Gamma_j}$ is defined (i.e. the sequent was encountered before), 
replace $a_j$ by $l_{\ind{\Gamma_j}}$. If $\ind{\Gamma_j}\in A_p\cup E_p$, add $\ind{\Sigma_i}$ to $tcs$.
\item For each remaining $a_j$ (i.e. for each new sequent), set $\ind{\Gamma_j}=\ind{\Sigma_i}j$.
\item \emph{Optional:} If $m_i=1$, then $\wedge\{a_1\}=a_1$ and $\ind{\Gamma_1}=\ind{\Sigma_i}$.
\item If $m_i=0$, then add $\ind{\Sigma_i}$ to $tcs$.
\item Replace any $a_j$ with $a_{\ind{\Gamma_j}}$.
\end{itemize}

\noindent\textbf{Propagation in the proof formula:}
Propagation of $A$ ($E$) through the proof formula is realized by iterative approximation of the
fixpoints. 
Any of the rules from Figure~\ref{fig:propagation} is applied repeatedly for each $pj\in tcs$ until
no rule may be applied for this specific $pj$ any more, $pj$ is removed from $tcs$ afterwards:
\begin{footnotesize}
\begin{figure}[!h]
  \begin{center}
\begin{tabular}{| l | l | l |}
\hline
Condition(s) & Changes to $A_p$ ($E_p$) & Changes to $tcs$\\
\hline
$\phi|_{pj}=\wedge\emptyset$ & add $pj$&\\
\hline
$\phi|_{pj}=\vee\emptyset$ & (add $pj$)&\\
\hline
$\exists q\in Pos.\phi|_{q}=l_{pj}$, $pj\in A_p$ & add $q$ & add $q$\\
\hline
$\exists q\in Pos.\phi|_{q}=l_{pj}$, $pj\in E_p$ & (add $q$) & add $q$\\
\hline
$\phi|_{p}=\wedge\{\psi_1,\ldots,\psi_n\}$, $\forall i\in\{1\ldots n\}. pi\in A_p$ & add $p$ & add $p$\\
\hline
$\phi|_{p}=\vee\{\psi_1,\ldots,\psi_n\}$, $\exists i\in\{1\ldots n\}. pi\in A_p$ & add $p$ & add $p$\\
\hline
$\phi|_{p}=\wedge\{\psi_1,\ldots,\psi_n\}$, $\exists i\in\{1\ldots n\}. pi\in E_p$ & (add $p$) & add $p$\\
\hline
$\phi|_{p}=\vee\{\psi_1,\ldots,\psi_n\}$, $\forall i\in\{1\ldots n\}. pi\in E_p$ & (add $p$) & add $p$\\
\hline
 \end{tabular}
  \end{center}
  \caption{The simplification rules for propagation in proof formulas}
  \label{fig:propagation}
\end{figure}
\end{footnotesize}

Propagation adds the positions of all newly added empty conjunctions (and disjunctions) to
$A_p$ ($E_p$) and propagates them through the formula. The set $tcs$ is empty after each
completed propagation step.

The set of all atoms $a_j$ that appear in $pf(g)$ is the relevant subset of the set $X$ of
unexpanded sequents of the proof graph.

\begin{lemma}
$\{\Gamma\in Seq\mid \ind{\Gamma}\in A_p\} = A$, $\{\Gamma\in Seq\mid \ind{\Gamma}\in E_p\} = E.$

\begin{proof}
The constructed proof formula is a direct representation of the proof graph. 
Repeated application of the rules from Figure~\ref{fig:propagation} iteratively approximates the
fixpoint $A$ ($E$) and stores the according positions in $A_p$ ($E_p$).

For more details, see the appendix.
\end{proof}
\end{lemma}

It follows that $\phi$ is provable iff $[]\in A_p$ and that if $[]\in E_p$, $\phi$ is not provable.\\

\noindent\textbf{Simplification:} The following are valid methods for the simplification of the proof formula:

Contract links correctly.

If $\psi_i=\psi_j$, then $\wedge\{\psi_i,\psi_j,\ldots\}$ may be replaced by $\wedge\{\psi_i,\ldots\}$,
remove $j$ from $I_k$. If $\psi_i=\psi_j$, then $\vee\{\psi_i,\psi_j,\ldots\}$ may be replaced by $\vee\{\psi_i,\ldots\}$,
remove $j$ from $I_k$. If $\phi_i=\psi_j$, $\psi_h=\wedge\{\phi_i,\phi_1,\ldots,\phi_n\}$, then $\vee\{\psi_h,\psi_j,\ldots\}$
may be replaced by $\vee\{\psi_h,\ldots\}$, remove $j$ from $I_{k}$.
If $\phi_i=\psi_j$, $\psi_h=\vee\{\phi_i,\phi_1,\ldots,\phi_n\}$, then $\wedge\{\psi_h,\psi_j,\ldots\}$
may be replaced by $\wedge\{\psi_j,\ldots\}$, $I_{k1}=\emptyset$, remove $h$ from $I_k$.

A homogenous disjunctive circle $\vee\{\vee\{\ldots\{\vee\{n_p,\ldots\},\ldots\}\ldots\},\ldots\}$ of length $j$ starting at
position $p$ may be replaced by $\vee\{\psi_i\mid i\in I\}\cup\{n_p\}$ where $I=\bigcup\{I_k\mid k=pl\}$ such that $l$ is a position
within the circle. Any reference to the removed positions must be set to $p$.

A homogenous conjunctive circle $\wedge\{\wedge\{\ldots\{\wedge\{n_p,\ldots\},\ldots\}\ldots\},\ldots\}$ of length $j$ starting at
position $p$ may be replaced by $\wedge\{\psi_i\mid i\in I\}\cup\{n_p\}$ where $I=\bigcup\{I_k\mid k=pl\}$ such that $l$ is a position
within the circle. Any reference to the removed positions must be set to $p$.\\

\noindent Notice that additionally to being more efficient than the previous variant, the new
methodology allows for a nice heuristic:
\begin{itemize}
\item Only those sequents $\Gamma$ which
have a representation in the current proof formula need to be explored (i.e. for $\Gamma$
to be expanded, $\ind{\Gamma}$ must be a valid position in the current proof formula). 
\end{itemize}
What about the unexpanded sequents which have no representation in the proof formula?
They either appear as sequent in a premise that already contains one
non provable sequent (s.t. the premise itself is not provable), or they are sequents
from a premise that is no longer needed to be shown to be provable (since there exists
already another provable premise for the same sequent). In both cases, the provability
(or non-provability) of the according sequents does not influence the overall outcome
of the proof.

\subsection {Removing Caching}

What if we propagate after every step and always 'empty' the graph? Does it remove the caching
while staying sound and complete and does it lead to PSPACE?

\section{Implementation}

\ldots

\section{Conclusion}  

\ldots

\section{Appendix (\& old stuff)}

\begin{definition}
A \emph{proof formula} is defined as follows for $i,j\in \mathbb{N}$:
\begin{quote}
$\mathcal{GF} \ni \psi,\phi := a_j \mid \top \mid \bot \mid \psi\wedge \phi \mid \psi \vee \phi \mid n_i$
\end{quote}

We also assign a number $i\in\mathbb{N}$ (its index) to any proof formula $\phi$ that we use and henceforth denote this
formula by $\phi^i$. An indexing function $\ind{\_}:(Seq+Prem)\rightarrow \mathbb{N}$ returns the index of
the representing proof formula for either a sequent $\Gamma$ or a premise $\Sigma$. This function is assumed
to be initially undefined for any sequent or premise (except for $\{\phi\}$).

A \emph{graph formula} $\vartheta$ for an index set $I$
is just a set $\vartheta=\{\psi^i \mid i\in I \}$. A \emph{wellformed graph formula} $\vartheta$ for an index set $I$
is a graph formula s.t. for any node operator $n_j$ that appears in any of the $\psi^i$, $j\in I$.
W.l.o.g. we assume that there is one $\psi^{\ind{\{\phi\}}}$ representing our
initial sequent $\{\phi\}$ and that all other $\psi^j$ are reachable from $\psi^{\ind{\{\phi\}}}$.
\end{definition}

Our intuition of a graph formula will be that for truth, falsedom, conjunction and disjunction it behaves as usual,
and the semantics of a node operator $n_i$ is just the semantics of the formula that the node points to (i.e. $\psi^i$).
An atom $a_j$ will represent the provability status of a sequent, i.e. if a sequent is known to be provable (not provable),
the according atom will in the end evaluate to $\top$ ($\bot$).

To achieve this, we assign two graph formulas $gf_1(g)=\{\psi_1^1,\ldots,\psi_1^n\}$ and 
$gf_2(g)=\{\psi_2^1,\ldots,\psi_2^m\}$ to any proof graph $g=(A,E,U,X,L_1,L_2)$ as follows:

Given a formula $\phi$, we initialize $gf_1(g_0)=gf_2(g_0)=\{a_{\ind{\{\phi\}}}\}$ and continue to expand the graph formula
in step with our expansion of the proof graph. We also initialize the sets $A_{seq}(g_0)=\emptyset$, $E_{seq}(g_0)=\emptyset$,
$A_{prem}(g_0)=\emptyset$ and $E_{prem}(g_0)=\emptyset$.

Say we expand sequent $\Gamma$, s.t. $g\stackrel{\Gamma}\rightarrow_E g'$ for $g'=exp(g,\Gamma)=(A',E',U',X',L_1',L_2')$.
This means that there are $n$ premises to $\Gamma$ (i.e. $|AP(\Gamma)|=n$) and for each $\Sigma \in AP(\Gamma)$,
$(\Gamma,\Sigma)\in L_1'$. Also, for each $\Sigma \in AP(\Gamma)$, there is an $m$ s.t.
$|\{\Gamma'\mid (\Sigma,\Gamma')\in AS(\Gamma)\}|=m$ and each such $(\Sigma,\Gamma')$ is contained in $L_2'$.
In other words, $\Gamma$ has $n$ premises, each premise consists of a number of sequents and the according
transistions are stored in $L_1'$ and $L_2'$.

Then the expansion of the first graph formula is realized as follows (where we obtain a new graph formula
$gf_1'(g)$ from $gf_1(g)$ by substituting parts of the graph formula):
\begin{enumerate}
\item If $n=0$, then there is no rule that may be applied to $\Gamma$, i.e. $\Gamma$ is not provable. The expanded
graph formula is obtained substituting $a_{\ind{\Gamma}}$ with $\bot$.
\item If $n=1$, then there is just one rule that may be applied to $\Gamma$. This allows us to directly continue without adding
a new proof formula, that is, we replace $a_{\ind{\Gamma}}$ with $b_{\Sigma}$ for
the one $\Sigma\in AP(\Gamma)$ s.t. $b_{\Sigma}$ represents $\Sigma\in AP(\Gamma)$ and is defined as follows:
\begin{itemize}
\item if $\ind{\Sigma}$ is yet undefined (i.e. the premise does not yet appear in the
proof graph), then $b_{\Sigma}=\psi^{\ind{\Gamma}}$ and $\ind{\Sigma}:=\ind{\Gamma}$, else
\item $b_{\Sigma_j}=n_{\ind{\Sigma_j}}$.
\end{itemize}
\item If $n>1$, we globally substitute $a_{\ind{\Gamma}}$ by the
\emph{disjunction} of all $b_{\Sigma_j}$ where each $b_{\Sigma_j}$ stands for the premise of one of the $n$
possible rule applications to $\Gamma$, s.t. $b_{\Sigma_j}$ represents $\Sigma_j\in AP(\Gamma)$ and is defined as follows:
\begin{itemize}
\item if $\ind{\Sigma_j}$ is yet undefined (i.e. the premise does not yet appear in the
proof graph), then $b_{\Sigma_j}=n_k$ for a new $k\notin I$, and set $\ind{\Sigma_j}=k$, else
\item $b_{\Sigma_j}=n_{\ind{\Sigma_j}}$.
\end{itemize}
\end{enumerate}
We continue with the following case distinction for each new $\psi^k$:
\begin{enumerate}
\item If the according $m=0$, then the premise contains no sequent and is hence provable. The according $\psi^k$ is
directly defined as $\top$.
\item If $m=1$, then $\Sigma_j$ contains just one sequent. This allows us to directly continue without adding
a new proof formula. This means that $\psi^k$ is defined as follows
(for the one $\Gamma'$ s.t. $(\Sigma_j,\Gamma')\in L_2'$).
\begin{itemize}
\item If $\ind{\Gamma'}$ is yet undefined (i.e. the sequent does not yet appear in the
proof graph), then $\psi^k=a_{\ind{\Sigma_j}}$ and $\ind{\Gamma'}:=\ind{\Sigma_j}$, else
\item $\psi^k=n_{\ind{\Gamma'}}$.
\end{itemize}
\item If $m>1$, the new formula $\psi^k$ consists of the \emph{conjunction} of all the $m$ atoms
$a_{\Gamma_l}$ for sequents from $\Sigma_j$ (where $\Gamma_l \in \{\Gamma' \mid (\Sigma_j,\Gamma')\in L_2'\}$).
Each $a_{\Gamma_l}$ is defined as follows:
\begin{itemize}
\item if $\ind{\Gamma_l}$ is yet undefined (i.e. the sequent does not yet appear in the proof graph),
then $a_{\Gamma_l} = n_k$ for a new $k\notin I$ and set $\ind{\Gamma_{l}}=k$, else
\item $a_{\Gamma_l} = n_{\ind{\Gamma_l}}$.
\end{itemize}
\end{enumerate}
Finally, for every $n_k$ that was added in the last step, add a new proof formula $\psi_k$ to the graph formula,
s.t. $\psi_k=a_{\ind{\Gamma_l}}$.

This builds a graph formula $gf_1'(g)=gf_1(g')$ that completely characterizes the current state of the proof while
usually being smaller than the proof graph. However, we maintain the property that the
graph formula contains separate proof formulas for any different combination of sequents and
premises that has yet been encountered (even though the provability of the formula in question may not
depend on many of these combinations).

\begin{definition}
The \emph{contraction} of a graph formula $\phi=\{\psi_1,\ldots,\psi_n\}$, $I=\{1,\ldots,n\}$ is the graph formula that is obtained from 
$\phi$ as follows:
\begin{quote}
For every $i\in I$, s.t. $\psi^i=n_j$ for some $j$, replace any occurance of $n_i$ by $n_j$, remove $\psi^i$ from $\phi$.
\end{quote}
\end{definition}

\begin{example} The graph formula $gf_1(g)$ is the contraction (with contracted circles) of a graph formula that is isomorphic to $g$.
\end{example}

If we try to further reduce the size of the graph formula while preserving only the information needed to
directly show (or refute) provability, we can do better. We construct an even smaller graph formula by
expanding as follows:

Replace $a_{\ind{\Gamma}}$ with $b_{\ind{\Sigma_1}}\vee\ldots\vee b_{\ind{\Sigma_n}}$ where
$AP(\Gamma)=\{\Sigma_1,\ldots,\Sigma_n\}$ (if $n=0$, replace $a_{\ind{\Gamma}}$ with $\bot$).
For any premise $\Sigma_i$ to $\Gamma$ we now distinguish three cases:

\begin{enumerate}
\item $\Sigma_i\in A(g)_{prem}$ ($\Sigma_i\in E(g)_{prem}$), i.e. the premise has been shown to be provable (not provable). Then replace
$b_{\ind{\Sigma_i}}$ with $\top$ ($\bot$).
\item $\psi^{\ind{\Sigma_i}}$ is contained in $gf_1(g')$ but not in $gf_1(g)$. Then leave $b_{\ind{\Sigma_i}}$ untouched.
\item $\psi^{\ind{\Sigma_i}}$ is contained in $gf_1(g')$ and in $gf_1(g)$ and also $\Sigma_i\notin E(g)_{prem}\cup A(g)_{prem}$, i.e. the
premise was encountered before and is not decided yet. Then take the flattening of the whole sub graph formula from $gf_1(g')$
at $\ind{\Sigma_i}$ and add it to the current graph formula. Replace $b_{\ind{\Sigma_i}}$ by $n_{\ind{\Sigma_i}}$.
Simplify this (and make any other appearance of the formula point to $\ind{\Sigma_i}$).
\end{enumerate}

Next, replace every remaining $b_{\ind{\Sigma_i}}$ with $a_{\ind{\Gamma_1}}\wedge\ldots\wedge a_{\ind{\Gamma_m}}$ where
$\Sigma_i$ is assumed to contain just the $m$ sequents $\Gamma_1,\ldots,\Gamma_m$ (if $m=0$, replace the
according $b_{\ind{\Sigma_i}}$ with $\top$). We distinguish the cases:

\begin{enumerate}
\item $\Gamma_i\in A(g)_{seq}$ ($\Gamma_i\in E(g)_{seq}$), i.e. the sequent has been shown to be provable (not provable). Then replace
$a_{\ind{\Gamma_i}}$ with $\top$ ($\bot$).
\item $\psi^{\ind{\Gamma_i}}$ is contained in $gf_1(g')$ but not in $gf_1(g)$. Then leave $a_{\ind{\Gamma_i}}$ untouched.
\item $\psi^{\ind{\Gamma_i}}$ is contained in $gf_1(g')$ and in $gf_1(g)$ and also $\Gamma_i\notin E(g)_{seq}\cup A(g)_{seq}$, i.e. the
sequent was encountered before and is not decided yet. Then take the flattening of the whole sub graph formula from $gf_1(g')$
at $\ind{\Gamma_i}$ and add it to the current graph formula. Replace $a_{\ind{\Gamma_i}}$ by $n_{\ind{\Gamma_i}}$. Simplify
this (and make any other appearance of the formula point to $\ind{\Gamma_i}$).
\end{enumerate}

The constructed graph $gf_2(g)$ fully represents the current state of the proof (as stored in graph $g$)
while being minimal.

\begin{definition}
The \emph{flattening} of a graph formula $\phi=\{\psi_1,\ldots,\psi_n\}$, $I=\{1,\ldots,n\}$ is the graph formula that is obtained from 
the concraction of $\phi$ as follows:
\begin{quote}
For every $i\in I$, s.t. $n_i$ occurs only one time in $\phi$, replace $n_i$ by $\psi^i$, remove $\psi^i$ from $\phi$.
\end{quote}

\end{definition}

\begin{example} The graph formula $gf_2(g)$ is the flattening (with flattened circles) of a graph formula that is isomorphic to $g$.
\end{example}

Contraction of circles: Whenever a link $n_i$ to a proof formula $\psi^i$ is added to the graph formula $gf_1(g)$ s.t.
$\psi^i$ appears in one of the paths from the formula $\psi^{\ind{\{\phi\}}}$ (that 
represents $\{\phi\}$) to the proof formula that is currently being expanded, it may be possible to contract the
circle in the graph formulas.

In more detail: Let $\psi^1,\ldots,\psi^m$ be a circle of $m$ formulas from $gf_1(g)$,
i.e. $\psi^i$ contains at least one occurance of $n_{i+1}$ for $0<i<(m-1)$ and $\psi^m$ contains $n_1$.

If the circle is homogenous, i.e. either for all $0<i<m$, $\psi^i=n_j\vee\ldots\vee n_k$ (disjunctive
homogenous circle) or for all $0<i<m$, $\psi^i=n_j\wedge\ldots\wedge n_k$ (conjunctive homogenous circle),
we may collapse the circle in $gf_1(g)$ as follows: 

The indexing set $\mathcal{I}$ of a circle is defined as follows: $\mathcal{I}=\bigcup\limits_{0<i<m}{ind(\psi^i)}\setminus\{1,\ldots,m\}$,
where $ind(n_j\vee\psi):=\{j\}\cup ind(\psi)$, $ind(n_j\wedge\psi):=\{j\}\cup ind(\psi)$.

A homogenous circle may be replaced by a formula $\psi^c$.

For disjunctive homogenous circles, $\psi^c:=n_c\vee\bigvee\limits_{i\in\mathcal{I}}n_i$,
For conjunctive homogenous circles, $\psi^c:=n_c\wedge\bigwedge\limits_{i\in\mathcal{I}}n_i$.

Replace every occurance of $n_i$ (for $0<i<m$) in the current graph formula by $n_c$.

Flattening of circles: For $gf_2(g)$, define $\psi^c$ as the flattening of the whole sub graph formula from $gf_1(g)$ at the
position $i$. Replace every occurance of $\psi^c$ or $n_{i}$ by $n_c$.

\begin{definition}
Any sequent $\Gamma$ (premise $\Sigma$) s.t. $i\in\lbag\Gamma\rbag$ ($i\in\lbag\Sigma\rbag$) is said to have a
\emph{representation} in $\psi^i$.
\end{definition}

\begin{example}
all.ex in graph iff contraction eval to true
\end{example}


Now that we defined the expansion of the two graph formulas, we describe propagation in
the second graph formula, where we expand the sets $A(g)_{seq}$, $E(g)_{seq}$, $A(g)_{prem}$ and $E(g)_{prem}$:

The propagation is in fact a process of propositional simplification,
executed on the current $gf_2(g)$. The following propositional tautologies (and their symmetric
counterparts) are applied:
\begin{itemize}
\item $(\top\vee\phi)\rightarrow\top$ (only one premise needs to be provable). Also set $A(g')_{seq}=A(g)_{seq}\cup tops_{seq}$,
where $tops_{seq}$ are all the sequents that have a representation in $\top\vee\phi$. Also set $A(g')_{prem}=A(g)_{prem}\cup tops_{prem}$,
where $tops_{prem}$ are all the sequents that have a representation in $\top\vee\phi$.
\item $(\bot\vee\phi)\rightarrow\phi$ (not provable premises are removed from the formula). Also set $E(g')_{seq}=E(g)_{seq}\cup bot$ where $bot$
is the sequent that is represented by $\bot$. Also set $E(g')_{prem}=E(g)_{prem}\cup bot$ where $bot$
is the premise that is represented by $\bot$.
\item $(\top\wedge\phi)\rightarrow\phi$ (provable sequents are removed from the formula). Also set $A(g')_{seq}=A(g)_{seq}\cup top$ where $top$
is the sequent that is represented by $\top$. Also set $A(g')_{prem}=A(g)_{prem}\cup top$ where $top$
is the premise that is represented by $\top$.
\item $(\bot\wedge\phi)\rightarrow\bot$ (all sequents need to be provable). Also set $E(g')_{seq}=E(g)_{seq}\cup bots$, where $bots$ are all
the sequents that have a representation in $\bot\wedge\phi$. Also set $E(g')_{prem}=E(g)_{prem}\cup bots$, where $bots$ are all
the premises that have a representation in $\bot\wedge\phi$.
\item $(\psi^i=\top) \rightarrow (n_i \rightarrow \top)$. Trivial proof formulas may be collapsed. Also set $A(g')_{seq}=A(g)_{seq}\cup top$ where
$top$ is the sequent that is represented by $\top$. Also set $A(g')_{prem}=A(g)_{prem}\cup top$ where
$top$ is the premise that is represented by $\top$.
\item $(\psi^i=\bot) \rightarrow (n_i \rightarrow \bot)$. Trivial proof formulas may be collapsed. Also set $E(g')_{seq}=E(g)_{seq}\cup bot$ where $bot$
is the sequent that is represented by $\bot$. Also set $E(g')_{prem}=E(g)_{prem}\cup bot$ where $bot$
is the premise that is represented by $\bot$.
\end{itemize}

As soon as $\{\phi\}\in A(g)_{seq}$ or $\{\phi\}\in E(g)_{seq}$, we are finished.\\

The following lemma establishes the correctness of this optimization:

\begin{lemma}
Given a graph $g=(A,E,U,X,L_1,L_2)$,
\begin{quote}
$A=A(g)_{seq}$,\\
$E=E(g)_{seq}$.
\end{quote}
\begin{proof}
We treat the case of $A$, the case of $E$ is dual.

First we show that $A$ is contained in $A(g)_{seq}$.

We proceed by induction over all expansion and propagation steps of the graph:
$A'$ is a subset of $A'(g)_{seq}$ when the sets are initialized (induction base).
Expansion does not change $A$ or $A(g)_{seq}$. Propagation does change both sets.
So we assume, that $A$ is a subset of $A(g)_{seq}$ and have to show that the inclusion $A'\subseteq A'(g)_{seq}$
holds (where $g\rightarrow_P g'$).

So let $\Gamma\in A'$. By assumption, $\Gamma$ was added to $A$ during the last step of propagation.
So, $\Gamma$ is in the greatest fixpoint of $W^L$, which means that $\exists(\Gamma,\Sigma)\in L'_1.
\forall(\Sigma,\Gamma')\in L'_2.\Gamma'\in A'$. That is intuitively, either $\Gamma$ itself was expanded since the
last propagation or enough of the respective $\Gamma'$ have been set to true since the last propagation.

Assume, $\Gamma$ has been expanded. We show that $\Gamma$ is also added to $A(g)_{seq}$, distinguishing
several cases:
\begin{itemize}
\item The case that there is no premise to $\Gamma$ is impossible since $\Gamma$ is in the fixpoint.
\item Any premises to $\Gamma$ that are already known to be provable (i.e. all contained sequents are in $A$) or known to be 
not provable (i.e. one contained sequent is in $E$), are also contained in $A(g)_{prem}$ (or $E(g)_{prem}$).
\item For any premises to $\Gamma$ that have been encountered before but are still undecided, existence of a node that represents the
premise is created.
\item For any premises to $\Gamma$ that have not been encountered before, a disjunct that represents the
premise is created.
\end{itemize}
For each premise $\Sigma_j$ to $\Gamma$, we distinguish upon the contained sequents:
\begin{itemize}
\item If $\Sigma_j$ contains no sequents, the according node/disjunct is set to $\top$ s.t. $\Gamma$ is added to $A(g)_{seq}$ upon propagation.
\item Any already decided sequents $\Gamma'\in A$ (or $\Gamma'\in E$) from $\Sigma_j$ are by assumption also contained in
$A(g)_{seq}$ (or $E(g)_{seq}$).
\item For any sequents in $\Sigma_j$ that have been encountered before but are still undecided, existence of new node that represents the
sequent is ensured.
\item For any sequents in $\Sigma_j$ that have not been encountered before, a conjunct that represents the
sequent is created.
\end{itemize}

Now, since we know that $\exists(\Gamma,\Sigma)\in L'_1.
\forall(\Sigma,\Gamma')\in L'_2.\Gamma'\in A'$ (i.e. we know that there is either an empty premise to $\Gamma$ or
a premise to $\Gamma'$ whose sequents are all in $A'$. For each premise, we have a disjunct, representing it. For each
contained sequent, we have conjunct, representing it. Since there is a premise to $\Gamma'$ whose sequents are all in $A'$,
by assumption we know that there is one disjunct whose conjuncts all simplify to $\top$.

Circles?

For the other direction, we show that $A(g)_{seq}$ is contained in $A$.

Again we proceed by induction, initialization is the base case, expansion does not change the sets.
If propagation adds a sequent $\Gamma$ to $A(g)_{seq}$, then one of the following three rules was applied,
s.t. $\Gamma$ had a representation in the simplified formula:

\begin{enumerate}
\item $(\top\vee\phi)\rightarrow\top$ (only one premise needs to be provable). Also set $A(g')_{seq}=A(g)_{seq}\cup tops_{seq}$,
where $tops_{seq}$ are all the sequents that have a representation in $\top$. Also set $A(g')_{prem}=A(g)_{prem}\cup tops_{prem}$,
where $tops_{prem}$ are all the sequents that have a representation in $\top$.
\item $(\top\wedge\phi)\rightarrow\phi$ (provable sequents are removed from the formula). Also set $A(g')_{seq}=A(g)_{seq}\cup top$ where $top$
is the sequent that is represented by $\top$. Also set $A(g')_{prem}=A(g)_{prem}\cup top$ where $top$
is the premise that is represented by $\top$.
\item $(\psi^i=\top) \rightarrow (n_i \rightarrow \top)$. Trivial proof formulas may be collapsed. Also set $A(g')_{seq}=A(g)_{seq}\cup top$ where
$top$ is the sequent that is represented by $\top$. Also set $A(g')_{prem}=A(g)_{prem}\cup top$ where
$top$ is the premise that is represented by $\top$.
\end{enumerate}

In the first case, $\Gamma$ was either represented by $(\top\vee\phi)$ or by $\top$. By induction, we only treat the first variant.
We know that the disjunction was introduced due to the fact that there is at least one provable premise to $\Gamma$ (the one having a
representation in top). Obviously, this premise contains only provable sequents s.t. $\Gamma\in A'$ as required.

In the second case, $\Gamma$ was represented by $\top$. How could it be represented by $\top$? Either because there just one empty premise to it
or since there were other rules applied before (induction).

In the third case, $\Gamma$ was represented $\top$. How could it be represented by $\top$? Either because there just one empty premise to it
or since there were other rules applied before (induction).
\end{proof}
\end{lemma}

Intuitively, the fixpoint computation is just the same as constructing the complete
graph formula every time and then simplifying it (but using previously known simplifications stored in $A/E$).

The new method gets rid of constructing the complete formula again and again by storing it
and just \emph{expanding} it with all the sequents which were added since the last propagation
step.

In more detail, $A$ equals the set of sequents which are represented by atoms in the
(unsimplified but flattened) graph formula which may be set to $\top$ by a successful leaf or by simplification.
Correspondingly, $E$ equals the set of sequents which are represented by atoms in the
(unsimplified but flattened) graph formula which may be set to $\bot$ by an unsuccessful leaf or by simplification.\\

\bibliographystyle{myabbrv}
\bibliography{coalgml}


\end{document}
